\input cantar.sty
\zacatek{zpěvník}{českých písniček}


\song{vrány}{}

/a9 když mlha k večeru se zvedá
/ a slunce dlouhý stíny naposledy
kreslí, /F kreslí mezi /a mech
a vůně /C hořícího /G jehličí
to jak /a na ohýnku /d čaj si vaříš
/a do písniček /e provoní ti /a9 dech \S

opřeni o stromy a na kamenech
sedíme a zpíváme o rose na kolejích
o kouzlu všech tuláckých rán
kdy člověk nemusí nic jenom prostě být
a koukat, jak je pěknej svět
a hlavně není sám

\R písnička /C pohladí ti /G otevřený /a rány
   poslední /d dobou ňák je toho na nás /a moc
   kdekdo /C sbírá jen jak v /G polích setbu /a vrány
   a tady /d plameny ti dá/E vaj dobrou /a noc **

a dlaně do potoka ponoříš
pohladíš uhlazenej kámen
nevím jak vám mně tohle připomíná křest
a ruku na čelo si položíš
a kapky po tváři ti stékaj
jak někdy pěkně umí život vonět, kvést

\r písnička...

když mlha k večeru se zvedá
a slunce dlouhý stíny naposledy kreslí
kreslí mezi mech...



\song{kroužek}{}

/E rozhozený vlasy v trávě /A chvíli spí
nohy /H7 unavený z těžkejch /E bot
kdo /A ví, co zd/E á se jí, /A co se jí zd/E á
co tam /Fis za víčkama schováno m/H7 á \S

na zápěstí z kůže kroužek skrývá dlaň
tak nechte ji tu chvilku spát
ať pocit jen má bezpečí odevzdaná
šla s námi přes den všechno stejně jak kluk

\R podívej /A usmívá se /E tiše vždyť sp/H7 í
   co kdo z nás /A vlastně jeden /E o druhým v/H7 í
   tak pojďte /E hrát **

ať probudí ji písnička ať zpívá s ní
ještě než otevře oči hraj
počkej ta bude koukat až zjistí kde je
do kolen opře hlavu vykulená

\r podívej...

rozhozený vlasy v trávě chvíli spí
nohy unavený z těžkejch bot...



\song{hance}{}

ještě /D máš pod kůží /G smích
ještě /D pod mou hlavou ruku /A máš
tiše /D pod závěsy vklouzl k nám /G den
zeptat se /D kolik si mě ještě /A dáš
a než ti /G pod nos sedne z verandy /A kouř
pomalu /D sprcha smývá dnešní /G noc
ne nejsem /D nešťastnej vždyť mám tě
/G mám tě rád a /A to je tak /D moc \S

žádný velký slova sliby a nic
objetí něha pomoc činy jsou víc
plujeme po řece kde není snad břeh
myšlenkám poroučíme jen žádnej spěch
tak hezky zvolna jednou ty jednou já
už se mi po nocích už o tobě zdá
tak jenom opatrně zvolna
po špičkách jak v lásce se má \s

ještě mám pod kůží smích
ještě od snídaně mléka chuť
a den na mě přes tvý mávnutí dých
dneska se stýskat bude buď jak buď
jakub si někde brumlá to svoje dádá
až ho oblíkneš tak pusu mu dej
a řekni, že mě máš ráda
že nám bude ve čtyřech hej \s

žádný velký slova sliby a nic
objetí něha pomoc činy jsou víc
plujeme po řece kde není snad břeh
myšlenkám poroučíme jen žádnej spěch
tak hezky zvolna jednou ty jednou já
už se mi po nocích už o tobě zdá
tak hezky opatrně zvolna
po špičkách jak v lásce se má \krat3



\song{podvod}{}

/e na dlani jednu z tvých řas, do tmy se /a koukám
/D hraju si písničky tvý, co jsem ti /G psal \S

je skoro /a půlnoc a z kostela zvon mi noc připo/e míná
půjdu se /a mejt a pozhasínám, co bude /H7 dál? \S

pod polštář dopisů pár, co poslala's dávám
píšeš, že ráda mě máš a trápí tě stesk \s

je skoro půlnoc...

\R chtěl jsem* to /a ráno
   / kdy naposled snídal jsem s tebou
   ti /e říct, že už ti nezavolám
   pro jednu pitomou /a holku
   pro pár nocí /D touhy
   podved jsem /G všechno o čem doma si /H7 sníš
   teď je mi to l/e íto *

kolikrát člověk může mít rád tak opravdu z lásky
dvakrát či třikrát, to ne i jednou je dost
je skoro půlnoc...

\rr

%\bye


\song{vlasy}{}

/D chvilku tu zůstaň ty vlasy z čela nedávej si
/G vítr už ví, kam je dát
a /A povídej mi o tom, co mezi námi může se stát
co pro tebe to znamená, když jeden chce tě mít r/A7 ád \S

chvilku tu zůstaň a beze slov mi namaluj
sem na dlaň jak vypadá smích
je to bílá rovná čára nebo srdíčko a na něm sníh
nebo smutek kterej nepláče jen kapky nosí na očích \s

chvilku tu zůstaň já zkusím se tě dotknout
uvidíš, co to s tebou udělá
jestli vůbec nic nebo husí kůže naskočí a proč
kouř z parníku uvidíš nebo slunce jako jabko a proč \s

chvilku tu zůstaň ty vlasy z čela ...




\Song{sedmikráska}{}

/a v řece plavou bílý lístky sedmikrásky někdo blízký
/d druhému se ptal zda na něj /a myslí vzpomí/E7 ná má nemá /a rád \S

smutně plavou květy vodou rozum pláče nad náhodou
nemá nemá řeko němá pospíchej ať nebolí to tak

\R /A chvíli si myslíš, že /A7 svět ztratil tvar
   /d prostor se zúžil na má dáti dal
   jak /G7 mokrá sirka připadáš si zbytečná
   jak /C zapálený trsy tr/E7 av, h/a m...  **

uschla kytka na kamenech vezmi ji a v knížce nech ji
pro vzpomínku u básniček ze kterých ti po večerech čet

\R chvíli si myslíš, že svět ztratil tvar
   prostor se zúžil na má dáti dal
   jak mokrá sirka připadáš si zbytečná
   jak zapálený trsy trav, jé... **

v řece plavou bílý lístky sedmikrásky někdo blízký
druhému se ptal zda na něj myslí vzpomíná má nemá rád \s



\song{proužek}{}

ten /E mír /H7 když za kopcem /A sví/E tá
když /A vystřídá se s /H7 nocí /E den
a /E proužek dýmu z /H7 dřeva slunce /A ví/E tá
/A z kytky motýl /H7 vylít /E ven \S

\R /E někdo škrtne sirkou někdo /A zpí/E vá
   /A někdo ještě zkouší /E spát
   u potoka /H7 do vody se /A dí/E vá
   /A jedna holka /H7 co mám /E rád **

vlasy sváže do uzlu a kouká
pak vodu dlaní rozčeří
stud jí sluší tak jako všem holkám
co na svou krásu nevěří \S

\r někdo...

z odpadlýho kousku starý kůry
srdce jsem jí udělal
a večer když už začly lítat můry
tajně jí ho do spacáku dal

\R někdo škrtne sirkou někdo zpívá
   někdo ještě zkouší spát
   u potoka do vody se dívá
   jedna holka co mám rád \krat3 **



\song{skládanka}{}

tak od dneška /A vím jak zavíráš /h oči když pusu ti /A dávám
jak otáčíš hlavu všechno už /h víš to, co bych /A chtěl
snad se mi /E7 zdáš nebo nechápu nic
snad jsem jen /A blázen co chtěl bych /D víc
ty můj /A vánku skládanko /h hezká z kouzelnejch /E chvil \S

do závěsů nutí se den a my máme půlnoc
cítím tvůj dech a trochu se bráním hladit tě víc
poprvé blízko blízko tě mám
a tvoje tvář je plná stop mých
začnu se bát jestli to moje rád ti neublíží \s

a už se mi stýská pitomej čas už musíš jít
uklidím okurky polštáře naposledy pusu ti dávám
ruce si chvíli nebudu mýt
pel z tvojí kůže nechci se vzít
tak zase zítra ty můj vánku milovaný \s

tak od dneška vím, jak zavíráš oči, když pusu ti dávám
jak otáčíš hlavu, všechno už víš to, co bych chtěl
snad se mi zdáš nebo nechápu nic
snad jsem jen blázen, co chtěl bych víc
ty můj vánku, skládanko hezká z kouzelnejch chvil
milova/A ná



\song{kočovníci}{}

jo, /a stáda hnát je pěknej job tak k /G jatkám stepí /a dlouhou
jak honákem se můžeš stát, já /G řeknu pravdu /a pouhou:
to /C chtěli skot pro /G queensland, do /C kempsy jak je /G zvykem
/a přišel jsem a vzali mě --- a /G tak jsem kočo/a vníkem

\R tak /C podej kolem /G láhev, /C podle starejch /G zvyků
   /a na zdraví se dneska pije s /G bandou kočo/a vníků **

už dobytek se počítá a všechno už se chystá
a muži sedí na koních a brzy hnem se z místa
sou zlí i dobrý tady, ir, němec v jednom šiku
advokáti, doktoři v tý bandě kočovníků \s

já takhle ráno stádo pás, kde tráva byla svěží
že přerazí mě ať du pryč, hned usedlík sem běží
já říkám:  ``tiše, brácho, já bouchám někdy mžikem
vždyť tu mluvíš s poctivým, cti dbalým kočovníkem!'' \s

prej dobytek nám zabaví! no tohle pro nás není
je moc těžký nás nachytat a přimět k zaplacení
a maso v krámě koupit je proti našim zvykům
zaběhlej se najde kus --- ten patří kočovníkům \s

což já bych někdy tričko štíp? jen ptej se kamaráda!
to leda, když de osadou a zrovna je den prádla
mě malý děti zlobí, maj vždycky plno křiku:
``mami všechno posbírej, jde banda kočovníků.'' \s

a děvče v sydney povídá: ``pryč nechoď, jsem tak sama!''
já na to: ``škoda, do sedla už nevejde se dáma.''
tak chlapci, táhnem zpátky, ať kdo chce co chce říká
někde poblíž marano prej hledaj kočovníka!



\song{kde našli muži hrob}{}

/d v pustinách dálné /(F) beznaděje -- tam /C našli muži /d hrob
/d bílé, kde žhavé /(F) slunce hřeje -- tam /C našli muži /d hrob
/F nad mrtvolou tichou jejich skrání /C západní vítr nemá stání
/F stráž s nimi drží v pusté pláni, tam /C našli muži /d hrob \S

tam, kde se smrtka s létem snoubí, tam našli muži hrob
tam, kde je léto symbol zhouby, tam našli muži hrob
ve slaném buši slunce bělí lebky těch, kteří neuspěli
dingo, kde jenom štěkne bdělý, tam našli muži hrob \s

ve žlutých vodách s divým během -- tam našli muži hrob
ve zrádných tůních s měkkým břehem -- tam našli muži hrob
perlivý pruh se matně blýská za vodní brázdou ptakopyska
živý hlas nikde nezastýská -- tam našli muži hrob \s

k domovu ruce rozpínají -- tak našli muži hrob
mír smutný ve svých tvářích mají -- tak našli muži hrob
jiného směru pro ně není, věčně jsou v domov ponořeni
v mateřské sladké konejšení, tak našli muži hrob \s

nocí se mrtvý jezdec žene, opouští muži hrob
stádo je hrůzou poplašené, opouští muži hrob
v divokém rytmu nohy buší, dobytek vždycky mrtvé tuší
v oživlých tělech není duší -- opouští muži hrob \s

zeptej se hlídky, ta ti poví: ``vždycky je blízko hrob.''
volají mrtví noční sovy -- z míst, kde je vždycky hrob
smích slyší hlídka v mlžné páře, vidívá kradmé bledé tváře
šedivé stíny za hvězd záře blízko, kde vždy je hrob \s

strádání, dřina, věčná sucha -- tak najdou muži hrob
v dalekých farmách vydřiducha -- tam najdou muži hrob
vousatí muži, zelenáči, padnou a ani nezapláčí
zaroste všechno do bodláčí -- tak najdou muži hrob \s

vydřiduch v klubu pije whisky, co je mu lidskej hrob
zlatý svůj řetěz laská stisky -- co je mu lidskej hrob
na jeho článcích lidská jména měla by tam být vyražena
článek je život -- to je cena, tak najdou muži hrob \s



\song{koleje v prérii}{}

sta /e mil jdou koleje v prérii
jak /G stříbrný šíp míří dál
a /e tak se patrick najmout dal
a /H7 dřel se s touhle /e dráhou

\R a /e den co den a den co den a hej a hou
   a /G den co den a den co den a hej a hou
   a /e den co den a den co den a hej a hou
   on /H7 dřel se s touhle /e dráhou **

a rval se statečně s přírodou
i s šéfem kvůli výplatě
vzdor míval v tváři vousaté
když dřel se s touhle dráhou

\r a den co den...

kdo vůbec dnes ještě může znát
co tento úsek útrap stál
když tenkrát patrick práci vzal
a dřel se s touhle dráhou \s

\r a den co den...



\song{blízko little big hornu}{}

tam, /a kde leží little big horn je indiánská zem
tam přijíždí generál custer /d se svým praporem
mo/a drý kabáty jezdců, stíny /d dlouhejch karabin
a z /a indiánskejch signálů po /A nebi letí dým \S

\R říkal to jim bridger: ``já měl jsem v noci /E sen
   pod sedmou kavalerií jak krví rudne /A zem
   kmen siouxů je statečný a /A7 dobře svůj kraj /D zná''
   proč /E custer neposlouchá ta /A slova varov/a ná **

tam blízko little big hornu šedivou prérií
táhne generál custer s sedmou kavalerií
marně mu stopař bridger radí: ``zpátky povel dej
jedinou možnost ještě máš --- život si zachovej'' \s

tam blízkou little big hornu se vznáší smrti stín
padají jezdci z koní, hřmí střely z karabin
límce modrejch kabátů barví krev červená
kmen siouxů je statečný a dobře svůj kraj zná \s

pak všechno ztichlo a jen tam-tam duní nad krajem
v oblaku prachu mizí siouxů vítěznej kmen
cáry vlajky hvězdnatý po kopcích vítr vál
tam uprostřed svých vojáků leží i generál \s

\R říkal to jim bridger: ``já měl jsem v noci sen
   pod sedmou kavalerií jak krví rudne zem
   kmen siouxů je statečný a dobře svůj kraj zná!''
   proč custer neposlouchal ta slova varovná **



\song{oranžový expres}{}

/D už z rodnýho ranče vidím jen komín a stáj
hej, /G už z rodnýho ranče vidím jen komín a /D stáj
a rychlík v /A barvě pomeranče už mě veze, tak good /D bye \S


tak jako světla herny mě stejně vždycky rozruší
svit brzdařský lucerny, komáři jisker v ovzduší
a k tomu oranžovej expres mi houká do uší \s

{\i recitativ:} \s
\halign{#\hfill\qquad & # \cr
kam jedeš tuláku?   &nevím \cr
na new york?        &nevím \cr
nebo na nashville?  &nevím, mě stačí, když slyším, jak ty \cr
            &pražce drncaj duda, duda, duda, ... \cr}\s

sedím si na uhláku a vyhlížím přes okraje
ve svým tuláckým vlaku šmátrám po lahvi tokaje
píseň oranžovýho vlaku si zpívám do kraje \s

už z rodnýho ranče nevidím komín a stáj
hej, už z rodnýho ranče nevidím komín a stáj
a rychlík v barvě pomeranče sviští na new york --- good bye!



\Song{hlídej lásku skálo má}{}

/D jak to v žití chodívá láska k lidem /A přichá/D zí
přijde jen se rozhlédne a zase /G odchá/D zí
mě potkala ve skalách šel bosej, /G jenom /D tak
měl svý džíny vybledlý, na zádech /A starej /D vak \S

\R /D hlídej lásku skálo má, než se s ránem /A vytra/D tí
   čeká na nás těžká pouť, až se s ránem /G vytra/D tí
   mezi lidma je těžké plout, až se s ránem /G vytra/D tí
   víš, co umí člověk, pojď, než se s ránem /A vytra/D tí **

když mě ahoj povídá, úsměvem mě pohladí
prý jsem jedna z mála snad, co jí tulák nevadí
sedli jsme si na stráni dole zpíval řeky proud
den zmizel za obzorem, stín skryl náš tichej kout



\song{zapomeň lásko}{}

/g zapomeň /d lásko /g na to, že j/A7 á
jsem /D jenom ješitnej /D7 chlap
že /g věčnej str/d ach, abys /g byla jen /A7 má
dělá /D ze mě horšího /D7 snad \S

\R jsi /g milá, takže jako mrak jsou /A7 chvíle, kdy jsem sám
   kdy /D7 necítím tvůj dech na ústech /g svých
   jsi hezká, takže jako mrak jsou /A7 tváře těch, co znám
   a /D7 pokřik vran je každej jinej /g smích **

snad  ďábel sám mi napovídá
když pláčeš kvůli mně
jsem trdlo, který zapomíná
jak málo lidí jako ty je \s

každej den, i když mračím se snad
provázíš můj každičkej krok
v myšlenkách, slovech pořád jen rád
ozývá se lásko vždyť já jsem cvok



\song{šel průzračnou nocí}{}

šel /a průzračnou nocí a táhl z něj rum
/ tak pět křížků moudrosti měl
/G někde se rozdat /a chtěl
když /a opustil ves i poslední dům
/ tak oheň v dálce uviděl
/G a tam se rozdat /a šel \S

\R jste /C proradná /G banda /C bláznivejch /G lidí
   /E na který my se /a dřem
   /C člověk se /G za vás /C červená, /G stydí
   /E a diví co nosí d/a en **

tam uprostřed trampů byl i jeho kluk
a ten se pomalu zved
táto, prosím tě, mlč!
když dostal facku, tak neřek´ ani muk
jen volíz rozbitej ret
táto, prosím tě, mlč! \s

\R kdo z nás je proradná banda bláznivejch lidí
   a kdo se na nás dře?!
   kdopak se za nás červená stydí
   a diví co nosí den! **

šel průzračnou nocí a táhl z něj rum
tak pět křížků moudrosti měl



\Song{růže z papíru}{}

do tvých /d očí jsem se zbláznil a teď /E nemám, nemám /g klid
/a hlava třeští, asi tě mám /d rád
stále někdo říká, vzbuď se, /E věčně trhá /g nit
/a studenou sprchu měl bych si /d dát \songgg

\R /D7 na pouti jsem vystřelil /g růži z papíru
   /C7 dala sis ji do vlasů, kde /A7 hladívám tě já
   v /d tomhle smutným světě jsi má /g naděj na víru
   že /a nebe modrý ještě smysl /d má **

přines jsem ti kytku, no co koukáš, to se má
tak jsem asi jinej, teď to víš
možná trochu zvláštní v dnešní době, no tak ať
třeba z ní mou lásku vytušíš



\Song{toronto}{}

na /d břehu řeky sázavy je /A7 tichý, klidný kraj
na /d břehu řeky modravý tam /g osadu prej maj
když měsíc kraj ten ozáří a /d peřej zašumí
tu kamarádi z osady si s /A7 lesem rozu/D mí \S

\R každý, kdo patříš mezi nás, /G víš co je kama/D rád
   každý, kdo musel odejít, tak /E vrátil by se /A rád
   /G toronto má osado svý /D krajem kouzelným
   říčko moje sázavo jen /G u vás je /A7 dobře, to já /{D A d} vím **

ozvěna hlasy navrátí, když soumrak přišel tam
tu kamarádi z osady nechtějí být nikdo sám
oheň zaplál černou tmou a píseň lesem zní
mraky letí oblohou, jen stromy tiše spí \s

na břehu řeky sázavy akáty zašumí
tu kamarádi z osady i skálám rozumí
když vláček nocí zahouká, jdem všichni domů spát
jen vlajka s listem javoru tu bez nás bude vlát



\song{tisíc mil}{}

v nohách /G mám už tisíc /e mil
stopy /a déšť a vítr /C smyl
a můj /a kůň a já jsme /D cestou znave/G ni \S

těch tisíc mil, těch tisíc mil
má jeden směr a jeden cíl
bílej dům, to malý bílý stavení \s

je tam stráň a příkrej sráz
modrá tůň a bobří hráz
táta s mámou, který věřej dětským snům \s

těch tisíc mil, těch tisíc mil
má jeden směr a jeden cíl
jeden cíl, ten starej známej bílej dům \s

v nohách mám už tisíc mil
teď mi zbejvá jen pár chvil
cestu znám a ta se tam k nám nemění \s

těch tisíc mil, těch tisíc mil
má jeden směr a jeden cíl
bílej dům, to malý bílý stavení \s

kousek dál a já to vím
uvidím už stoupat dým
šikmej štít nad střechy čnít k nebesům \s

těch tisíc mil, těch tisíc mil
má jeden směr a jeden cíl
jeden cíl, ten starej známej bílej dům



\song{vlajka}{}

vše tone v /d snách a život kolem /A7 ztich
jen dole v tmách kol ohně slyšet /d smích
tam srdce všem jen spokojeně /g zabuší
/d z písniček známých /A7 vše jistě vytu/D ší: \S

\R /D vlajka /G vzhůru /A letí, k radosti svých /D dětí
   hned se s /Fis mraky /G snoubí
   vlát bude /D zas, než mládí /A7 čas opustí /D nás **

po létech sám až zabloudíš v ten kraj
a staneš tam, kde býval kdys tvůj ráj
vzpomeneš chvil těch, kterés míval tolik rád
tak jako kdysi ozvěnou slyšíš hrát \s

\r vlajka...



\Song{eldorádo}{}

v dálných dálkách zámoří
ční prý zlaté pohoří
příchozího pohostí
nádherou a hojností \s

dík těm svůdným pověstím
zástupy šly za štěstím
chátra i ti bohatí
a vírou, že se vyplatí: \s

\R jít a hledat eldorádo
   zbavené vší bídy člověčí
   jít a hledat eldorádo
   kde je láska, mír a bezpečí **

báchorce té uvěří
dávno už jen někteří
spíš, než zlatonosný štít
nám dnes rozum káže  jít \s

\r jít a hledat...



\song{jožin z bažin}{}

/e pa, pa, pa, papada, ...
/e jedu takhle tábořit /H7 škodou 100 na /e oravu
spěchám, proto riskuji, /H7 projíždím přes /e moravu
/D řádí tam to /G strašidlo, /D vystupuje z /G ba/H7 žin
/e žere hlavně pražáky, /H7 jmenuje se /e jo/D žin \S

\R /G jožin z bažin močálem se /D plíží
   jožin z bažin k vesnici se /G blíží
   jožin z bažin už si zuby /D brousí
   jožin z bažin kouše, saje, /G rdousí
   /C na jožina z /G bažin /D koho by to napa/G dlo
   /C platí jen a /G pouze /D práškovací leta/G dl/{H7 e} o **

pa, pa, pa, papada,...
projížděl jsem dědinou cestou na vizovice
přivítal mě předseda, řek mi u slivovice:
živého či mrtvého jožina kdo přivede
tomu já dám za ženu dceru a půl jzd \s

\r jožin...

říkám, dej mi, předsedo, letadlo a prášek
jožina ti přivedu, nevidím v tom háček
předseda mi vyhověl, ráno jsem se vznesl
na jožina z letadla prášek pěkně klesl \s

\R jožin z bažin už je celý bílý
   jožin z bažin z močálů ven pílí
   jožin z bažin dostal se na kámen
   jožin z bažin tady je s ním ámen
   jožina jsem dohnal, už ho držím johoho
   dobré každé lóve, prodám já ho do zoo **



\song{hejkal}{}

/a divný jekot po lesích se prohání
až v /d žilách tuhne /a krev a zuby /F cvakaj so/E s
/a utichá až u potoka pod strání
jó, /d takovýhle /a řvaní by /E nesnes ani /a pes
/d žhavý rudý /a oči a /d drápy /a krvavý
/F kosti chřastěj v rytmu kasta/E nět
a /d strašidelný /a vytí a /d skřeky chrapla/a vý
tak /F to je hejkal na to vemte /E jed \S

\R že u nás /C hej, /F hej, /C hejkal straší v lese
   jen ten, kdo něco snese, tam /G může v noci jít
   jeho /C hej, /F hej, /C hejkání se nese
   kaž/F dej se strachy /C třese, k ohni /F nesesedne/C me se
   \[ ne/F boť za boudou v /C lese zase /G hejkal začal /C výt. \]**

kdo z vás tady na hejkaly nevěří
ten může u nás přespat, až se zastaví
nevystrčí špičku nosu ze dveří
a bude jásat, že se dožil rána ve zdraví
jenom kalný oči a rysy ztrhané
kalhoty si bude muset prát
a děs a hrůza v hlase --- jó, to mu zůstane
až koktavě bude povídat \s

\r že u nás hej...



\song{hej šup námořníci}{}

/d táhlo už k večeru, na malém škuneru
/C dospěli k názoru, /A7 piráti na obzoru
/d že se ukázali, všichni hned mazali
/C do kajut pro zbraně /A7 rozjása/d ně
/C a kapitán si /F brousil zuby
/C tuhle čeládku /F že vyhubí --- /A7 jó
/d omrk situaci, rozdělil všem práci
/C plivnul si do dlaně, /A7 zařval na /d ně: inu! \S

\R /C hej, šup, do nich /F námořníci hned
   z /C vás který by zbled, /F nevydrží u nás
   /C hej šup, kdo se /F nám ukáže snad
   /C může ho to stát /F vaz
   jestli /B piráti nás nepřepe/F rou
   žraloci /G náramně se naže/C rou a proto:
   hej šup do nich /F námořníci hned
   /C před námi by zbled /F ďas **

jeden pirát klacek, dostal hned pár facek
spadnul přes palubu, natlouk si přitom hubu
a hned začalo rvaní, za velkýho řvaní
šel námořníků roj roznášet boj
sbalili pirátům kapitána
a potom práskla jen jedna rána --- {\b bum!!!}
šoupli ho do moře a za ním nahoře
vítězně dodali, halekali: inu! \s

\r hej, šup,...



\song{zelený pláně}{}

/a tam, kde zem /d duní /a kopyty /E stád
/a znám plno /G vůní, co /C dejchám je tak /G rád
/F čpí tam pot /G koní a /C voní tymi/a án
/d kouř obzor /G cloní, jak /C dolinou je /E hnán
/a rád žiju /d na ní, tý /a plá/E ni zele/a ný \S

tam, kde mlejn s pilou proud řeky hnal
já měl svou milou a moc jsem o ni stál
až přišlo psaní, ať na ni nečekám
prý: ``k čemu lhaní?'' a tak jsem zůstal sám
sám znenadání v tý pláni zelený \s

\R /F dál čistím chlív a /G7 lovím v oře/C ší
   /F jen jako dřív mě /G žití netě/C ší
   /a když hlídám stáj a slyším vítr /d dout
   /a prosím, ať jí /G poví, že mám v srdci /E troud **

kdo ví až se doví z větrnejch stran
dál že jen pro ni tu voní tymián
vlak hned ten ranní ji u nás vyloží
a ona k spaní se šťastná uloží
sem, do mejch dlaní, v tý pláni zelený



\song{starý honec krav}{}

po /d zasmušilé pustině jel /F starý honec krav
den /d tmavý byl vzduch ševelil ve /F větru stébel trav
ten /d honec k nebi pohlédl a v hrůze zůstal stát
když z /B rozedraných oblaků, viděl /d stádo krav se hnát \S

\R jipí já /F jé, jipí já /d jou, to /B přízraky táhnou /d tmou **

ten skot měl nohy z ocele a oči krvavé
a na bocích mu plápolaly cejchy řežavé
a oblohou se neslo jejich kopyt dunění
a za nimi jeli honáci až k smrti znavení \s

ti muži byli zsinalí a kalný měli zrak
a marně stádo stíhali jak mračno stíhá mrak
a proudy potu smáčely jim tvář a košili
a starý honec uslyšel jejich jekot kvílivý \s

tu honec zavolal a pravil: ``pozor dej
svou duši hříchu vyvaruj a ďáblu odpírej
bys nemusel pak po smrti se věčně věků kát
a nekonečnou oblohou to stádo stále hnát.'' \s



\Song{pirát henry}{}

/a tři bratři žili kdys v /F zemi /G skot/a ské
v domě zchudlém jim /D souzeno /e žít
ti /a kostkama /{G Fis F } metali, /a kdo musí jít, /D kdo musí /e jít
/F kdo z nich má /C na moři /G pirátem /a být \S

los padl a henry už opouští dům
ač je nejmladší, z nich vybrán byl
by koráby přepadal, na moři žil, jen na moři žil
své bratry z nouze tak vysvobodil \songgg

po dobu tak dlouhou, jak v zimě je noc
a tak krátkou jak zimní je den
už plaví se henry, když před sebou objeví loď pyšnou loď
napněte plachty a kanóny ven \s

čím kratší byl boj, tím byl bohatší lup
z vln už ční jenom zvrácený kýl
teď henry je boháč, když boháče oloupil, loď potopil
své bratry z nouze tak vysvobodil \s

do anglie staré dnes smutná zvěst
smutnou novinu dostane král
ke dnu klesla pyšná loď, poklady henry si vzal, on si vzal
třeste se moře on vládne tam dál



\Song{válka růží}{}

už /d rozplynul se /G hustý dým, /d derry down, hej /A down a down
/d nad ztichlým polem /g válečným, derry /{d A} down
jen /F ticho stojí /C kol ko/A lem a /d vítěz plení /B vlastní z/A em
je válka /d růží --- /g derry derry /A derry down, /d down \S

nečekej soucit od rváče, derry down, hej down a down
kdo zabíjí, ten nepláče, derry down
na těle mrtvé krajiny se mečem píšou dějiny
je válka růží, derry derry derry down, down \s

dva erby, dvojí korouhev, derry down hej down a down
dva rody živí jeden hněv derry down
kdo změří, kam se nahnul trůn, zda k yorkům nebo \rest{k lancasterům}
je válka růží, derry derry derry down, down \s

dva rody, dvojí korouhev, derry down, hej down a down
však hlína pije jednu krev, derry down
ať ten či onen přežije, vždy nejvíc ztratí anglie
je válka růží, derry derry derry down down



\song{za svou pravdou stát}{}

máš /a všechny trumfy mládí a /G ruce čistý /a máš
jen /d na tobě teď /a záleží na /E jakou hru se d/a áš --- /E musíš \S

\R /a za svou pravdou /E stát
   za svou pravdou /a stát
   musíš /a za svou pravdou /E stát
   za svou pravdou /a stát **

už víš, kolik co stojí, už víš, co bys rád měl
už ocenil jsi kompromis a párkrát zapomněls \s

\r že máš za svou pravdou stát...

už nejsi žádný elév, co prvně do hry vpad
už víš, jak s králem ustoupit a jak s ním dávat mat \s

\r tak hleď za svou pravdou stát...

teď přichází tvá chvíle, teď nahrává ti čas
tvůj  sok poslušně neuhnul a ty  mu zlámeš vaz \s

\r neměl za svou pravdou stát...

tvůj potomek ctí tátu, ty vštěpuješ mu rád
to heslo, který dobře znáš z dob, kdy jsi bejval mlád \s

máš všechny trumfy mládí a ruce čistý máš
jen na tobě teď záleží na jakou hru se dáš \s

\r musíš za svou pravdou stát...



\song{mlýny}{}

\R /G slyším mlýnský kámen, jak se otá/G7 čí
   /C slyším mlýnský kámen, jak se otá/G čí
   já /G slyším mlýnský kámen, /H7 jak se otá/e čí
   /C ot/D áč/G í, otá/D čí, otá/C čí **

ty mlýny /G melou celou /C noc, melou /G celý den
melou /C7 bez výhod a melou /G stejně všem
melou doleva, /C melou /G doprava
melou /A pravdu i lež, když zrovna /D vyhrává
melou /G otrok/C áře, melou /G otroky
melou /C7 na minuty, na hodiny, /G na roky
melou /H7 pomalu a jistě, ale /e melou /C včas
já už /G slyším /D7 jejich /G hlas \S

\r slyším...

ó, já chtěl bych aspoň na chvíli být mlynářem
pane já bych mlel, až by se chvěla zem
to mi věřte, uměl bych dobře mlít
já bych věděl, komu ubrat, komu přitlačit
ty mlýny čekají někde nad námi
až zdola zazní naše volání
až zazní jeden lidský hlas
no tak už melte, je čas! \s

\r slyším...



\song{žízeň}{}

když /C kapky deště buší na /F rozpá/a le/G nou /C zem
já toužím celou duší dát /F živou vodu /C všem
už v knize knih je psáno: bez /F vody /a nel/G ze /C žít
však ne každému je dáno z /F řeky pravdy /C pít \S

\R já mám /G žízeň, věčnou /{C C7} žízeň
   stačí ř/F íct, kde najdu vlá/C hu a zchladím /F žá/C hu páli/G vou
   já mám žízeň, věčnou /{C C7} žízeň
   stačí ř/F íct, kde najdu vlá/C hu a zmizí /{F C} žízeň **

stokrát víc než slova hladká jeden čin znamená
však musíš zadní vrátka nechat zavřená
mně čistá voda schází, mně chybí její třpyt
vždyť z moře lží a frází se voda nedá pít \s

jak vytékají říčky zpod úbočí hor
tak pod očními víčky se ukrývá můj vzdor
ten pramen vody živé má v sobě každý z nás
a vytryskne jak gejzír, až přijde jeho čas



\song{farao}{}

jen /D žhavý písek žhavá poušť a /G žhavý vzduch tam /D byl
pod žhavým sluncem chudý lid a /G řeka jménem /D nil
ti lidé byli otroci, nad /G nimi /C kara/G báč /D stál
nad karabáčem velekněz a /h nade /A všemi /D král \S

říkali mu farao a byl tak vysoko, že nevrhal ani stín
zato měl sýpky, sklepy, zlato a lesklá těla otrokyň \s
říkali mu farao a ten když rukou hnul duněly bubny k~obřadům
tančily kněžky hlavy padaly to aby nepad jeho trůn \s

\R \[ svůj /D trůn měl farao rád, měl /A rád, měl /D rád
   nechtěl ho nikomu dát to na žá/A dnej /D pád. \]**

říkali mu farao a musel všechno mít to víme z dávných knih
jeho vojska by nikdo nepřehlíd, kdyby šel týden kolem nich \s
říkali mu farao a z knížat pozemských ten nejmocnější byl
až jeden člověk hlavu zdvih a jeho vůli se postavil \s
říkali mu mojžíš a měl tu výhodu, že se nebál o svou moc
svému lidu slíbil svobodu a k útěku zvolil noc \s

\R \[ měl mojžíš svobodu rád, měl rád, měl rád
   nechtěl se svobody vzdát to na žádnej pád.\]**

říkali mu mojžíš a jako lodivod vedl k moři národ svůj
za ním se vojsko dalo na pochod a farao volal stůj! \s
co udělal mojžíš? holí udeřil, vlny se zvedly v mocný val
na jednom břehu mojžíš byl, na druhém zuřil král \s

\r svůj trůn měl farao rád...

jen žhavý písek žhavá poušť a žhavý vzduch tam byl
pod žhavým sluncem chudý lid a řeka jménem nil \s
ten příběh dávno odvál čas, jen krále neodvál
kus mojžíše je v každém z nás, tak jak to bude dál?



\song{už se nám čas krátí}{}

/a kde je, kde je /C sláva říma
/G kde je, kde je /C sláva /C7 říma
/F v dávných troskách /a dávno /e dřímá
/a tam, /C kde i /a světlo /G mizí v /{C E7} tmách \S

\[ kudy bloudí prázdná slova? \]
zjeví se a ztratí znova
tam, kde i světlo mizí v tmách \s

\R \[ už se, už se nám čas krátí. \]
   přijde den a půjdem spáti
   tam, kde i světlo mizí v tmách **

\[ kam se, kam se ztratí činy? \]
mrtvé činy z naší viny
tam, kde i světlo mizí v tmách \s

\[ kam se, kam se ztrácí touha? \]
únava nám zbývá pouhá
tam, kde i světlo mizí v tmách \s

\r už se...

\[ kde jsou, kde jsou velké lásky? \]
půvabné jak sedmikrásky
tam, kde i světlo mizí v tmách \s

\[ kde se, kde se skrývá pravda? \]
pálená i mrazem zvadlá
dál vábně voní věčnou tmou \s

už se, už se nám čas krátí
kde se, kde se skrývá pravda
přijde den a půjdem spáti
dál vábně voní věčnou tmou



\song{až se k nám právo vrátí}{}

chci /d sluncem být a ne planetou
/A7 až se k nám právo /d vrátí
chci /d setřást bázeň staletou
/A7 až se k nám právo /d vrátí \S

\R \[ /d já čekám dál, já /C čekám /F dál
   /B já čekám /A dál, /A7 až se k nám právo /d vrátí \]**

kam chci, tam půjdu, co chci budu jíst, až se...
a na co mám chuť, to budu číst, až se...
už nechci kývat, chci svůj názor mít, až se...
já chci svůj život jako člověk žít, až se... \s

\r já čekám...

chci klidně chodit spát a beze strachu vstávat, až se...
své děti po svém vychovávat, až se...
už se těším, až se narovnám, až se...
své věci rozhodnu si sám, až se... \s

\r já čekám...



\Song{loď dětí}{}

už /G plachty vzhůru letí, náš /C vítr začal /D7 vát
je /G slyšet písně /e dětí a /G hudba /D7 začla /G hrát \S

\R tak jen /C7 dál, milé děti, honem /G dál, milé děti
   tak jen /C7 dál, děti /A7 boží
   ta /G loď /e dost /a mí/D7 sta /G má! **

kdo má už po krk bídy, ten u nás najde kout
na lodi nejsou třídy a každý může plout \s

až v dáli nad vlnami zas vyhoupne se zem
tak těm, co plují s námi bude patřit všem \s

už plachty...



\song{trh ve scarborough}{}

/d příteli máš do /C scarborough /d jít
/F dobře /d vím, že půj/(G) deš tam /d rád
tam dívku /F najdi na market /C street
/d co chtěla /G dřív /B mou /C ženou se /d stát \S

vzkaž jí, ať šátek začne mi šít
za jehlu rýč však smí jenom brát
a místo příze měsíční svit
bude-li chtít mou ženou se stát \s

až přijde máj a zavoní zem
šátek v písku přikaž jí prát
a ždímat v kvítku jabloňovém
bude-li chtít mou ženou se stát \s

z vrkočů svých ať uplete člun
v něm se může na cestu dát
s tím šátkem pak ať vejde v můj dům
bude-li chtít mou ženou se stát \s

kde útes ční nad přívaly vln
zorej dva sáhy pro růží sad
za pluh ať slouží šípkový trn
budeš-li chtít mým mužem se stát \s

osej ten sad a slzou jej skrop
choď těm růžím na loutnu hrát
až začnou kvést tak srpu se chop
budeš-li chtít mým mužem se stát \s

z trní si lůžko zhotovit dej
druhé z růží pro mě nech stlát
jen pýchy své a boha se ptej
proč nechci víc tvou ženou se stát



\song{trudova žena}{}

/e umřela dneska /a trudovi že/H na a /Fis trud jí /{h D e H} nevěří
/H vstaň /e žen/D ko /G hore, srdén/D ko /e mo/H je
/a šak ťa nic /{C7 H7 e} nebolí \S

trudovu ženu už oblíkajú...
trudovu ženu do truhly kladú...
trudovu ženu na máry nesú...
trudovej ženě už vyzváňajú...
trudovu ženu na krchov vezú...
trudovu ženu do hrobu kladú...
trudovu ženu už pochovali...



\Song{domov na zemi}{}

/D jak léta jdou svět /G pro mně ztrácí /D glanc
všichni se rvou a /A7 duši dávaj' v šanc
a /D za pár šestáků vás /G prodaj' věřte /D mi
už víc nechci mít /G domov /D svůj /A7 na ze/D mi! \S

\R čas žádá svý a mě se krátí dech
   když před kaplí tu zpívám na schodech
   svou píseň o nebi, kde bude blaze mi
   už víc nechci mít domov svůj na zemi! **

po jmění netoužím, jsme tu jen na krátko
i sláva je jak dým, jak prázdný pozlátko
já koukám do voblak, až anděl kejvne mi
už víc nechci mít domov svůj na zemi! \s

teď říkám svý good bye světskýmu veselí
těm co si užívaj´ nechci lízt do zelí
sem hříšná nádoba, však spása kyne mi
už víc nechci mít domov svůj na zemi! \s

v určenej čas kytara dohraje
zmlkne můj hlas na cestě do ráje
o tomhle špacíru noc co noc zdá se mi
už víc nechci mít domov svůj na zemi! \s



\song{filimi}{}

/e čert aby vzal už tuhle trať
kdo /G nemáš práci tak se ztrať
/e že nemáš prachy?! --- no tak ať!
jó, tak se /h na to /e dívám! \S

\R /e filimi jori jůri ej, /G filimi jori jůri ej
   /e filimi jori jůri ej, vo tom /h si teď /e zpívám **

jen pražec chop a kolej suň
chyť lano, táhni jako kůň
po tíhou jako medvěd fuň
jó, tak se na to dívám! \s

z kůže se loupeš jako had
je vedro, že by jeden pad´
na vodu smíš jen vzpomínat
jó, tak se na to dívám! \s

když konečně máš vody dost
určitě přes ni stavíš most
kláda ti ráda zlomí kost
jó, tak se na to dívám \s

na rukách už jsem potěžkal
většinu těch okolních skal
ještě to cejtí každej sval
jó, tak se na to dívám! \s



\song{stará archa}{}

\R \[/(G) já m/(D) ám /G kocábku náram, /D7 náram, /G náram
   kocábku náram, /D7 náram/G nou. \] \krat4 **

pršelo a blejskalo se sedm neděl, kocábku náram náramnou
noe nebyl překvapenej, on to věděl, kocábku náram náramnou \s

\r já mám...

\R /G archa má cíl, archa má směr
   plaví se k araratu na /D se/G ver **

\r  já mám...

šém, chám a jáfet byli bratři rodní, kocábku...
noe je svolal ještě před povodní, kocábku...
kázal jim naložiti ptáky, savce, kocábku...
ryby nechte zachrání se samy hladce, kocábku... \s

\r  já mám...
\r  archa má...
\r  já mám...

přišla bouře zlámala jim pádla, vesla, kocábku...
tu přilétla holubice snítku nesla, kocábku...
na zemi pak vyložili náklad celý, kocábku...
ještě, že tu starou dobrou archu měli, kocábku... \s

\r  já mám...



\song{autobusy přijíždějí}{}

/C autobusy přijíždějí, autobusy přijíždě/G jí
/C možná, že už zítra, /e možná, že už /a zítra
/F pojedeme /C za na/G děj/C í \S

\[ známe zemi vyvolenou \]
\[ do vysněné brány \]
nevejdeme po kolenou \s

\[ pojedeme staří, mladí, \]
\[ že tu zůstat nechcem \]
našim pánům nejvíc vadí. \s

\[ policajti s tváří bledou \]
\[ marně do nás buší \]
stejně s námi nepojedou. \s

\[ pod jednou či podobojí \]
\[ spojme svoje srdce, \]
svoboda, ta za to stojí. \s

\[ povedeme život nový \]
\[ nepobrali všechny \]
snad se vozy vrátí, kdo ví \s



\Song{život je jen náhoda}{}

/G život je jen /C7 náhoda, /G jednou si dole jednou /G7 nahoře
/C život plyne /c jako voda a /G smrt je /D jako mo/G ře \S

každý k moři dopluje, někdo dříve a někdo později
kdo v životě miluje, ať neztrácí naději \s

/C až uvidí v životě /G zázraky, /C které jenom láska u/G mí
/A7 zlaté ryby vyletí nad mraky, /D pak po/{D\dim D7} rozumí \S

že je život jak voda, kterou láska ve víno promění
láska že je náhoda a bez ní štěstí není



\song{červená řeka}{}

jsem /C potulnej /C7 kovboj, já se /F potlou/C kám
a od ranče k /e ranči se /d najímat /G7 dám
a v /C těch mlžnejch /C7 horách na /F konci štre/C ky
potkal jsem /e holku od čer/d ven/G7 ý ře/C ky \S

pak začlo mi trápení a spousta běd
když táta se bál, abych mu ji nesved
a v těch mlžnejch horách na konci štreky
dal hlídat dceru od červený řeky \s

já schůzku jsem si s ní dal uprostřed skal
abych se s ní konečně pomiloval
a v těch mlžnejch horách na konci štreky
já líbal holku od červený řeky \s

sotva mi však řekla: ``miláčku můj,''
ze skal se ozvalo: ``bídáku, stůj!!!''
a v těch mlžnejch horách na konci štreky
stál její táta od červený řeky \s

tam pušky se ježily, moc jich bylo
mé štěstí se najednou vytratilo
a v těch mlžnejch horách na konci štreky
já obklíčenej byl u červený řeky \s

pak /G ke slovu přišla má /F pistol a /C pěst
já /G poslal je na jednu z /C nejdelších /a cest
a z těch mlžnejch hor tam na konci štreky
ved jsem si holku od červený řeky



\song{dívka s vlasem medovým}{}

chci /C vyprávět vám příběh o jedné dívce s vlasem medo/G vým
ale /G jak to bylo tenkrát v noci, to se přesně nikdy nedo/C vím
já vím jen, že jsem spěchal a řek´ si tenkrát tak si cestu /F zkrať
/G spodem kolem jezera, kde cestu kříží železniční /C trať \S

tam  v záři bílých světel vidím dívku u přejezdu stát
asi něco se jí přihodilo je to v jejím obličeji znát
už dlouho tady čekám a ještě nikdo mi nezastavil
odvezte mě prosím domů je to odtud necelých pět mil \s

tvář má bílou jako sníh a paže jako mramor šedavý
proč stála tam u kolejí tak sama v černé noci kdopak ví
tam před tím domem zastavte já za tátou teď musím domů jít
tak zmáčknu klakson, zastavím a čekám až jí přijdou otevřít \s

z chodby v nočním šeru vrhá lámpa na silnici zář
pane, já vám vezu dceru proč máte tedy zamračenou tvář?
po dívce pohled stáčím a marně kolem rozhlížím se tmou
vedle mne je místo prázdný stejně jako cesta přede mnou \s

ten muž se nejdřív diví zřejmě tomu, co se dovídá
pak zblízka na mě civí, když tichý hlasem ke mně povídá
já nevím milý pane, zda je to mýlka nebo krutý žert
tak nastartujte ten svůj bourák zmizte odsud ať vás vezme čert \s

předevčírem moje dcera zabila se tak se radši ztrať
pět mil odtud u jezera, kde cestu kříží železniční trať
já dodnes tady jezdím a hledám dívku s vlasem medovým
ale jak to bylo tenkrát v noci, to se snad už nikdy nedovím
ale jak to bylo oné noci, to se snad už nikdy nedovím



\song{chajda kůrová}{}

je bob mý jméno, říkám vám, jak před váma tu sem
já zažil sem už všelicos, když křižoval jsem zem
já kdysi byl i bohatej, teď trop sem, pámbuví
živí mě teď moje obec v chajdě kůrový \s

jen jeden škopek dostal sem, prej nemůže víc bejt
ten slouží na čaj, na maso, či když chci nohy mejt
dál kotlík mám žejdlík s šálkem, všechno fórový
každej ale cení kuchyň v chajdě kůrový \s

a z nábytku v ní nejni nic nez bedna vod džinu
tu používám k sezení i jako spižírnu
a kdo by její víko zved, má kouzlo hotový
mouchy budou ho hnát kolem chajdy kůrový \s

když víko necháš sundaný a mouchám šanci dáš
tak prostě s masem, cos tam dal, se víckrát neshledáš
já nenadávám, udělám si břicho hadrový
co je z vočí, sejde z mysli v chajdě kůrový \s

jak něžnej vítr provívá tu každou skulinou
když prohání se kolem chajdy letní krajinou
a dveře mají pokroucený trámy futrový -
zadušením neumřu tu v chajdě kůrový \s

to v zimě je zas jiná rozkoš v mojí chajdě žít
když přijde slóta, vítr hučí nebo začne lít
i saze vyrvou z komína ty deště surový
černej pepř na maso dají v chajdě kůrový \s

když přide večer, ustelu si na zem u krbu
jak voko zavřu, blecha hlodne, snad se udrbu
pak do půlnoci kleju na tý rány morový
jinak žiju si jak císař v chajdě kůrový \s

je macatejma blechama ta chajda prolezlá
když po ní plácneš, prskne na tě, jako kočka zlá
a včera hrály karty vo to, blechy čertový
která po mě drápem sekne v chajdě kůrový



\song{když náš táta hrál}{}

když /C jsem byl chlapec malej tak /C7 metr nad zemí
/F scházeli se farmáři tam /C u nás v přízemí
/ mezi nima můj táta u piva sedával
/G a tu svoji nejmilejší /C hrál \S

teď už jsem chlap jak hora, šest stop a palců pět
už jsem prošel celý státy a teď táhnu zpět
kdybych si ale v světě moh ještě něco přát
tak slyšet zase svýho tátu hrát \s

ta písnička mě vedla mým celým životem
když jsem se toulal po kolejích, žebral za plotem
a když mi bylo nejhůř, tak přece jsem se smál
když jsem si vzpomněl, jak náš táta hrál \s

to už je všechno dávno, táta je pod zemí
když je ale noc a měsíc, potom zdá se mi
jako bych od hřbitova, kam tátu dali spát
zase jeho píseň slyšel hrát \s



\Song{blues folsomské věznice}{}

můj /C děda bejval blázen texaskej ahasvér
a na půdě nám po něm zůstal /C7 vošoupanej kvér
ten /F kvér obdivovali všichni kámoši z oko/C lí
a /G máma mi řikala nehrej si s tou pisto/C lí \S

jenže i já byl blázen tak zralej pro malér
a ze zdi jsem sundával tenhleten dědečkův kvér
pak s kapsou vyboulenou chtěl jsem bejt chlap all right
a s holkou vykutálenou hrál jsem si na bonnie and clyde \s

ale udělat banku, to není žádnej žert
sotva jsem do ní vlítnul hned zas vylít jsem jak čert
místo jako kočka já utíkám jak slon
takže za chvíli mě veze policejní anton \songgg

teď vokno mřížovaný mně říká, že je šlus
proto tu ve věznici zpívám tohle Folsom blues
pravdu měla máma, radila: ``nechoď s tou holkou!''
a taky mně říkala ``nehrej si s tou pistolkou''



\Song{šlapej dál}{}

/F hej nandej na sebe modrý /C7 džíny
vlak houká, v kopci je poma/F lej
tak vstávej z tý udusaný /C7 hlíny
hej, tuláku už je ráno a tak oči otví/F rej \S

\R šlapej dál, táhni ke všem /C7 čertům
   tohleto je město prokla/F tý
   šerif náš ten nerozumí /C7 žertům
   a tak tohle ráno mohlo by bejt pro tebe dost /{F Fis G} zlý **

/G hej plandej už dál jak vede /D7 cesta
a koukej si chytit ňákej /G vlak
šerif náš je z pepřenýho /D7 těsta
a jak zmerčí tuláka tak začne řádit jako /G drak \S

\r šlapej dál...

/A hej nandej na sebe modrý /E7 džíny
vlak houká v kopci je poma/A lej
jeď dál, kam povedou tě /E7 šíny
a k našemu městu už se nikdy víckrát nevra/A cej \S

\r šlapej dál...



\song{pískající cikán}{}

/G dívka /a loudá se /G vini/D cí
/G tam, kde z/a ídka je /h níz/D7 ká
/G tam, kde str/a áň končí /h voní/C cí
si /G písni/C čku někdo /{G C D} píská \S

ohlédne se a ``pro pána!''
v stínu, kde stojí líska
švarného vidí cikána
jak leží, písničku píská \s

chvíli tam stojí potichu
písnička si ji získá
domů jdou spolu do sklípku
je slyšet cikán jak píská \s

jenže tatík jak vidí cikána
pěstí do stolu tříská
ať táhne pryč vesta odraná
groš nemá, něco ti zpíská \s

teď smutnou dceru má u vrátek
jen bůh ví jak se jí stýská
``kéž vrátí se mi zas nazpátek
ten, který v dálce si píská!''\s

a šídla honí se na louce
v trávě rosa se blýská
cikán v rozmarným klobouce
jde dál a písničku píská \s

na závěr zbývá už jenom říct
v čem je ten kousek štístka
peníze často nejsou nic
má víc, kdo po svém si píská



\song{valčíček}{}

\R tuhle /C písničku chtěl bych ti /G lásko dát
   ať ti každej den připomí/C ná
   \[ toho, kdo je tvůj, /C7 čí ty jsi /F a kdo má /C rád
   ať ti každej den /G připomí/C ná \]**

kluka jako ty hledám už spoustu let
takový trochu trhlý mý já
\[ dej mi ruku pojď půjdem šlapat náš svět
i když obrovskou práci to dá. \]\s

fakt mi nevadí, že nos jak bambulku máš
ani já nejsem žádný ideál
\[ hlavně co uvnitř nosíš a co ukrýváš
to je pouto co vede nás dál. \]



\Song{divoké včely}{}

/a nějak umírá nám /{a\heart} láska
my jako /d hejna divých /E včel jdeme /a dál
každý vztah je vlastně /{a\heart} sázka
každý /d ráno může /E zmizet, my jdeme /a dál \S

\R /a řekněte, kdopak za to /G může
   kdo z nás má /G7 právo něco /E brát?
   /a kdo učil lidi zlobu /G dýchat
   kdo na vo/G7 jáky chce si /E hrát? **

už zase bohatých je spousta
a čím víc peněz lásky míň, jdeme dál
a tak nám zbývá jenom doufat
že už zítra, zítra snad budeme dál \s

nějak umírá nám láska
my jako hejna divejch včel jdeme dál



\song{tulácký ráno}{}

/d posvátný je mi každý ráno
/A7 když ze sna budí šumící /d les
/d a když se zvedám s písničkou známou
/A7 a přesky chřestí o skalnatou /d mez \S

\R /d tulácký ráno na kemp se snáší
   /B za chvíli půjdem /C toulat se d/F ál
   /d a vodou z říčky oheň se zháší
   /B tak zase půjdem /C tou/A7 lat se /d dál **

posvátný je mi každý večer
když oči k ohni vždy vrací se zpět
tam mnohý z pánů měl by se kouknout
a hned by věděl, jakej chcem svět \s

posvátný je mi každý slovo
když lesní moudrost a přírodu zná
bobříků sílu a odvahu touhy
kolik v tom pravdy, však kdo nám ji dá



\Song{jarní tání}{}

/d když jarní tání /g cesty sněhu /F zkříží
a nad le/g dem se /A voda obje/d ví
voňavá zem se /g sněhem tiše /F plíží
tak nějak /g líp si /A7 balím proč bůh /d ví \songgg

\R přišel čas /B slunce zrození a /F tratí
   na kterejch /B potkáš kluky ze všech /{F A} stran
   hubenej /d joe, čára, ušoun se ti /g vrátí
   oživne /B kemp, /C jaro vítej k /d nám **

kdo ví, jak voní země, když se budí
pocit má vždy jak zrodil by se sám
jaro je lék na řeči, co nás nudí
a lidi co chtěj zkazit život nám \s

zmrznout by měla, kéž by se tak stalo
srdce těm pánům, co je jim vše fuk
měl bych pak naději, že i příští jaro
bude má země zdravá jako buk



\Song{na kameni kámen}{}

jako /C suchej, starej strom
jako /Cmaj7 všeničící hrom, jak v poli /C tráva
připa/C dá mi ten náš svět
plnej /Cmaj7 řečí --- a čím /a víc tím /G líp se /C mám \S

\R budem /F o něco se rvát
   až tu /D nezůstane st/G át na kameni /{C Cmaj7 C} kámen
   a jestli /C není žádnej bůh
   tak nás /Cmaj7 vezme země, /a vzduch /G no, a potom /C ámen **

a to všechno proto jen
že pár pánů chce mít den bohatší králů
přes všechna slova co z nich jdou
hrabou pro kuličku svou, jen pro tu svou \s

možná jen se mi to zdá
a po těžký noci přijde, přijde hezký ráno
jaký bude nevím sám
taky jsem si zvyk na všechno kolem nás



\song{severní vítr}{}

jdu s /D bolavou patou, mám /h horečku zlatou
jsem /G chudý, jsem sláb, nemo/D cen
a /D hlava mě pálí a v /h modravé dáli
se /G leskne a /A třpytí můj /d sen \S

kraj pod sněhem mlčí, tam stopy jsou vlčí
tam zbytečně budeš mi psát
sám v dřevěné boudě sen o zlaté hroudě
já nechám si tisíckrát zdát \s

\R /D severní vítr je /G krutý
   /D počítej lásko má s /A tím
   k /D nohám ti dám zlaté /G pruty
   nebo se /D vůbec /A nevrá/D tím **

tak zarůstám vousem a vlci už jdou sem
už slyším je výt blíž a blíž
už mají mou stopu, už větří, že kopu
svůj hrob a že stloukám si kříž \s

zde leží ten blázen, chtěl dům a chtěl bazén
a opustil tvou krásnou tvář
měl plechovej hrnek, v něm pár zlatejch zrnek
a nad hrobem polární zář



\song{hrášek}{}

jak v /D mořích slunce s tmou se ztrácí
a pokřik racků šel už /D7 spát
tak /G utekla nám láska, jó, to se /D stává
a /E najednou tě holka nemám /A rád
mám /D dost už těch tvejch věčnejch nářků
a prožárlenejch houpejch /D7 scén
tak /G někdy zase brnkni, jó, to se /D stává
prostě /A táhne mě to z kruhu /D ven \S

\R já /D půjdu až tam, na místa, kde už to znám
   pár stromů /G ohniště a tunel a /D trať
   a kašlu /D na celej svět v něm v tobě zas jsem se zplet
   a po stý /G řeknu sobě hloupneš, tak pl/D ať **

jak naboptnalý klíčky hrášků
byly cíle tvý a já chtěl víc
já předělat chtěl svět, jó, to se stává
kolik bylo nás a stále nic
tak pozdravuj ty svoje šmínky lásko
a do zrcadla zkoušej řvát
hold utekla nám láska, jó, to se stává
a najednou tě holka nemám rád \s

\r já půjdu...



\song{zpívám a meč svůj v ruce mám}{}

\R \[ /C zpívám /G a meč /F svůj v /G ruce /{C (a)} mám. \](4x) **

mou /C píseň se uč a /a nesmíš se bát
/F já víc takových /C znám
ten, kdo s druhým zpívá, /E není /a sám
já meč v ruce má/F m \S

bezmocné, bídné, ponížené
zástupy z nemanic
svůj hněv křičte s námi z plných plic
a bude nás víc \s

stará jak lidstvo je naše víra
den účtování vin
kdy náš věčný hněv se změní v čin
můj meč vrhne stín \s



\Song{půlnoční vlak}{}

zas /C půlno/G ční vlak /C houká /C7 tmou
tu /F chvíli /G dávno /C znám
tak /F dávno /G znám jak /C lásku /a svou
co /F zbloudila /G cestou k /C nám \S

\R tmou dál jiskry /G syčí, když /C hlásí svůj /F pád
   už nechci být /C ničí, už /G nechci se /C vdát **

měl žár jak hříbě půlroční
jak dítě rád se smál
ten proklatej vlak půlnoční
to hříbě sladký mi vzal \songgg

svý štěstí v mracích zářivých
čtem v sladký závrati
pak náhlý mráz je změní v sníh
a  nikdy je nevrátí \s

když zahouká vlak půlnoční
ja marný zkoušet spát
měl žár jak hříbě půlroční
ten vlak mi ho neměl brát \s



\Song{stánky}{}

/D u stánků /G na levnou krásu
/D postávaj a /g smějou se času
/D s cigaretou a s /A holkou, co nemá kam /D jít \S

skleniček pár a pár tahů z trávy
uteče den jak večerní zprávy
neuměj žít a bouřej se a neposlouchaj \s

\R jen /G zahlídli svět, maj /A na duši vrásky
   tak /D málo je, /g málo je lásky
   /D ztracená víra /A hrozny z vinic neposbí/D rá **

u stánků na levnou  krásu
postávaj a ze slov a hlasů
poznávám, jak málo jsme jim stačili dát \s

\r jen...



\song{igelit}{}

u/G krytý v stínu lesa, /D igelit
to kdyby přišel k /C ránu /G déšť
pod /e hlavou boty, nůž, tá/D tovu bundu šitou z maská/G čů
k rá/G nu se mlhy zvednou
a /D ptáci volaj, hele, lidi /C sví/G tá
pak /e větvičky si nalámou na /D oheň aby, uvařili /G čaj \S

\R a všichni se /e znaj, /D7 znaj, /G znaj
   a blázněj a /C zpí/G vaj
   a po cestách /e dál, /D7 dál, /G dál
   hledaj /a normální /G svět **

ukrytý v stínu lesa k večeru
znavený nohy skládaj
kytara zpívá o tom, jak dřív bylo líp
ten, kdo neví, nepochopí
nepromíjí čas nic, všechno vrátí
ta chvilka, co máš na život ti plyne jak od ohýnku dým \s

\r a všichni...

ukrytý v stínu lesa, igelit
to kdyby přišel k ránu déšť



\song{napsal jsem jméno svý }{}

/C napsal jsem /G jméno svý /C na zd/C7 i
/F na všechny zdi, který /C znám
/G napsal jsem /F jméno svý /C na zdi
/F ráno než /C otevřu /G krám
/C před léty /G ve stejnou /C chvíl/C7 i
/F v pravici držel jsem /C nůž
a /G všichni ti, /F co po něm /C zbyli
s nadšením /G hráli mi /C tuš \S

v okolí týhletý války
tenkrát už stávala zeď
i popsaný popravčí špalky
stály tam, co stojí teď
slyším zas slavičí hlasy
a na tvářích rozlitej hněv
jak barvičky tý jejich krásy
která nám tak pije krev \s

/F přiklušou zdáli tím /C večerní /G tichem
/F k zemi se svalí a /C dusí se /G smíchem
/F líce si hladí, než /C spálí je /G plamen
a /F setřesou mládí jak /C zlou tíhu s /G ramen \S

z půlnoci rána jak kyjem
a najednou kdo by to řek
jsme chytří a trochu moc pijem
a bijem svý děti ten vztek
napsal jsem jméno svý na zdi
a čekám až setře je čas
napsal jsem jméno svý na zdi
stejně jak každýho z vás, stejně jak každýho z vás



\song{víla}{}

jde /C po skosený /F trá/C vě a jde tak, jak se to /F pí/C še
s větrem v /d zádech s kytkou v /g hla/{d A d} vě, důstojně a /g ti/d še
a z polí /C slítají se /F ptá/C ci a /G každej z nich jí /C vě/G ří
/F ona se pak /B vra/F cí a má v /C rukou jejich /F pe/C ří \S

je to můj člověk s hlavou v kříži umírající v horku
můj plakát na refýži i má socha u new yorku
je to můj voják beze jména i ten parník v ústí něvy
je to jediná má žena jenže ta, která to neví \s

místo k /a  ní však mířím /d ně/a kam
kde je ten, co se tak /d vzte/a ká
že ta /B hrůza kterou /{Es B} čekám není /F ta, která mě /B če/F ká \S

ten můj pocit nemá hranic ta má víra bez rezervy
že už nezmůžu se na nic ani na pevný nervy
natož na nějakou vílu a že už nevím o co  kráčí
třeba nemám sílu a třeba mi to stačí



\Song{podobenství o náramcích}{}

/G jestlipak /D zpívali /G zpěváci /D smělí
/G arabský /D píšťaly /G jestlipak /D zněly
/C zdali si všimli tam /F ve víry /C doby
co /G matkám /D milenkám /G zápěstí /D zdobí \S

\R /G švartnový /D náramky /G jak smůla /D černý **

před časem na stráni seděli tepci
na srdcích orvaní ve tvářích hebcí
vsadili kabáty i nový džíny
kdo za zlaté prodá ty náramky z hlíny \songgg

\r švartnový...

matky a milenky s důvěrou vstaly
synové zemřeli milenci spali
a ženy když vstoupily do prázdných loží
zlatníkům skoupili veškerý zboží \s

\r švartnový...



\Song{hvězdy}{}

/F schovej mi /C kousek /B od ve/C če/F ře
a počkej co /{B F} poletí /C hvězd
po /F urážkách /C zpívej /B o dů/C vě/F ře
po /{B F} ranách /{B F} pohlaď mi /C pěst \S

\R /C nikdy bych nebyl tak hroší a psí
   /d kdyby mi /{g d} nebylo /C zle
   jak /C pěšímu pluku když za vesnicí
   /F počítá /B do /{F B C} neděle **

skoro se zdá že proměníš svět
a to snad má lásko i teď
kdy s holými zády už tisíce let
čekáme na odpověď \s

\r nikdy bych nebyl...

/F skoro se zd/C á, že /B promě/C níš /F svět



\song{zdálo se mně, zdálo}{}

/e zdálo se mně, zdál/C o
/e ej na nedělu rán/a o
/e že byla muzik/C a
/e a na ní /a chlapců má/e lo \S

na ní chlapců málo
a dívek na tisíce
třema řade stále
samé klebetnice \s

třema řade stále
a klebete vázale
na jedno panenko
všecke jich skládale



\Song{pojďme se napít}{}

/D pěkně tě vítám /{G A} lásko /D má
tak trochu /fis zbitá a /G víc /D soukrom/{A G fis} á
pě/e kně tě /D vítám a /G čemu vděčím /fis za tak vzácnou /h chvíli
/{G fis e } jednou /D za sto l/{A G fis} et \S

\R \[ pojď/e me se /D napít, pojďme se /{G fis e} napít
   ať ná/D m maj/e í z čeho /G slzy /D týct \]**

v nohách ti dřepí bílej pták
už nejsme slepí a zlí --- naopak!
už nejsme slepí a rozdejchaní jako když se spěchá
bránou vítězství \s

\r pojďme...

v očích ti svítí a sládne dech
a něco k pití tu ční na stolech
a něco k pití a nepospíchej stoletá má lásko
když už nemáš kam \s

\r pojďme...



\song{tři kříže}{}

/d dávám sbohem /C břehům prokla/a tejm
který v /d drápech má /C ďábel /d sám
/d bílou přídí /C šalupa ``my /a grave''
míří k /d útesům, /C který /d znám \S

\R jen tři /F kříže z bí/C lýho kame/a ní
   někdo /d do písku /C posklá/d dal
   slzy v /F očích měl a v /C ruce znave/a ný
   lodní /d deník, co /C sám do něj /d psal **

první kříž má pod sebou jen hřích
samý pití a rvačka jen
chřestot nožů, při kterým přejde smích
srdce kámen a jméno ``sten'' \s

já, bob green, mám tváře zjizvený
štěkot psa zněl, když jsem se smál
druhej kříž mám a spím pod zemí
že jsem falešný karty hrál \s

třetí kříž snad vyvolá jen vztek
katty rodgers těm dvoum život vzal
svědomí měl, vedle nich si klek...
{\i recitativ:
  vím, trestat je lidský, ale odpouštět
     božský, ať mi tedy bůh odpustí.}

\R jen tři kříže z bílýho kamení
   jsem jim do písku poskládal
   slzy v očích měl a v ruce znavený
   lodní deník a v něm, co jsem psal **



\song{frankie dlouhán}{}

kolik je /C smutného, když /F mraky černé /C jdou
lidem nad hla/G vou, /F smutnou dála/C vou
já slyšel příběh, který /F velkou pravdu /C měl
za čas odle/G těl, /F každý zapo/C mněl \S

\R měl kapsu /G prázdnou frankie dlouhán
   po státech /F toulal se jen /C sám
   a že byl /F veselej, tak /C každej měl ho /G rád
   tam ruce k /F dílu mlčky přiloží a /C zase jede d/a ál
   a /F každej, kdo s ním /G chvilku byl
   tak /F dlouho /G se pak sm/C ál **

tam, kde byl pláč, tam frankie hezkou píseň měl
slzy neměl rád, chtěl se jenom smát
a když pak večer mraky tiše usínaj
frankův zpěv jde dál, nocí s písní dál \s

tak frankieho vám jednou našli, přestal žít
jeho srdce spí, tiše smutně spí
bůh ví jak za co tenhle smíšek konec měl
farář píseň pěl, umíráček zněl



\song{snad jsem to zavinil já}{}

/a zas jsi tak /C smutná, kdo se /D má na to /d7 kou/E kat?
nic /a jíst ti ne/C chutná, v hlavě /D máš asi /d brouka
tak /C nezou/H fej, /B nic to /a není
/C za chví/H li /B se to /a změní
/d7 snad jsem /G to zavinil /{C E} já \S

zkus zapomenout na všechno, co je pouhou
tmou obarvenou načerno smutnou touhou
tak nezoufej, nic to není
za chvíli se to změní
snad jsem to zavinil já \s

/a já, /C já, /D7 já, /d7 já
/a já, /C já, /D7 já, /d7 já
/{a C D d} já, \quad\qquad /{a C D d a} já



\Song{zuzana}{}

přichá/D zím až z alabamy se svým banjem sám a /A sám
a teď /D jdu do louisiany navštívit svou /A milou /D tam \S

\R /G ó, zuzano, zu/D zano mám tě /A rád
   přichá/D zím z tý velký dálky k vám
   na banjo /A budu /D hrát **

slunce praží s nebe hromy hřmí, můj cíl je dalekej
jenže láska má mě provází, zuzano neplakej \s

včera v noci měl jsem krásnej sen a teď ho povím vám
zpíval jsem zuzaně pod oknem: zuzano mám tě rád \s

zuzana se z okna nakloní a šátkem zamává
vzdychne slzu z oka uroní. ach neplač lásko má



\song{valčík na rozloučenou}{}

kon/C čí ten /a čas, jenž /d byl pl/G7 ný
nejhez/C čích /Cisdim nocí a /{d G} dní
po/C slouchej /a jak ti/F chý val/G7 čík
na roz/C louče/G7 nou nám /C zní \S

již musíme se rozloučit, končí čas krásných dní
teskně a tiše valčík ten na rozloučenou nám zní \s

\R ach, /C zas valčík /e tančíme /C spol/G7 u
   však /C divně nám /A7 zní tento/{d G7} krát
   a /d přec stopy našeho /d7 bolu
   ty /C nesmí na /D7 nás býti /{d7 G7} znát **

na shledanou si řekneme, když k loučení je čas
na shledanou, mí přátelé, vždyť sejdeme se zas \s

proč s tím se máme rozloučit, co každý z nás měl rád?
proč s kým se máme rozejít bez víry na návrat \s

\R má všechno to, pro co jsme žili
   jen tak zapomenuto být?
   to říkám si v poslední chvíli
   když musíme se rozejít **

hm... \s

to není žádné loučení, byť chvěl se nám i hlas
jsme pevně v kruhu spojeni a sejdeme se zas!



\song{slavíci z madridu}{}

/a lalalalalalala, /e lalalalalalala, /H7 lalalalalala/e la
/a lalalalalalala, /e lalalalalalala, /H7 lalalalalala/e la \s

/e nebe je modrý a /H7 zlatý, bílá je sluneční /e záře
horko a sváteční /H7 šaty, vřava a zpocený /e tváře
vím, co se bude /H7 dít, býk už se v ohradě /e vzpíná
kdo chce, ten může /H7 jít, já si dám sklenici /{e E7} vína \S

\R /a žízeň je veliká, /e život mi utíká
   /H7 nechte mě příjemně /e snít
   /a ve stínu pod fíky /e poslouchat slavíky
   /H7 zpívat si s nima a /e pít **

ženy jsou krásný a cudný, mnohá se ve mně zhlídla
oči jako dvě studny, vlasy jak havraní křídla
dobře vím, co znamená pád, do nástrah dívčího klína
někdo má pletky rád, já radši sklenici vína \s

nebe je modrý a zlatý, ženy krásný a cudný
mantily, sváteční šaty, oči jako dvě studny
zmoudřel jsem stranou od lidí, jsem jako zahrada stinná
kdo chce, ať mi závidí, já si dám sklenici vína



\song{co jsem měl dnes k obědu}{}

/E představte si, /A7 představte si
/E co jsem měl dnes k /A7 obědu
/E představte si, /A7 představte si
/D7 co jsem měl dnes k obědu:
/G knedlíky /A7 se zelím, /D7 se zelím /G kyselým
/H7 to koukáte, /E to koukáte, /A7 co jsem měl dnes k /H7 obě/E du \S


pak jsem jed u stolu kdovíco v rosolu \s

kapustu vařenou jedli jsme ji s mařenou \s

sám jsem si za pecí zadělal telecí \s

škubánky maštěný baštil jsem jak praštěný \s

uzený na hráchu, střílel jsem ho na bráchu
jeseter na kmíně, koupil jsem ho v londýně
buchtičky se šodó, zapíjel jsem je vodó
borůvky na sádle, našel jsem je ve prádle
pirožky, blinčiky, portofěly, svinčiky
co je moc, to je moc syrečky pro olomóc
na chlebě romadur, od madam de pompadur
špenátovou konzervu, teďka všechno rozervu
vařené bravčové, sežrali ho bačové
žinčica z levoče, zná ji každý jihočech
koprovou vode dna dala tetka hodná
kůň i ocet na špeku, přinesli mě od fleků
kuřata smažený, ale už bez mařeny



\song{láska je tu s nami}{}

\R \[ /G láska /A je tu s nami, /D buďme s /h ňou \] \hfil (3x)
   /G chce sa nás dotk/A núť **

/D skákal pes /h přes oves, /G přes zelenou /A louku
šel za ním myslivec, péro na klobouku
pejsku náš, co děláš, žes tak vesel stále
řek bych vám, nevím sám, hop a skákal dále \s

\r láska...

en ten tyky dva špalíky, čert vyletěl z eletriky...
ovčáci čtveráci...
beskyde, beskyde...
polámal se mraveneček...
pec nám spadla...



\Song{skoč pod to auto!}{}

\R /a skoč pod to auto, chci vidět /E krev!
   skoč pod to auto, chci slyšet /a řev!
   nejsi má /A láska a ty to dobře /d víš
   tak skoč pod /E auto, je s tebou /a kříž **

když jsem tě viděl poprvé
ústa tvá rudá jako od krve
hned jsem si říkal, ty by se škrtilo
to by se hérdlo tvé drtilo \s

když jsem tě viděl, jak ses věšela
hned bych ti vrazil kudlu do těla
jak jsi tak něžně lapala po vzduchu
bylo to krásné, ejchuchu \s

kde jsou ty chvíle, když jsem tě měl rád
proč bych tě mlátil, když nechceš řvát
kde jsou ty chvíle, kdy jsem tě měl tolik rád
proč chceš mi, moje mládí, sbohem dát


\song{marion lee}{}

/D když hodiny v nálevně odbily šest a /A půl
/D vjela četnická patrola na hospodský /h dvůr
/D zaprášení, znavení řeč vedli při ča/A ji
/D že po stopách psance v šeru k horám spěcha/h jí
/G a když potom /D rum jim jazyk /G více rozvá/D zal
/G a měsíc jim /D cestu k lesům /h bíle ukázal
/G netušili, /D že o cíli /G jejich cesty /D ví
/G i hostinské /D dcera --- ta ma/h lá marion lee \nic /A \S

ví moc dobře, kde se skrývá ten, co ji má rád
tam se s koněm vytratí a chce ho varovat
oči zrudlé horečkou, v ústech jak na poušti
a jeho prst unaveně leží na spoušti
dvě jeskyně nad údolím čení černý chrup
tam se občas zatoulá jen šakal nebo sup
tam se skrývá a jeho zrak slídí v okolí
a jeho rty šeptají: ``tos ty, marion lee?'' \s

když k půlnoci měsíc stříbrem lesy posypal
zas patrola svoje koně popohnala v cval
se zlověstnou tichostí se začínal ten hon
pak ukázal velitel: ``hej, támhleto je on.''
měsíc mu stín prodlužoval na vrcholku skal
jeho širák zahlédli a plášť, co každý znal
i psancova koně všichni dobře poznali
netušili, že tím jezdcem je marion lee \s

honička se prodlužuje míli za mílí
jezdci jedou mlčenlivě, nikdo nestřílí
noc je tichá, pochmurná a vlhká jako hrob
za živého zajatce se platí dvojnásob
náskok se však nezkracuje, pořád cvalem dál
vpředu černý širák a pod ním černý plášť vlál
podél zátok s bažinou dál míli za mílí
odvádí je od úkrytu dál marion lee \songgg

když se k ránu mlha s tmou už začla vytrácet
když už každý z koní znaven k zemi klonil hřbet
ujížděla dívka dál, teď už nekryta tmou
za ní jezdci drmolící kletbu za kletbou
jeden zdvihl zbraň a střelil, zrak měl jako rys
tak se v sedle sesmekla a spadla na převis
sundali jí širák z očí když dojeli k ní
až teď všichni poznali malou marion lee \s

od těch dob má tahle pověst v kraji trvání
jak převlekem dívka život psanci zachrání
o něm nikdo z usedlíků neslyšel už víc
jediné, co lidé z hor ti k tomu chtějí říct:
že prý když je úplněk, noc bílá jako den
objeví se jezdec v černém plášti za kopcem
pravdu už se asi nikdo nikdy nedoví
možná, že to je duch statečné marion lee



\song{světla dostavníků}{}

krb /d sálá rudým teplem, spí /C muži nad jí/d dlem
knot /d lampy v stáji čadí, kůň /C cinkne udi/d dlem
tam /F venku svítí měsíc a stojí dosta/A7 vník
hlas /d pobrukuje tiše jen pruh /C světla do tmy /d vnik \S

a hovor cestujících už ruší noční klid
teď kletba z noclehárny, pak výkřik: ``nastoupit!''
pak mlasknutí a ``hyjé!'' a drž je skřípot kol
a po táborech pět set mil teď vyjel kob a spol \S\S


těch přepřahacích stanic, kde jinak není nic
pár hospod u milníku či na půl cestě víc?
pár jezdců u skalisek je z farem ztracených
a od stád ovcí u vod zní jen černých mužů smích \s

jak hlučí kempy v galgong, zpěv hřímá po barech
pak dálný zlatý západ, kde leklan zdvíhá břeh
snad dvacet tisíc hledá své štěstí, dusí bol
a pro ta srdce statečná sem přijel kob a spol \S\S


už zbledla hvězda ranní a dostudil už chlad
jak chutná doušek whisky, co podal kamarád
je skvělé horské ráno, z mlh vstává jasný den
a rozepínáš límec, jak jsi vzduchem opojen \s

ač dobrých cest je málo, jsi šťastný jako král
když od kol křemen lítá a duní koňský cval
a cval a cval a poklus jdem výmol nevýmol
kde zeleň splývá s oblohou, tam pádí kob a spol \songgg


svit luceren se dere tmou roklin bez lesů
a na pískovcích bleskne a strmém útesu
teď odraz plachty vozu a miht bílých vln
a plácek v hustém buši je prázdných lahví pln \s

jen záblesk prozrazuje, že rozvodnil se brod
kdos křičí do tmy: ``blázne, snad nechceš do těch vod!''
a jsme na druhém břehu, ta spousta kempů kol
zas nové trasy, nový směr vzít musí kob a spol \S\S


tam po svahu se šine vůz volský jako plaz
my svižně vzhůru vrchol a dolů o zlom vaz
zas podle horských bystřin a říček nahoru
kde ostrý obrys jezdce se ti jeví v obzoru \s

pak kolem ohrad z trní a lomů s trestanci
a spousty domků z kůry a krčmy ``na šanci''
my jedem polostepí i bušem s vůni smol
dnes ještě sto mil uvidí, jak svítí kob a spol



\song{píseň z bobří hráze}{}

/G holka teď už přestaneš civět netečně
protože /e za chvíli ti řeknu konečně
/a že naše láska /D nepřetrvá /G věky \s
ještě si dojez biftek s hranolky
jak nůž dovedu bejt ostrej na holky
co tahaj kluky pryč od bobří řeky \s

celý dny mi husí kůži nahání
slyšet jak lovíš bobříka kecání
takže felínu na usárnu balím
nemám už kam bych se jinam obrátil
jen jak dráb jít na půdu abych se nevrátil
takže se vůbec neptej kam se valím \s

klidně vypusť páru z těch svých zdymadel
sesuj se do svých televizních stínadel
nebo čum, kde je na něco ňáká fronta
chrasti náušnicema někde u baru
já už zandal do futrálu kytaru
s pampeliškou velikýho vonta \s

\R /G už jsme zas na tý starý /a dráze
   /D povídáme řeči z bobří /G hráze
   na zádech usárny nebo tele
   v hlavách starý mladý hlasatele
   už se do kalíšků za šera
   nalévá zelená příšera
   už jsme zase všichni naměko
   píseň zní daleko širokko **

mám chyby proti šípákům jsem zelenáč
bratrstvo kočičí pracky --- štětináč
nejsem vůbec žádnej mirek dušín
jak zachraňovat babičky na přechodu
tuláka na velikonoční pochodu
to jsou věci který jenom tuším \songgg

celý dny mi husí kůži nahání
že mě rychlonožka zase dohání
na šípáky boudu postavit musím
tenhle seriál by nebyl k ničemu
kdyby neměl pořádnýho ničemu
ještě šípy podrazit teda zkusím \s

jestlipak sbíraj míčky ve dvorcích
a vontové volí vonta na dvorcích
za šera se na druhou stranu bojím
já musím dělat celej život neplechu
a miluju svobodu a v tom se nepletu
já za svou pravdou každý den si stojím \s

\r už jsme...

konečně přešel sem poslední hranici
na půdě našel dopis pod správnou dlaždicí
už je holka divadýlko dohraný
nakládej do špajzu ty svoje flákoty
když mě z toho vobestíraj mrákoty
zbejvá mi flaštička poslední záchrany \s

nech si ty svoje videa a diska
ve sklepě někdo mučí malý psiska
a já s nima prožívám ty malý smrti
slepice hluchá z televizních morů
zdrhnu ti zkratkou průchodů a dvorů
píseň beze slov volá ze starý čtvrti \s

vod velikonoc až do dušiček
vaříš mi svůj votrávenej čajíček
jenže já ho leju přes rameno na zem
dál si vyvaluj ty svoje vnady na dece
eště dneska večír budu na štrece
eště dneska večír budu zase blázen \s

\r už jsme...



\song{hotel hillari}{}

tvař se /e trochu nostalgicky, už tě /C nikdy nepotkám
/A máš to jistý provždycky, nastav /C uši vzpomínkám
jak tě /e znám i v této chvíli, ved bys /C řeči peprný
jak /A tenkrát, když nám tvrdili, že je /C vítr stříbrný \S

\R a /e tváře měli kožený, my jim /C zdrhli z průvodu
   zaho/A dili lampióny a /C našli hospodu
   ale /e taky jacquese brela a s ním /C smutek z cizích vin
   a /A žádostivost těla a pak /C radost z volovin
   a ta nám /e zbejvá **

po večerech pro diváky, dělali jsme kašpary
pak na zemi dva spacáky --- náš hotel hillari
slavný sliby jsme už znali, i to, jak se neplní
a cenzoři nám kázali o správným umění \s

a tak válčím s nostalgií, bují ve mně jako mech
a pořád všechno slibuji, starý hesla na domech
tys už splatil všechny dluhy, i za hotel hillari
a já vyhážu ty černý stuhy, funebrákům navzdory \s

\R vždyť mají tváře kožený my jim zdrhnem z průvodu
   zahodíme lampióny najdeme hospodu
   a tam svýho jacquese brela a s ním smutek z cizích vin
   a žádostivost těla a pak radost z volovin
   a ta nám zbejvá **



\song{battledress}{}

/E zapnu vzpomínkovou frézu, /gis romantiky battledressů
/A a najednou je mi šestnáct /E let \s
když jsem byl do čundru janek, dřív než zpíval wabi daněk
když jsem našel celtovanej svět \s

strašně jsem chtěl než zestárnu battledress mít i usárnu
a pak jsem stál na nádraží i já
zabalený saky paky na sobě tu bundu khaki
s domovenkou czechoslovakia \s

\R silver /E a, silver /gis b, pak poslední pozdrav z palu/A by
   šlapou /H bačkůrky /E strachu /H i holínky /{C H E} záhuby
   právě slét bílý květ, noc je protržená obálka
   už se k lesům snáší valčík, švarc a opálka **

horečně se potí záda a ruka už stendan skládá
mercedes se blíží od břežan
pak ožehne výbuch zdola pyšný profil protektora
slepou gabčík odhazuje zbraň \s

\R silver a, silver b, véčkem zdraví ruka z paluby
   šlapou bačkůrky strachu i holínky záhuby
   silver a, silver b, bubny letí černá kutálka
   vzpomínám na ta jména gabčík, kubiš, opálka **

půl čtvrté a potom celá, motory zní u kostela
v kterém sedm stínů tají dech
už se vojsko staví do řad, jidášové žijí pořád
a už salvy duní na schodech \s

\R silver a, silver b, holky krásný válka neválka
   vzpomínám na ta jména: bublík, hrubý, opálka
   silver a, silver b, smrťák roztočil už osudí
   it´s a long way to tipperary, kov na spánku zastudí **

\R \[ silver a, silver b, ozvěny stenganů dováží
   reslovkou kráčí dolů, ke smíchovskýmu nádraží \] **



\song{batalion}{}

\R  /e víno /G máš a /D marky/e tánku
    /G dlouhá noc /D se /e pro/h hý/e ří
   /e víno /G máš a /D chvilku /e spánku
    /G díky, dí/D ky /e ver/h bí/e ři **

/e dříve než se rozední, kapitán k /G osedlání /D rozkaz /e dá/h vá
/e ostruhami do slabin /{G D} koně /e po/h há/e ní
/e tam na straně polední čekají /G ženy zlaťá/D ky a /e slá/h va
/e do výstřelů karabin /G zvon /D už /e vy/h zvá/e ní \S

\R víno na kuráž a pomilovat markytánku
   zítra do burgund batalion zamíří
   víno na kuráž a k ránu dvě hodiny spánku
   díky, díky vám královští verbíři **

rozprášen je batalion, poslední vojáci se k zemi hroutí
na polštáři z kopretin budou věčně spát
neplač sladká marion, verbíři nové chlapce přivedou ti
za královský hermelín padne každý rád \s

\r víno na kuráž...

\r víno máš...



\song{citrónky}{}

/C kdysi v /a dálném /F oceá/G nu
/C klesl /a malý /F parník ke /G dnu
/C jen tři /a malé /F citrón/G ky
zůstaly na hladi/C ně \S

\R \[ a duva duva duva duva do vody ťap ťap \]
   jen tři malé...  (resp. jin.) **

jeden vám praví přátelé
netvařte se tak kysele
vždyť je to přece veselé
patří nám moře celé \s

a tak pluli dál a dál
jeden jim přitom na kytaru hrál
až připluli --- kdo to ví
až na ostrov korálový \s

když připluli k ostrovu
vyšplhali se nahoru
přišla stará babajzna
a ta ty citrónky zblajzla \s

byla to stará babajzna
byla to stará kravajzna
ta je slupla i s kůrou
a tak zakončila baladu mou



\song{v hlubokém údolí}{}

v /d hlubokém údolí /g stojí tábor /d náš
když /g oheň jasně /d zahoří
tu /A listí stromů /B hovoří
že /A7 je to přítel /d náš \S

zář jeho červená nad les se vznáší
vše kolem sebe ozáří
a listí stromů proráží
na zem se v stín snáší \s

vysoko nechť vzplanou plameny tvoje
ať každý s námi velebí
to kouzlo celé přírody
jež nás obklopuje \s

oheň dohasíná, noc již nastává
i příroda se ke spánku
z večerních to radovánků
v šeď mlhy ukládá



\Song{lucka}{}

/G žala lucka trávu, žala u je/D tele
/ zakous se jí cvrček rovnou do pr...
/G dudlaj dudlaj dudlaj, dudlaj dudlaj /D daj
dudlaj dudlaj dudlaj, dudlaj dudlaj /G daj \S

běží lucka domů, hele mami, hele
zakous se mi cvrček rovnou do pr... \s

zavolali kněze, kněze gabriele
ať zažene lucce cvrčka od pr... \s

kněz gabriel káže, jenom hubou mele
nezažene lucce cvrčka od pr... \s

šel okolo voják na zádech měl tele
ukous lucce cvrčka rovnou u pr... \s

a z huby mu smrdělo ještě tři neděle
jak tam ukous lucce cvrčka u pr... \s



\song{až uslyším hvízdání}{}

/a pověz mi můj příteli co /G uděláš když rozdělí
tě s /F někým jeho /G smích jak /d žhavý /E klín
a /a on si myslí jak se zdá, že /G postačí když zahvízdá
hned /F něhou změkneš /G jako /d para/E7 fín \s

\R /C až uslyším /F hví/C zdá/G ní
   /F až uslyším /B hví/F zdá/C ní
   /C až uslyším /F hví/C zdá/G ní
   /d ukryji své zklamání za /C nekonečnou hrou a mlhou /G ranní
   /d ukryji své zklamání za /C rozzářenou tvář a pou/G smání
   /d ukryji své zklamání a /C vezmu jeho hlavu do svých /a dlaní **

a pověz mi jak lámeš mříž co uděláš když nevěříš
že kroužkovaní ptáci zpívají
oni hvízdají a pyšní jsou že létat mohou nad tebou
a na zemi ti z ruky zobají \s

\R \[ až uslyším hvízdání \] (3x)
   ukryji své zklamání za nekonečnou hrou a mlhou ranní
   ukryji své zklamání za rozzářenou tvář a pousmání
   ukryji své zklamání a narovnám jim křídla když se zraní **

poslyš když mi nevěříš ty také chceš jít pořád výš
a slunce je tak zlaté až se vlní
pak se můžeš když chceš blíž spálit víc než pochopíš
a budeš rád když nenajdeš jen trní \s

\R \[ až uslyším hvízdání \] (3x)
   ukryji své zklamání za nekonečnou hrou a mlhou ranní
   až uslyším hvízdání
   ukryji své zklamání a vezmu jeho hlavu do svých dlaní **



\song{don't krak aneb alkoholická balada}{}

a roste mi brada a křivnou mi záda
můj hlas už je na hmoždince
a rostou mi zuby až lezou mi z huby
až spočinou na holínce \s

\R sedím na rozjetým stolku
   v ruce pár mizernejch koltů
   chci se sprovodit ze světa
   bohužel nesplní se mi ta věta **

a rostou mi uši, ó jak mi to sluší
vidím se v zrcadýlku
a padaj mi vlasy, skinhead budu asi
s oprátkou na zátylku \s

\R sedím na rozjetým stolku
   okenou leštím svou lebku
   chci se sprovodit ze světa
   bohužel nesplní se mi ta  věta **

a šilhaj mi voči a moje vobočí
ztrácí svůj původní tvar
a tělo se klátí jako mojí máti
když můj otec navštíví bar \s

\R sedím na rozjetým stolku
   u sebe mám jednu holku
   chci ji sprovodit ze světa
   bohužel nesplní se mi ta věta **

a ruce se třesou vodku neunesou
a pro co budu teď žít
obrostlý plísní teď zahynu žízní
vodu přec nebudu pít \s

\R sedím na rozjetým stolku
   před sebou gorbačov vodku
   chcou ji sprovodit ze světa
   bohudík nesplní se jim ta věta **

a tvrdnou mi vousy a já si je brousím
každý den na pilníku
a nos mám jak skobu, chystám se do hrobu
však rakev je v nedohlednu \s

\R sedím na rozjetým stolku
   v hlavě už mám jednu kulku
   sprovodil jsem se ze světa
   bohudík splnila se mi ta věta **



\song{hruška maslova}{dobeš}

/C za našu stodolu hruška maslova
/G a za ňu stojí dřevjany /C plot
/ za plotem meška rodina bušova
/G povim vam o nich takovy /C hlod \S

raz idu ze šichty a vidim bušku
cholera jasna něvidi plot
drape se s kyblem na našu hrušku
takovy je to zgyzděny rod \s

\R /F pomahej pan buh, hned jak ju zjavim
   /C červeny zlosťu u lisně pravim
   / hledatě bryle --- tuž to se bavim
   /G tuž to sem přišel akorat vhod, a/{C --- D} ú **

/{G...} napoly mrtva pod ovocnanem
ležela buška, moja oběť
všeci mysleli, že je s ňu amen
dokud nězačla na hubě mleť \s

\R vyškantovala mi esi sem chory
   když cely svět řeši uzemni spory
   mirovu cestu, no braly mě mory
   a ozval se mi v žaludku vřed, aú   **

povidam, kdyby šlo enem o politiku
to bysme spolem popili
našli bysme si drahu putyku
a tyděň bysme řešili \s

\R vy stě si ale popletla pojem
   tu idě o hrušky a mam taky dojem
   že to se řeši jedině bojem
   za použití násilí, aú **

za našu stodolu hruška maslova
a za ňu stoji železny plot
ostnate draty, brana dubova
a čtyry dogy hlidaju vchod




\song{husličky}{}

\[ /D čiže jste husličky, /{G D} čije
/e kdo vás tu /D zane/A chal \]
/e na trávě /A pová/D lané, /e na trávě /A pová/{D fis} lané
/e u paty /D oře/A ch/{e D A} a \S

\[ kdože tu trávu tak zválal
aj modré fialy \]
\[ že ste husličky samé \]
na trávě ostaly \s

kery tu muzikant usnul
co sa mu přišlo zdát
kery tu mládenec usnul
co sa mu přišlo zdát
co sa mu enem zdálo
bože co sa mu v noci zdálo
že už dál nechtěl hrát \s

\[ zahrajte husličky samy
zahrajte zvesela \]
až sa tá bude trápit
bože až sa tá bude trápit
\[ kerá ho nechtěla \]



\Song{a te réhradice}{}

/d a te réhradice /C na pěkný ro/F vině
/G teče tam vo/a děnka /d dolů po /G dě/C dině
/g je pě/a kná, je /d čistá \S

a po tej voděnce drobný rebe skáčó
pověz mně, má milá, proč tvý oči pláčó
tak smutně, žalostně \s

pláčó, one pláčó, šuhajo vo tebe
že zme sa dostali daleko vod sebe
daleko vod sebe \s

proč by neplakaly, dyž hlavěnka bolí
musijó zaplakat šohajovi kvůli
šohajovi kvůli



\song{sbohem galánečko}{}

/D sbohem galá/h nečko, /e já už musím /{A D} jíti
/A sbohem galá/fis nečko, /h já už musím /{E A} jíti
\[ /e kyselé ví/A nečko, /D kyselé /G ví/D neč/A ko
/D podala´s /G mě k /{D A D} pití \]\S

\[ ač bylo kyselé, preca som sa opil \]
\[ eště včíl sa stydím \]
co jsem všecko tropil \s

\[ sbohem galánečko, rozlúčme sa v pánu \]
\[ kyselé vínečko \]
podala´s mě v džbánu \s

\[ ale sa něhněvám, že´s mňa ošidila \]
\[ to ta moja žízeň \]
ta to zavinila



\Song{vracaja sa domů}{}

/G vracaja sa /D domů /a od betléma /D do vsetí/G na
/a nesl sem sa /D jak ta laňka
/G nesl sem sa /G7 jak ta laňka --- /C potre/{D G} fená
/C a že bylo /D po betlémě /G živo veli/a ce
/G voněl jsem jak /a mariánek /G od sli/D vovi/C ce \S

vracaja sa domů znavený a šťastný velmi
až sem sebú od radosti
z tej velkej božskej milosti
praštil k zemi \s

ležím přemítám, najednú je u mňa žena
\[ a nevěří, že sa vracám \] od betléma
a že bylo po betlémě živo velice
a že nás tam josef nutil do slivovice \s

aj ženy, ženy --- nic na betlém nevěříjú
\[ a když přijdú na pútníčka \] hned ho zbijú
tož byl u nás na štědrý den betlém hotový
ještě dneska nedoslýchám na obě nohy



\song{třešničky}{}

když /G josef veselku /a chystal, byl už /G starý, skoro /a kmet
a /G dívka maria /a čistá /G gali/e lejs/a ký /D květ \s

jednou se šli spolu projít, josef nechtěl, ale šel
vždyť byli jen krátce svoji a vzduch plný včel \S

jednou se šli spolu projít, byl červen třešní čas
a jak tam pod stromem stojí zní mariin hlas:
moh bys jozífku prosím pár třešní natrhat
já dítě pod srdcem nosím a co když má hlad \s

tu josef si začal rvát vlasy, ó, já jsem naletěl
kdopak moh setřít asi tvojí nevinnosti pel
víš, já mám trpělivost svatou, ale to ti povídám
ať ten, kdo je dítěti tátou mu podá třešně sám \s

a zatímco josef tu běsní, náhle dětský hlásek slyš:
sehni se milá třešni k mojí mamince blíž
a hle, strom se k zemi sklání, maria trhá s úsměvem
a josef nemá ani zdání, jak z té šlamastyky ven \s

můj bože, já se tolik stydím, ale řekni, pane náš
kdy chceš přijít mezi lidi, kdypak narodit se máš
já myslím tak šestého ledna v době vánoc v noční čas
až se zjeví hvězda neposedná, tehdy přijdu mezi vás \s

roky skáčou jako hříbě, ale když se snese sníh
lidé zpívají si příběh o těch třech a o třešních



\song{z betléma se ozývá}{}

/G jaké divné světlo /C7 shůry na zem /G slétlo
nám narodil se /D pán
/G v chlévě panna /G7 čistá /C7 porodila /G krista
/G nám naro/e dil /D se /G pán \S

kopyta /G koní písek /G7 víří, /C7 jak se zpráva /G šíří
/G nám narodil se /D pán
/G od mrtvého /G7 moře zní k /C7 olivecké ho/A7 ře
/G nám /e naro/a dil /D se /G pán \S

\R \[ z /G betléma se ozý/D vá /C halelu, /D halelu
   z /G betléma se ozý/D vá /C nám naro/D dil se /G pán \] **

pastýři v olivovém háji /G7 na šalmaje hrají
/C nám narodil se /G pán
zatímco v /e nazaretu městě /H7 vyhrávají /e žestě
/C nám narodil se, nám všem naro/D dil se \S

\r z betléma...

z východu přebohatí páni dítěti se klaní
nám narodil se pán
poutníci  ??????  po svých k jeslím dary nosí
nám narodil se, nám všem narodil se \s

\r z betléma...



\song{johanka}{}

/a s hlavou /e skloně/d nou lidí zástup se tu /a dívá
nebe /e nad hla/d vou, slyšíš /G dětskej /a pláč
jenom s /e vírou /d svou stojí dívka plavá, /a bílá
oheň /e nad se/d bou, jenom s /G pravdou d/a ál \S

\R /E hej, muži, přidej /a oheň spí, vždyť /E páni se nudí /a jen
   ať /E plameny nesou /F zprávu zlou, jak /C skončil
   soudnej /G den
   s johan/a kou **

dík tvůj dal ti král, celá francie si zpívá
to se osud smál, smutek utíká
s naší johankou ke štěstí se země dívá
vítr zprávu vál, že se dýchat dá \s

\r hej, muži...

s hlavou skloněnou lidí zástup se tu dívá
nebe nad hlavou slyšíš dětskej pláč
popel s vánkem ví, co se v dívčím srdci skrývá
hra se zastaví, jiná začíná \s

\r hej, muži...



\song{jdem zpátky do lesů}{žalman}

/a7 sedím na kolejích, /D7 které nikam neve/{G C G} dou
/a7 koukám na kopretinu, jak /D7 miluje se s lebe/G dou
/a7 mraky vzaly slunce /D7 zase pod svou ochran/G u, /e ó
/a7 jen ty nejdeš, holka zlatá, /D7 kdypak já tě dosta/{G D7} nu \S

\R /G z ráje, my vyhnaní /e z ráje
   kde není už m/a7 ísta, prej něco se chy/G stá, /D7 ó
   /G z ráje nablýskaných /e plesů
   jdem zpátky do /a7 lesů /C za nějaký ča/G s **

vlak nám včera ujel ze stanice do nebe
málo jsi se snažil, málo šel jsi do sebe
šel jsi vlastní cestou, a to se dneska nenosí
i pes, kterej chce přízeň, napřed svýho pána poprosí \s

\r z ráje...

už tě vidím zdálky, jak máváš na mne korunou
jestli nám to bude stačit, zatleskáme na druhou
zabalíme všecky, co si dávaj rande za branou
v ráji není místa, možná v pekle se nás zastanou \s

\r z ráje...



\song{elektrický valčík}{}

/c jednoho letního večera na návsi pod starou /G7 lípou
hostinský antonín kučera vyvalil soudeček s /c pípou
neby/As lo to posvícení, neby/c la to neděle
v naší /As obci mezi kopci plni/G ly se /G7 korbe/{G6 G7} le. \S

\R /C byl to ten slavný den, kdy k nám byl zaveden
   elek/G7 trický /{Gdim G} proud
   /G7 byl to ten slavný den, kdy k nám byl zaveden
   elek/C trický /{Cdim C} proud
   /C7 střída/F vý,  /G střída/e vý
   /a silný /d7 elek/G7 trický /C proud
   /C7 střída/F vý,  /G střída/e vý
   /a zkrátka /d elek/G7 trický /{c G7 c G7} proud **

{\i recitativ:} \s
kdo tu všechno byl:
okresní a krajský inspektor
hasičský a recitační sbor
poblíže obecní váhy
tříčlenná delegace z prahy
zástupce nedaleké posádky
pod vedením poručíka vosátky
početná skupina montérů
(jeden z nich pomýšlel
na dceru sedláka krušiny)
dále: krojované družiny
alegorické vozy
italský zmrzlinář antonio cosi
na motocyklu indián
a svatý jan z kamene vytesán \s

\r byl to ten...


\songgg{}

na stránkách obecní kroniky
ozdobných písmem je psáno:
``tento den pro zdejší rolníky
znamená po noci ráno.''
budeme žít jako v praze
všude samé vedení
jedna fáze, druhá fáze
třetí pěkně vedle ní \s

\r byl to ten...

{\i recitativ 2:} \s
z projevu inženýra maliny
z elektrických podniků:
vážení občané, vzácní hosté!
s elektřinou je to prosté
od pantáty vedou dráty
do žárovky nade vraty
odtud proud se přelévá
do stodoly, do chléva
při krátkém spojení dvou drátů
dochází k takzvanému zkratu
kdo má pojistky námi předepsané
tomu se při zkratu nic nestane
kdo si tam nastrká hřebíky
vyhoří --- a začne od píky
{\b do každé rodiny
elektrické hodiny!} \s

\r byl to ten...



\song{telegrafní cesta}{poutníci}

(to) /G tenkrát dávno šel /e pustinou muž
a /D na pravým místě tam /C vykácel buš
z klád /C postavil d/D ům že měl /G sílu jak b/D ýk
a /C rozoral zem jako /a válečník \S

jenže za ním jdou další a ti umí víc
vázaný krovy a zdi z vepřovic
a do zlatejch polí a bučení krav
po ňáký době zní telegraf \s

a už je tu kostel a konečně most
a železná ruda a zločinnost
a okresní město má okresní soud
a ta stará trať jméno telegraph r/e oad \S

těžký jdou časy teď znova a znova
skončila válka a chystá se nová
rozmoklou stezkou, co prošel ten skaut
v deseti proudech jdou provazy aut
jak /(C?) dravá ř/G eka \s

a /a rádio hlásí, že v noci byl mráz
/e lidi jdou z práce a nemají čas
jenže /D vlak domů má /h velký zpoždě/e ní \S

už nemůžu dělat, co kde bych kdy chtěl
třeba v pralesích kácel --- to bohužel
můžu jen sklízet, co zasil jsem sám
a zaplatit všechno, co komu kde mám \s

ti šedaví ptáci na drátech z mědi
vo tomhletom kódu už ledacos vědí
ti můžou letět a zapomenout
na celej řád týhle telegraph road \songgg

víš, že bývaly časy, kdy bylo to zlý
kdy spali jsme v dešti a promrzlí
teď když mi říkáš --- tak vem si co chceš
moc dobře cítím, že už je to lež \s

ale důvěřuj ve mě, dej kočímu bič
já vezmu tě s sebou a odvedu pryč
ode všech temnot a vysokejch zdí
od všeho strachu, co v ulicích spí
já prošel jsem pamětí kdekterej kout
a viděl jen smutek jak pýcha se dmout
a chtěl bych zapomenout
na všechny ty zákazy vjezdu, který jsou rozsetý
po celý telegraph road



\song{blažek}{dobeš}

/C měl sem rad blažka jak /F svojiho /C synka
/F z jedneho /C hrnka sme kafe pi/G li
/F vymetli jsme /C kina, /F takovi jsme /C byli
a žadnu srandu sme /G nězka/C zili \S

žizeň sme hasili v knajpě na nadraži
a jak byl v raži, jak už ho měl
do jedne tlapy chytnul dva chlapy
aji do lochu bych misto ňho šel \s

loni se u nas povyšovalo
na vyšši mista a za vyšši plat
to se robi dycky automaticky
jak kerysi chachar zdrhně na zapad \s

za ten děn pluvnul by sem si do ksichtu
bo sem se zdržel u jedne ženy
a jak sem po absenci přišel na šichtu
tak už byl blažek povyšeny \s

tym padem blažek mi včil robi šefa
tuž to je trefa, tuž to je gol
tuž to je drama, takovy fama
ulice mama, otec alkohol \s

dejte se všeci přejeť rychlikem
kozla zahradnikem stě zrobili
a pytlaka hajnym a zloděja tajnym
to stě se pěkně vybarvili \s

dyť je to gauner a sviňa sprosta
duševně silně zakřiknuty
na prstach s bidu počita do sta
a vyšmatlane nosi buty \s

od jiste doby němam rad blažka
k svatku mu ruku něpodam
a jak na mopedu přes kaluž pojedu
schvalně ho dycky ochlustam \s



\song{hrobař}{}

/G v mládí jsem se učil hrobařem
/e jezdit s hlínou, jezdit s trakařem
/C kopat hroby byl můj ideál \S

jezdit s hlínou, jezdit s vozíkem
s černou rakví s bílým pomníkem
toho bych se nikdy nenadál \s

že do módy přijde kremace
černý hrobař bude bez práce
toho bych se nikdy nenadál \s

kolem projel vůz milionáře
záblesk světel pad mi do tváře
marně skřípou kola brzdící \s

stoupám vzhůru, stoupám ke hvězdám
tam se s černou rakví neshledám
sbohem bílé město zářící \s

sbohem moje město
vzpomínat budu přesto
jak jsem poznal tvůj smích
i tvůj pláč \s

na na nanana na...



\song{zapomenuty trumf}{dobeš}

/C z uhla prach a kašel a /F kdybys farať /C kaj šel
tož /F dycky /G trpi /C pajšel --- to /G každy haviř /{C G} vi
tož /C vylezli sme z ďury a /F shodili mun/C dury
a v /F ramci /G fizkul/C tury sme /G turu podni/C kli
/F hrnul sem se přikladem /C bo sem nětušil
/F že se budě losovať /G aby přehled byl
kdo s /C kym zrobi dvojicu aby v /F červencovym /C hicu
z vitko/F vic na /G ostra/C vicu do /G cila dora/C zil \S

štajger vypsal ceny --- dvě kože z hyeny
aby moh byt odměněny ten nejzdatnějši tym
něvim jak to přijdě štěsti stalo kajsi indě
něchalo mě po krk v bryndě esi uhodnětě s kym
bořivoja lupeňa mladeho kluka
vytahnula moja ruka přimo z klobuka
pozdravil tu mě matě měl tenisky od batě
od teplakuv měl gatě latu na latě \S

po startu hned jak v lize dostal se do krize
pry ponožka ho hryze a z hladu je mu mdlo
nic něbylo z te tury bo nas předjižďaly fury
šlapaly po nim kury no hrozne divadlo
jela kolem sanitka a už se hrnul k ni
esli pomoc potřebuješ enem si řekni
nabral sem vzduch do ňader a kopnul ho do hader
řeci zaber gyzde zaber, bo sme posledni \s

brzo rezignoval, už sem ho něprovokoval
psychicky něštenkroval k rychlosti větši
stejně jak noha mině nohu a tak se modlim k bohu
to jako bezpartijni mohu, ale nic to něřeši
posledni eso sem ale v kapsi měl
sam sem o nim do včilejška ani něvěděl
byl sem s tym tak smiřeny, že pozbuděmy ceny
ty kožuchy z hyeny, že sem si něvzpomněl \songgg

dal sem mu chleba a kus špeku a ze sameho vzteku
sem žuchlal enem veku --- no co sem měl robiť
taky peň se těžko hleda, co idě rano bez oběda
ani plan se podvesť něda, musi se překročiť
na šachtě su take hromske zakony
že jak něni špeka, tož něsu vykony
aj kdyby sem byl dajny, hněď jak sme vyšli z lajny
moh sem ten cizokrajny kožuch něska ponosiť



\Song{čert ví, kdy kotvy zvednem}{}

/d náš lodivod už je zas jak kára
/C náš lodivod už je zas jak kára
/d náš lodivod už je zas jak kára
/F koukej, /A7 jak se /d klátí

\R čert ví, kdy kotvy zvednem (3x)
   když to s ním tak mlátí **

hej, rychle všechen rum přes palubu (3x)
ať nepije dále

\R čert ví, kdy kotvy zvednem, (3x)
   nejdřív na tři krále **

náš lodivod už není jak kára (2x)
náš lodivod je fit a kárá
námořníky hloupý

\R čert ví, kdy kotvy zvednem, (3x)
   až se rum zas koupí **



\song{david a goliáš}{J+V+W}

/D lidi na li/h di jsou jako /A saně
/D člověk na člo/h věka jako /A kat
/D podívejte /H se na ně, /e musíte /A7 naří/{D Cis7} kat
/fis obr do pidimužíka /Cis7 mydlí
/fis domnívaje se, že vyhra/Cis7 je
/A klidně /Fis7 seďme /h na ži/E7 dli
/A čtěme bibli, tam to všechno je: \S

/D samuelova kniha nám /e poví/A7 dá
/D jak na žida přišla veli/e ká bí/A7 da
/D jak ti /D7 bídní /h filiš/D7 tíni %
/G válku vést ne/g byli líní
/D až potkali /B7 davi/A7 da \S

david šel do války volky nevolky
z velké dálky nesl bratrům homolky
v pochodu se cvičil v hodu %
dal si pro strýčka příhodu
tři šutry do /{B7 A7 D} tobolky. /\ D7 hej \S

/B hej --- kam se /D valej vždyť jsou /E7 malej!
takhle /A7 goliáš ho /Fis7 provokuje
/e7 david slušně /A7 salutuje
/D když mu ale obr plivnul /e do o/A7 čí
/D david se otočí, prakem /e zato/A7 čí
/D když za/D7 čínáš, /h no tak /D7 tu máš
/G byl jsi velkej, /g já měl kuráž %
/D a jakej byl /{E7 A7 D A7 D} goliáš



\song{bedna od whisky}{}

/a dneska už mě /C fóry ňák /a nejdou přes py/E sky
/a stojím s dlouhou /C kravatou na /a bedně /E od whi/a sky
/a stojím s dlouhým /C obojkem /a jak stájovej /E pinč
/a tu kravatu, co /C nosím, mi /a navlík /E soudce /{a A} lynč

\R tak /A kopni do tý /D bedny, ať /E panstvo neče/A ká
   jsou /A dlouhý schody /D do nebe a /E štreka dale/A ká
   do /A nebeskýho /D baru, já /E sucho v krku /A mám
   tak /A kopni do tý /D bedny, ať /E na cestu se /{A a} dám **

mít tak všechny bedny od whisky vypitý
postavil bych malej dům na louce ukrytý
postavil bych malej dům a z vokna koukal ven
a chlastal bych tam s billem a chlastal by tam ben \s

kdyby si se hochu, jen pořád nechtěl smát
nemusel jsi dneska na týhle bedně stát
moh si někde v suchu tu svoji whisky pít
nemusel jsi dneska na krku laso mít \s

až kopneš do tý bedny, jak se to dělává
do krku mi zůstane jen dírka mrňavá
jenom dírka mrňavá a k smrti jenom krok
mám to smutnej konec a whisky ani lok

\R tak kopni do tý bedny...
   \ \vdots
   tak kopni do tý bedny{\b . \dag} **



\song{proklatej vůz}{}

\R /a čtyři /E bytelný /a kola /F má náš proklatej /G vůz
   tak /C ještě p/F ár /C dlouhejch /G mil /F zbejvá /G dál
   /F tam je c/G íl
   a tak /C zpívej ó /F san/G ta /{C C7} cruz
   /F polykej whisky a /C zvířenej prach
   /G nesmí nás /C porazit stra/C7 ch
   až /F přejedem támhleten /C pískovej plác
   pak /G nemusíš se /G7 už rudochů /C bát **

george, už jsem celá roztřesená, zastav
zalez zpátky do vozu, ženo!

\r jen tři bytelný kola...

tatínku, tatínku, už mám plnej nočníček
synku, to není nočníček, to je soudek s prachem!!!

\r už jen dvě bytelný kola...

seno, seno, wiki, jedeme jak s hnojem
jó, koho jsem si naložil, toho vezu

\r už jen jediný kolo...

george, když já se tak strašně bojím indiánů!
zatáhni za sebou plachtu, ženo, a mlč!!!!!!!!

\r už ani jediný kolo nemá náš bytelnej vůz...



\song{bim bam}{}

/D hvězda zářila a /A vzduch se chvěl
hvězda zářila a /D vzduch se chvěl
/D7 hvězda zářila a /G vzduch se chvěl
jasná /D noc nad be/A7 tlémem /D stála

\R /D tichounce tam /{G A} hráli
  /D na zvonky si /G zvonili /A7 bim /D bam
   přišli lidé z /{G A} dáli, /D povídali a /G chodili /A7 sem /D tam
   /G ó, /D jak se máte, /G ó, /D proč se ptáte
   tichounce tam /{G A} hráli, /D na zvonky si /G zvonili /A7 bim /D bam **

\[ v chlévě dítě jako růže květ \]
králové vzdávají mu díky \s

\[ látky, šperky radost pohledět \]
k tomu datle a čerstvé fíky

\r tichounce...

\[ k ránu všichni tiše usnuli \]
k jejich štěstí či k jejich smůle \s

\[ hvězda zářila a vzduch se chvěl \]
pokoj všem lidem dobré vůle

\r tichounce...



\song{kytka}{nedvědi}

otvírám /G lásku na stránce /D rád
přišel jsem /A milá má něco ti /D dát
zeptat se, co /G děláš a jakej byl /D den
pohladit /A tvář, tu kytku si /D vem

\R ty jsi tak jiná, tak jiná, kdo ví
   jestli má touha tě neporaní
   ty jsi tak jiná, pojď ruku mi dej
   s tebou je celej svět jak vyměněnej **

sedíme tu spolu a slova si jdou
propletený prsty ležej na kolenou
oči jako čert a malinkej nos
ze všech je nejhezčí, tiše už dost

\r ty jsi tak...



\Song{slunovrat}{nedvědi}

/a stíny nad obzorem pomalu jd/G ou, noc se /a snáší
paseky osvítí zář borovejch klád zapále/E nejch

\R /a sešli se zapomeno/C ut /G a zvednout hlavu to sta/d čí
   /a jen trochu popadnout de/e ch než půjdou zas d/a ál **

ohnutý lžíce a nůž, dům z igelitu, lístek zpátky
boudy na zastávkách, kde potkáváš lidi, co potkat jsi chtěl

\r sešli se...

stíny nad obzorem pomalu jdou, právě svítá
šlápoty ohniště klid, smutnej je kemp opuštěnej

\r sešli se...



\song{růže z texasu}{}

/D jedu takhle večer stezkou dát /G stádu k řece /D pít
v tom potkám holku /h hezkou, až /G jsem vám z koně /A slít
měla /D kytku žlutejch květů, snad /G růží, co já /D vím
znám plno hezkejch /h ženskejch k světu ale /e tahle /A hraje /D prim

\R kdo si /G kazí smysl pro krásu ať s /D tou a nebo s tou
   dej si říct, že kromě texasu tyhle /E růže neros/A tou
   ať máš /D kolťák nízko u pasu, ať jsi /G třeba zloděj /D stád
   tyhle žlutý /h růže z /G texasu budeš /e pořád /A mít už /D rád **

řekla, že tu žije v ranči jen sama s tátou svým
a hrozně ráda tančí, teď zrovna nemá s kým
tak jsem se jí nabíd, že půjdu s ní a rád
a že se dám i zabít, když si to bude přát \s

hned si dala se mnou rande a přišla přesně v půl
a dole teklo rio grande měsíc po něm plul
když si to tak v hlavě srovnám, co víc jsem si moh přát
ona byla krásná, štíhlá, rovná, zkrátka akorát \s



\Song{montgomery}{}

/D déšť ti holka smáčel /G vlas/e y
/A z tvých očí zbyl prázdnej /D kruh
kde je zbytek tvojí /G krás/e y
/A to ví dneska jenom /D bůh

\R /D z celý jižní eska/G dron/e y /A nezbyl ani jeden /D muž
   v montgomery bijou /G zvony/e , /A déšť ti smývá ze rtů /D růž **

na kopečku v prachu cesty leží i tvůj generál
v ruce šátek od nevěsty, ale ruka leží dál \s

tvář má zšedivělou prachem, zbylo v ní pár těžkých chvil
proužek krve stéká prachem, déšť mu slepil vlas jak jíl



\song{ruty šuty arizona texas}{}

/A šinu si to starým coloradem
v tom uslyším výkřik vzdále/E7 ný
/A jedu podle /A7 hlasu /D a k svému ú/d žasu
/A leží tam muž k /E7 zemi skole/A ný

\R /A ruty šuty arizona texas, ruty šuty arizona /E7 má
   jedu podle hlasu... **

sehnul jsem se k jeho obličeji
poznal jsem v něm svého dvojníka
\[ jeho tělo bylo probodáno noži
teplá krev z něj proudem utíká \]\s

\R ruty šuty...
   jeho tělo bylo... **

pravil ke mně hlasem  zmírajícím
příteli můj, parde jediný
\[ byla to jen léčka, já to nevěděl
jako starej vůj jsem naletěl \]\s



\Song{guadalcanal}{}

/d vlny moře bouří, větry dují, loď /g unáší v /d dál
američtí námořníci plují na /g guadalca/A7 nal
/d harmonika tesknou píseň zpívá do /B přísvitu /d hvězd
/g jak ta píseň zní, /d mnohý z nich to ví, /A7 že se možná nevrá/D tí \songgg

\R /D krásná mere/{fis F\heart } dith, /G víš, kdo šel se /{D H7} bít
   /e a plul na křiž/D níku /A7 na guadalca/D nal
   krásná mere/{fis F\heart } dith, /G až zas budeš /{D H7} snít
   /e nezapomeň /D nikdy /A7 na guadalca/D nal
   jak tam pod palmami /fis pad /D ten, co měl tě tolik /A7 rád
   /D krásná mere/{fis F\heart } dith, /G chceš se pomo/{D H7} dlit
   /e za ty, jež ne/D vrátil /A7 nám guadalca/D nal **

v širém moři na ostrově malém, o kterém ty sníš
v háji stinných mlčenlivých palem prostý jen kříž
pod ním mnohý z mladých námořníků spí věčný svůj sen
když se hvězdy skví v ticho půlnoční v palmách větví píseň zní

%\unitlength = .5mm
%\begin{picture}(30,80)
%\put(3,45){\makebox{\f Fis\heart}}
%\multiput(0,0)(4,0){6}{\line(0,1){40}}
%\put(0,40){\line(1,0){20}}
%\multiput(0,39)(0,-10){4}{\line(1,0){20}}
%\multiput(16,25)(-4,-10){3}{\circle*{3}}
%\end{picture}



\Song{niagára}{}

na břehu /E niagáry stojí /H7 tulák starý
na svou první lásku vzpomí/E ná
jak tam stáli spolu, díva/H7 li se dolů
až jim půlnoc spadla do klí/E na

\R teskně hučí nia/H7 gára, teskně hučí do no/E ci
   \[ komu vášeň v srdci /A hárá, tomu /E není /H7 pomo/E ci \]
   střemhlav do propasti padá /H7 proud
   na něm tebe vidím, děvče, /E plout
   \[ škoda, že ten přelud /A krásný nelze /{E H7 E} obejmout \]**

osud tvrdou pěstí zničil lidské štěstí
i ten nejkrásnější jara květ
i ten kvítek jara vzala niagára
nevrátí jej nikdy zpět



\song{když se načančám}{}

/C rumělku ve tvář/F ích ležím v polštář/G ích
to mě právě /C baví
nic mě netí/F ží, jen si prohlíž/G ím
všechny šperky co /C mám
tak svět mi radosti /F chystá
a v /G kramflecích jsem si /C jistá
když se načanč/F ám, když se načanč/G ám
když se načan/C čám \S

po drahých kobercích chodit ve špercích
to mi zkrátka sluší
hosty přijímám, jen se zajímám
copak ušít si dám
a svět mi rukama tleská
jsem roztomilá a hezká
když se načančám...

\R /G jen já, jen /F já, jen o mě tu /C kráčí
   /G jen já, jen /F já jsem princezna /C tvá
   ta, která udělá /F hačí a k /G vladaření jí /C stačí
   když se načan/F čá, když se načan/G čá
   když se načan/C čá **

čím je to, čím, že spím-li či bdím
tak stále jsem švarná
příčinu znej, já ze všech jsem nej
že na sebe dbám
ať projdu zámkem či smrčím
jsem dokonalá a frčím
když se načančám...



\song{noc na karlštejně}{}

/d ač mám spánek /a bezesný, mě /G včera /A sen se /d zdál
/d já ač dávno /a nejsem s ním, mě /G navští/A vil sám /d král

\R \[ ře/C kl: ``/F lásko má, já /C stůňu
   svoji /g pýchu já jen /d hrál
   kvůli /F vám se vzdávám /B trůnu
   kleno/F tů i /C kate/{F (A)} drál'' \] **

já ač mám den poklidný, dnes nevím kudy kam
trápí mě sen ošidný a trápí mě král sám

\r řekl...



\Song{já budu chodit po špičkách}{}

/D zavři oči /G a jdi spát, /e vždyť už bude /A brzy den
/D nech si o mně /G něco zdát, /e ať je krásný /A ten tvůj sen

\R /e já budu chodit /A po špičkách, /e snad tě tím nevzbu/A dím
   /e áááááž slunce /A vyjde v tmách, /G polibkem tě /A probudím **

jdi si lehnout, ať už spíš, ať z toho snu něco máš
ráno, až se probudíš, polibek mi taky dáš \s

rozhoď vlasy na polštář a sni o mně krásný sen
nebo ráno nepoznáš, že je tady nový den



\song{montana}{}

/D jděte všichni k /D6 ďasu s /G mravným živo/A7 tem
/D jednou budu /D6 stejně /G ležet za plo/A7 tem
/D stará kmotra s /H7 kosou je /e rychlejší než /Fis7 já
jsem  /G povoláním /D střelec a /A7 víc mě neče/D ká
/G měl jsem kdysi dívku do/D7 le v montaně
/G přemejšlel jsem o ní /E7 zamilova/A7 ně
/D chodili jsme /D6 spolu po/G čítat hvězdič/A7 ky
/D mně teď zbyl jen /D6 kousek /A7 smutný písnič/D ky

\R /D adios mon/D6 tano, mu/G síme /A7 jít
   ne/D ní nám /D6 přáno /G neukon/A7 čit
   co /D bylo nám /D6 dáno mě/G sícem, který /A7 plál
   /D je doko/D6 náno a /G život /A7 jde /D dál **

kdopak za to může, ptám se dokola
že pro tři liščí kůže z něj byla mrtvola
ukázal na mě prstem šedivý koroner
hochu, boj se boha, byl to tvůj revolver
od té doby se mi nerozednívá
nevoní mi tráva a slavík nezpívá
na prsou mám kámen, v očích mi sedí strach
fialové dálky mi zakryl šedý prach \s

nemám nikde stání, jak ve vichřici dým
nemám slitování s koněm uštvaným
řítím se jak jezdci z apokalypsy
a na mý stopě vyjou zatoulaní psi
jednou na vás všechny zařvu holari
za klobouk si dejte ty všivý dolary
do všech vašich sporů mi dávno houby je
vraťte mi mou lásku a rodný prérie!



\song{quién sabe}{}

/e kdo ví, proč se /h vracím z toulek /H7 sám
kdo /e ví, proč v srdci /h prázdnej pocit /Fis mám
quién /H7 sabe, kde zůstal /e dům, dřevěnej /a dům
v tom domě /Fis krb, kamennej /H7 krb
quién /e sab/{C a} e, quién sabe \S

kdo ví, kde je země, kterou znám
kdo ví, proč se vracím nepoznán
quién sabe, kde je láska, kde je dům
kde mě čeká kamarád \s

kdo ví, proč mi zůstal jenom pláč
kdo ví, kdo mi poví, co jsem zač
quién sabe, kdo je tulák, kterej má
prázdnou náruč, prázdnou tvář
quién sabe, quién sabe \s

kdo ví, kde je mekka všech mých cest
kdo ví, kdy se vrátím z cizích měst
quién sabe, kde je hrobník, kterej má
pro mě rakev cedrovou
quién sabe, quién sabe, kdo to ví...



\song{hobo}{}

/G7 já nechci bejt /C sám, když koleje /e duní
a temná /d noc /G7 do dáli ubí/C há
co z toho k sakru mám, že znám plno /e vůní
dalekejch /d cest /G pohledy v očích /C mám
tak co z toho /F mám, že v srdci mám /a touhu
jenom se /F mrknout za nejbližší /G7 strouhu
já často /C šněroval svý toulavý /e boty
musí to /d bejt, /G7 můj vlak má zele/C nou

\R /C tmou nákladní vlak /B těžce /G7 duní
   v /C mých očích jenom /B vítr /G7 šumí
   /C kraj kolem trati /F spí, /C jen mašinfíra /G7 bdí
   /C má ruku na rychlost/B ní /G7 páce
   /C má topič plný /B ruce /G7 práce
   /C já si na nápravě /F zpívám
   /C píseň, v /As které /G vlaky /C hřmí **

když vagóny hřmí a brzdař tě honí
necejtíš hlad, nohy tě nezebou
jen běžíš po střeše a chtěl bys bejt pod ní
píšťala řve a tunel před tebou
tak co z toho mám, že znám celý státy
když nesmím na oči vlastního táty
když stromy rozkvetou já stojím na trati
musí to bejt, můj vlak má zelenou \s

můj osobní vůz je vagón s vepřovým
můj rodný dům je bouda brzdaře
kde nechal jsem squaw, to už vám nepovím
snad ve frisku, tam jsem byl na jaře
tak co z toho mám, pořád se jen toulat
v špinavý ruce pár centů žmoulat
jednou se netrefím a budou mě sbírat
musí to bejt, můj vlak má zelenou \s



\song{pražce}{dobeš}

/D házím tornu na svý záda, feld flašku a /A sumky
/ navštívím dnes kamaráda z železniční prumky

\R vždyť je /D jaro, zapni si /h kšandy
   pozdravuj /A vlaštovky a muziko ty /D hraj **

vystupuji z vlaku, kterej mizí v dálce
stojim v české třebové a všude kolem pražce \s

pohostil mě slivovicí představil mě mařce
posadil mě na lavici z dubového pražce \s

provedl mě domem, nikde kousek zdiva
všude samej pražec, jen máňa byla živá \r to je to jaro...

plakáty nás informují: ``přijď pracovat k dráze''
pakliže ti vyhovují rychlost, šmír a saze \s

a jestli jsi labužník a přes kapsu se praštíš
upečeš si kávu na železničním pražci \s

a naučíš se skákat, tak jak to umí vrabec
když na nohu si pustíš železniční pražec \s

když má děvče z třebové rádo svého chlapce
posílá mu na vojnu železniční pražce \s

a když děti zlobí, tak hned je doma mazec
děda mráz jim nepřinese ani jeden pražec \s

před děvčaty z třebové chlubil jsem se silou
pozvedl jsem pražec --- načež odvezli mě s kýlou \s

pamatuji pouze ještě operační sál
pak praštili mě pražcem a já jsem tvrdě spal

\R a bylo jaro, zapni si kšandy
   lítaly vlaštovky a zelenal se háj **



\song{umbabaú}{o princezně, která ráčkovala}

/D umbabaú, umbabaumbaumba
/h umbabaú, umbabaumbaumba \s

/D někdo ráčí kráčet v brnění, má na něm rádoby /A7 er
/D když není urozen, /G  rychle je prozrazen
/D když ne/A7 říká /D krásné er
tak ten si /A7 říká /D o malér \s

moudrý král, princeznin bratranec
když vdával svých pět dcer
řek, to dál nesnesu a prchnul do lesů:
žádný ženich nemá er
kdo nemá er, ztrácí charakter

\R /G královské reggae, /D královské reggae
   /G královské reggae, /A7 královské reggae **

umbabaú... \s

ten, kdo chce mít pravou princeznu
musí být z vyšších sfér
kdo prosí o ruku, musí mít záruku
tou je jeho ryčné er
pouze ten může být premiér
a kdo nemá er, tak je amatér \s

mějte, prosím, trochu strpení
rozhodně všechno je fér
tak proč ty trampoty, příšerné drahoty
přece slyšíš to mé er
které mě pasovalo nad průměr

\r královské reggae...

umbabaú...



\song{saxana}{saxana}

\R /{A D A} saxano, v knihách vázaných v ků/E ži 
   /{A D A} zapsáno kouzel je /E víc než /A dost
   /{A D A} saxano, komu dech se z nich ú/E ží 
   /{A D A} saxano, měl by si /E říct už /A7 dost **

/A7 cizími slovy ti /C7 jedna z nich poví, že /D7 muži se loví
buď /F7 pan admi/G7 rál nebo /E7 král \S

vem oko soví, pak dvě slzy vdovy, to svař a dej psovi
co vyl a byl sám opodál \s

seď chvíli tiše a pak hledej spíše, kde ve všem se píše
že tát bude sníh, loňský sníh \s

najdeš tam psáno, jak změnit noc v ráno, jak zakrýt ne ano
a pláč v nocích zlých změnit v smích

\R saxano, v knihách vázaných v kůži
   zapsáno kouzel je na tisíc
   saxano, v jedné jediné růži
   saxano, kouzel je mnohem víc \s

   saxano, v knihách vázaných v kůži
   zapsáno kouzel je na tisíc
   saxano, v jedné jediné růži
   saxano, kouzel je mnohem víc **



\song{jedním tahem}{šíleně smutná princezna}

/e dejte mi /h malíři /C barvy všech /D druhů 
/e já budu /h do mraků /C malovat /D duhu
/e řekněte /h básníci, /C jak verš se /D skládá 
/e ona mne, /h one mne /C asi má /D ráda \S

já /C nejsem ani leonardo ani michelangelo
/D paletu mám amatérsky chudou
/a ale když mne paní múza políbila na čelo
/D rozhod jsem se skoncovati s /G nudou 

\R  /C samým /F blahem /C jedním /F tahem
    /C srdce z /D lásky /F malu/G ju
    /C proto/F že já, /C proto/F že já
    /C proto/D že já /F milu/G ju
    a /C maluju a /F miluju a /C miluju a /F maluju
    a /C miluju a /F maluju a /C miluju a /F maluju **

já neměl nikdy nadání, mě hýčkaly uměny
z kreslení jsem vždycky míval pětky
ale dneska vlastní duší kreslím fresky na stěny
fagule mi slouží místo štětky

\r  samým blahem...



\song{znám jednou starou zahradu}{šíleně smutná princezna}

znám /D jednu starou zahradu, kde /D9 hedvábná je tráva
má /D vrátka na pět západů a /D9 mně se o ní zdává
/G tam žije krásná princezna, má /C opá/F le/C nou /G pleť 
jen /e já vím /A7 jak je /C líbez/C7 ná
tak /F neblázni /{C7 G} a seď \S

/C ná ná na na na ná na na na ná na /F ná ná
/Es ná ná na na na ná na na na ná na /As ná ná
/C ná ná na na na ná a á /{F D} á \S

v té zahradě je náhodou i studna s černou mříží
a stará vrba nad vodou, co v hladině se shlíží
ten rybník s loďkou dřevěnou tu čeká na nás dva
tak pojď a hraj si s ozvěnou a zpívej jako já \s

/C ná ná na na na ná na na na ná na /F ná ná
/Es ná ná na na na ná na na na ná na /As ná ná
/C ná ná na na na ná a á /{F G7 C} á \S



\song{kdepak ty ptáčku hnízdo máš}{popelka}

/C kdepak ty ptáčku hnízdo máš, skrýš a záze/G mí? 
vždyť ještě léčky málo znáš, málo zdá se /C mi 
/ hej, břízo bílá, skloň se níž
dej ptáčku náruč svou a /F skrýš 
já ať můžu /C jít a v duši /G klid 
můžu pak /{E G7} mít \S

/C kdepak ty ptáčku hnízdo máš, kam dnes půjdeš /G spát? 
až sníh a mráz dá rukám plášť, sám se začnu /C bát 
/ hej, břízo bílá, skloň se níž
dej ptáčku náruč svou a /F skrýš 
já ať můžu /C jít a v duši /G klid 
můžu pak /{C C7} mít \S

/F já ať můžu /C jít a v duši /G klid /G7 můžu pak /C mít 
/C kdepak ty ptáčku hnízdo /G máš \qquad 3x
/C hmmmmmm/G mmm



\song{není nutno}{tři veteráni}

\R  /D není nutno, není nutno, aby bylo přímo vese/e lo
    /A7 hlavně nesmí býti smutno, natož aby se breče/{D A7} lo \s
    chceš-li trap se, že ti v kapse zlaté mince nechřestí
    nemít žádné kamarády, tomu já říkám neštěstí **

nemít /h prachy --- /D nevadí
nemít /h srdce --- /D vadí
zažít /h krachy --- /D nevadí
zažít /h nudu --- j/G ó, to /A vadí \S

\r není nutno...



\Song{afričančata}{nohavica}

/C v africe tam žijou sloni
/a a po/d dobná /G zvířa/C ta
/C mezi stromy je tam honí /a afri/d čan/{G C} čata
/F načančané /G afričanče /C jako /d uhel černé /G je
/C ráno skočí na saranče
/a a jede /d do gu/G ine/C je \s

děvčata tam honí kluky
a kluci zase děvčata
k nám to mají trochu z ruky --- afričančata
přes moře a přes potoky 2600 mil
na kole tak dva tři roky
no a pěšky ještě dýl \s

průšvih je když afričanče
černý jak mý boty
narodí se naší anče od nás z dolní lhoty
ale narodí či nenarodí --- vždyť je to vlastně putna
hlavně že nám nemarodí
a že mu u nás chutná



\song{hlídač krav}{nohavica}

/D když jsem byl malý říkali mi naši
/ dobře se uč a jez chytrou kaši
/G až jednou vyrosteš /A budeš doktorem /D práv \S

takový doktor sedí pěkně v suchu
bere velký peníze a škrábe se v uchu
já jim ale na to řek chci být hlídačem krav \s

já chci mít čapku s bambulí nahoře
jíst kaštany mýt se v lavoře
od rána po celý den
zpívat si jen
zpívat si pam pam padam pam ... \s

k vánocům mi kupovali hromady knih
co jsem ale vědět chtěl to nevyčet jsem z nich
nikde jsem se nedozvěděl jak se hlídají krávy \s

ptal jsem se starších a ptal jsem se všech
každý na mě hleděl jako na pytel blech
každý se mě opatrně tázal na moje zdraví \s

já chci mít čapku s bambulí nahoře
jíst kaštany mýt se v lavoře
od rána po celý den
zpívat si jen
zpívat si pam pam padam pam ... \s

dnes už jsem starší a vím co vím
mnohé věci nemůžu a mnohé smím
a když je mi velmi smutno lehnu do mokré trávy \s

s nohama křížem a rukama za hlavou
koukám nahoru na oblohu modravou
kde se mezi mraky honí moje strakaté krávy \s

já chci mít čapku s bambulí nahoře
jíst kaštany mýt se v lavoře
od rána po celý den
zpívat si jen
zpívat si pam pam padam pam ...



\song{když mě brali za vojáka}{nohavica}

/a když mě brali za vo/C jáka, /G stříhali mě doho/C la
/d vypadal jsem jako /a blbec
/E jak ti všichni dokola, /F la, /G la, /C la, /G la
/a jak ti všichni /E doko/a la \S

zavřeli mě do kasáren, začali mě učiti
jak mám správný voják býti
a svou zemi chrániti, ti, ti, ti
a svou zemi chrániti \s

na pokoji po večerce ke zdi jsem se přitulil
vzpomněl jsem si na svou milou
krásně jsem si zabulil, lil, lil, lil
krásně jsem si zabulil \s

když přijela po půl roce, měl jsem zrovna zápal plic
po chodbě furt někdo chodil
tak nebylo z toho nic, nic, nic, nic
tak nebylo z toho nic \s

neplačte vy oči moje, ona za to nemohla
protože mladá holka lásku potřebuje
a tak si k lásce pomohla, la, la, la
tak si k lásce pomohla \s

major nosí velkou hvězdu, před branou ho potkala
řek jí, že má zrovna volný kvartýr
tak se sbalit nechala, la, la, la
tak se sbalit nechala \s

co je komu do vojáka, když ho holka zradila
nashledanou pane fráňo šrámku
písnička už skončila, la, la, la
jakpak se vám líbila, la, la, la
nic moc extra nebyla



\song{tři čuníci}{nohavica}

/D v řadě za sebou tři čuníci jdou
ťápají si v blátě cestou nece/h stou
/G kufry nemají, cestu nezna/A jí
/e vyšli prostě do světa a /A vesele si zpívají: \S

ui ui ui ui uí... \s

levá pravá teď přední zadní už
tři čuníci jdou jdou jako jeden muž
žito chřoupají ušima bimbají
vyšli prostě do světa a vesele si zpívají... \s

auta jezdí tam náklaďáky sem
tři čuníci jdou jdou rovnou za nosem
lidé zírají důvod neznají
proč ti malí čuníci tak vesele si zpívají... \s

když se spustí déšť roztrhne se mrak
k sobě přitulí se čumák na čumák
blesky blýskají kapky pleskají
oni v dešti v nepohodě vesele si zpívají... \s

když kopýtka pálí když jim dojde dech
sednou ke studánce na vysoký břeh
do vody koukají kopýtka máchají
chvilinku si odpočinou a pak dál se vydají... \s

za tu spoustu let co je světem svět
přešli zeměkouli třikrát tam a zpět
v řadě za sebou hele támhle jdou
pojďme s nima zazpívat si jejich píseň veselou...



\song{kozel}{nohavica}

/G byl jeden pán \/ byl jeden pán
ten kozla /C měl \/ ten kozla měl
velice /D si \/ velice si
s ním rozu/G měl \/ s ním rozuměl \s

měl ho moc rád \/ měl ho moc rád
opravdu moc \/ opravdu moc
hladil mu fous \/ hladil mu fous
na dobrou noc \/ na dobrou noc \S


jednoho dne
se kozel splet
rudé tričko
pánovi sněd \s

když to pán zřel
zařval jéjé!
svázal kozla
na koleje \S


zapískal vlak
kozel se lek
to je má smrt
mečel mek mek \s

jak tak mečel
vykašlal pak
rudé tričko
čímž stopl vlak



\song{metro pro krtky}{nohavica}

/G prvá druhá /C třetí /D čtvrt/G á
/G na zahradě /C krtek /D vrt/G á
/e drápy má jak /C vývrtky, óó/G ó
/a vrtá metro /C pro krt/D ky, óó/G ó

\R rrrrrrrrrrrrrrrrrrrrrrrrrrr
   rrrrrrrrrrrrrrrrrrrrrrrrrrr
   rrrrrrrrrrrrrrrrrrrrrrrrrrr
   rrrrrrrrrrrrrrrrrrrrrrrrrrr **

každý kdo si zaplatí, óóó
smí se projet po trati, óóó
od okurek po macešky, jé
dál už musí každý pěšky, jé

\r rrr...



\Song{gaudeamus igitur}{nohavica}

/a v čele šly panny, celé /C bosé by/G ly a nesly /a moranu
za nimi chlapci, kvítka /C pod koši/G lí klobouky /a na stranu
/F ve stájích ržáli vala/C ši a /G havran krákal k ú/a svitu
/F za stodoličkou za /C naší
/a gau/G de/a am/G us /C ig/d it/G ur
gaudeamus igi/a tur \S

když vítr zadul, morana jak sosna chřestila větvemi
úplně vzadu zima neúprosná ležela při zemi
ve stájích... \s

ten průvod jara, jara ještě v plínkách, sunul se pomalu
od úst šla pára, led o ledy břinkal a bylo tří králů
ve stájích... \s

šly panny s chlapci, přímrazky je zábly a hřála naděje
dva zvony bily jeden pro konstábly, druhý pro zloděje
ve stájích...



\song{grónská písnička}{nohavica}

/G daleko /a na severu /D je grónská /G zem
žije tam /a eskymačka s /D eskymá/G kem
\[ /G my bychom /a umrzli jim /C není zi/G ma
snídají /a nanuky /D a eskym/G a \]\S

mají se bezvadně vyspí se moc
půl roku trvá tam polární noc
\[ na jaře vzbudí se a vyběhnou ven
půl roku trvá tam polární den \]\s

když sněhu napadne nad kotníky
hrávají s medvědy na četníky
\[ medvědi těžko jsou k poražení
neboť medvědy ve sněhu vidět není \]\s

pokaždé ve středu přesně ve dvě
zaklepe na íglů hlavní medvěd
\[ dobrý den mohu dál na vteřinu
já nesu vám trochu ryb na svačinu \]\s

v kotlíku bublá čaj, kamna hřejí
psi venku hlídají před zloději
\[ smíchem se otřásá celé íglů
medvěd jim předvádí spoustu fíglů \]\s

tak žijou vesele na severu
srandu si dělají z teploměrů
\[ my bychom umrzli jim není zima
neboť jsou doma a mezi svýma \]



\song{pijte vodu}{nohavica}

\R \[ /C pijte vodu, pijte pitnou vodu
   pijte vodu a n/G epijte /C rum \]**

/C jeden smutnej eisenboňák
pil na pátém nástupišti /G eierko/C ňak
/C huba se mu slepila
diesel lokomotiva ho /G zabi/C la \S

v rodině u becherů
becherovku pijou rovnou ze džberů
proto všichni becheři
mají trable s játrama a páteří \s

pil som vodku značky gorbačov
a potom povedal som všeličo i volačo
vyfásol som za to tri roky
teraz pijem už len chlorované patoky \s

jeste\' smy ch\l opci z warzsawy
je\. zdzimy poc\c angem za robot\c a\ do ostravy
cztery trzy vodki i mnó\. zstwo piv
poprostu bardzo fajny kolektiv \s

jedna paní v americe
ztrapnila se převelice
vypila tam na ex rum
a poblila jim bílý dům



\song{vlaštovka}{nohavica}

/C vlaštovko leť /a přes čínskou zeď
/F přes písek /C pouště go/G bi
/C oblétni zem /a přileť až sem
/F jen ať se /C císař zlo/G bí
/e dnes v noci zdál se mi /a sen
/F že ti zrní nasypal /G ludvig van beethoven
/C vlaštovko leť, /a nás chudé veď /{F C} {\ } \S

zeptej se ryb, kde je jim líp
zeptej se plameňáků
kdo závidí nic nevidí
z té krásy z pod oblaků
až spatříš nad sebou stín
věz, že ti mává sám pan jurin gagarin
vlaštovko leť, nás chudé veď \s

vlaštovko leť rychle a teď
nesu tři zlaté groše
první je můj, druhý je tvůj
třetí pro světlonoše
až budeš unavená
pírka ti pofouká máří magdaléna
vlaštovko leť, nás chudé veď \s



\song{dolní lhota}{nohavica}

/E rozkřiklo se dneska ráno v dolní lhotě
/ že po poli chodil divný pán
/ viděli ho malí kluci dírou v plotě
/ nad hlavou měl kruh a v ruce džbán
/A psi štěkali koně ržáli krávy se bály
/ nad obilím tetelil se vzduch
/E staré babky ve fěrtochu povídaly
/D že do dolní lhoty přišel /E bůh. \S

seběhli se všichni lidé z blízka z dáli
nad kapličkou rozklinkal se zvon
hádali se u rybníka jako malí
jestli je to nebo není on
říkali si ťululum a janku hloupý
a do toho všeho štěkal pes
ještě štěstí že pak začly padat kroupy
jinak se tam hádají i dnes \s

v poli žita ráno svítá večer se stmívá
a je jedno, kdo to vlastně byl
auto jede, řeč se vede, píseň se zpívá
a mě ještě jeden refrén zbyl
kdo máš oči ke koukání tak se dívej
kdo máš uši ke slyšení slyš
kdo mi věříš tak se ke mně přidej a zpívej
kdo nevěříš, mlč jako myš



\song{svatební}{nohavica}

/D barokní /A varhaník /G navlík si /A paruku
a pudrem přemázl tvář
magda a jan se drží za ruku
a kráčejí před oltář \S

hou hou hou zvony bijí
hou hou hou a já v sakristii
hou hou hou tajně schován
hou hou hou zamilován \S

tři krásní velbloudi --- dar krále hasana
frkají před kostelem svatého matěje
bílý je pro magdu, hnědý je pro jana
ten třetí černý vzadu pro mě je \s

hou hou hou už jsou svoji
hou hou hou a já v černém chvojí
hou hou hou tajně schován
hou hou hou zamilován \s

na staré pramici po řece moravě
připlouvá kmotr jura
fidlá na housle, klobouk má na hlavě
a všichni křičí hurá \s

hou hou hou už jdou spolem
hou hou hou a já za topolem
hou hou hou tajně schován
hou hou hou zamilován \s

ech lásko bože lásko
zanechala si ně
a to sa nedělá
srdce ti vyryju na futra předsíně
abys nezapomněla \s

hou hou hou že byl jsem tady
hou hou hou umřel hlady
hou hou hou tajně schován
hou hou hou zamilován
tvůj zamilován



\song{mikymauz}{nohavica}

/a ráno mě probouzí /G tma, sahám si na zápěstí
/F zda mi to /E ještě tluče, /F zdali mám /E ještě štěstí
/a nebo je po mně a /G já mám voskované boty
/F ráno co /E ráno stejné /F probuzení /E do nicot/a y \S

není co, není jak, není proč, není kam
není s kým, není o čem, každý je v sobě sám
vyzáblý don quijote sedlá svou rozinantu
a bůh je slepý řidič sedící u volantu

\R /E zapínám /F telefon /d záznamník /E cizích citů
   /E špatné zprávy /F chodí jako /d policie /E za úsvitu
   /d jsem napůl bdělý a /G napůl ještě v noční pauze
   /C měl bych se smát, ale /E mám úsměv mikymauze
   /a rá/G na bych zru/F ši/E l **

dobrý muž v rádiu pouští čikoreu
opravdu veselo je asi jako v mauzoleu
ve frontě na mumii mám kruhy pod očima
růžový rozbřesk fakt už mě nedojímá \s

povídáš něco o tom, co bychom dělat měli
pomalu vychládají naše důlky na posteli
všechno se halí v šeru --- čí to bylo vinou
že dřevorubec máchl mezi nás širočinou

\R postele rozdělené na dva suverénní státy
   ozdoby na tapetách jsou jak pohraniční dráty
   ve spánku nepřijde to, spánek je sladká mdloba
   že byla ve mně láska je jenom pustá zloba
   dráty bych zrušil **

prokletá hodina, ta minuta, ta krátká chvíle
kdy věci nejsou černé, ale nejsou ani bílé
kdy není tma, ale ještě ani vidno není
bdění je bolest bez slastného umrtvení \songgg

zběsile mi to tepe a tupě píchá v tříslech
usnout a nevzbudit se, nemuset na nic myslet
opřený o koleno poslouchám tvoje slzy
na život už je pozdě a na smrt ještě brzy

\R co bylo kdysi včera je jako nebylo by
   káva je vypita a není žádná do zásoby
   věci co nechceš ať se stanou ty se stejně stanou
   a chleba s máslem padá na zem vždycky blbou stranou
   máslo bych zrušil **

povídáš o naději a slova se ti pletou
jak špionážní družice letící nad planetou
svlíknout se z pyžama --- to by šlo ještě lehce
dvacet let mluvil jsem a teď už se mi mluvit nechce \s

z plakátu na záchodě prasátko vypasené
kyne mi zatímco se kolem voda dolů žene
všechno je vyřčeno a odnášeno do septiku
jenom mně tady zbývá prodýchat pár okamžiků

\R sahám si na zápěstí a venku už je zítra
   hodiny odbíjejí signály dobrého jitra
   jsem napůl bdělý a napůl ještě v noční pauze
   chtěl bych se smát ale mám úsměv mikymauze
   lásku bych zrušil **

ráno mě probouzí tma, sahám si...



\song{až to se mnu sekne}{nohavica}

/d až obuju si rano /A černe papirove /d boty
/F až i moje stara pocho/C pi, že nejdu do ro/F boty
/g až vyjde dluhy pruvod smutečnich hostu
na /d slezsku ostravu od sykorova mostu
/A až to se mnu sekne
to bude /d pěkne, /g pěkne, fajne a /d pěkne
/A až to se mnu definitivně /d sekn/{A d A} e \S

aby všeckym bylo jasne, že mě lidi měli rádi
ať je gulaš silny, baby smutne, muzika ať ladi
bo jak sem nesnašel šlendrijan ve vyrobě
nebudu ho trpět ani co sem v hrobě
to bude pěkne... \s

s někerym to seka, že až neviš, co se robi
jestli pomohla by deka nebo teplo mlade roby
kdybych si moh vybrat chtěl bych hned a honem
ať to se mnu šlahne tak jak ze starym magdonem
to bude pěkne... \s

jedine, co nevim, esi startku nebo spartu
bo bych tam nahoře v nebi nerad trhal partu
na každy pad s sebu beru bandasku s rumem
bo rum nemuže uškodit, když pije se s rozumem
to bude pěkne... \s

já vim, že bože nejsi, ale kdybys třeba byl, tak
hoď mě na cimru, kde leži stary lojza miltak
s lojzu chodili sme do orlove na zakladni školu
farali sme dolu tak už doklepem to spolu
až to se mnu sekne
pěkne, to bude pěkne
až to se mnu definitivně sekne \s

až obuju si rano černe papirove boty
až i moje stara pochopi, že nejdu do roboty
kdybych co chtěl dělal všechno malo platne
mohlo to byt horši nebylo to špatne
až to se mnu sekne
kdybych co chtěl dělal všechno malo platne
mohlo to byt horši nebylo to špatne
až to se mnu \s

na na na...



\song{margita}{nohavica}

/e točí se točí /a kolo /e dokola a až nás /h smrtka /C zavolá
tak vytá/G hneme /e rance
sbalíme svý tě/a lesný /e ostatky, srazíme /h podpa/C tky
a potom dáme /G se do /e tance
/a tančí se tradičně /e gavotta na konci /h živo/C ta
na konci toho/G hle /e plesu
promiňte, že vám /a šlapu /e na nohu, já za to /h nemo/C hu
já tady vlastně /G ani /e nejsu

\R /a a krásná margita, /e tanečnice smrti
   /a má svetr vzoru pepi/e ta a zadkem vrtí
   /a začíná maškarní /e ples, kam's to /h vlez
   kam's to /C vlez, kam's to /H7 vlez, kam's to /e vlez? **

nějakej dobrák stáhnul roletu na tomhle parketu
je vidět jenom na půl metru
tváře se ztrácejí v šerosvitu každý hledí na margitu
v pepitovým svetru
kapelník pozvednul taktovku, vytáhnul aktovku
a v ní měl nějaký noty
tváře se točí v tombole, někdo si sedí za stolem
jiný se na parketu potí \s

pokojská v šest přijde do práce provětrá matrace
pod lůžkem najde zlatou minci
děvenka dole v recepci čte si o antikoncepci
a přemýšlí o krásném princi
šatnářce v šatně zbyly dvě vesty, no to je neštěstí
a co když ráno přijde kontrola
ručičky ukazují čtyři nula nula, jedna z nich se hnula
vše se točí dokola



\song{muzeum}{nohavica}

/D ve slezském muzeu /A oddělení
/h třetihor je bílý /G krokodýl a
/D medvěd a liška a /A kamenní /D trilobiti
/D chodí se tam jen tak co /A noha nohu
/h mine abys viděl, jak ten /G život plyne
/D jaké je to všechno /A pomíjivé /D živobytí
/G pak vyjdeš do parku a /D celou noc se
/e touláš noční opavou a /C opájíš se
/G představou jaké to /D bude v /G ráji
\R /D v pět třicet pět jednou z /A pravidelných
   /h linek sedm zastávek do /G kateřinek
   /D ukončete nástup, /A dveře se /D zavírají **

budeš-li poslouchat a nebudeš-li
odmlouvat, složíš-li svoje maturity
vychováš pár dětí a vyděláš dost peněz
můžeš se za odměnu svézt na velkém
kolotoči, dostaneš krásnou knihu
s věnováním zaručeně
a ty bys chtěl plout na hřbetě krokodýla
po řece nil a volat tutanchámon
vivat vivat po egyptskému kraji

\r v pět třicet pět...

pionýrský šátek uvážeš si
kolem krku ve valtické poručíš si
čtyři deci rumu a utopence k tomu
na politém stole na ubruse
píšeš svou rýmovanou odysseu
nežli přijde někdo, abys šel už domů \songgg
ale není žádné doma jako není žádné
venku není žádné venku to jsou jenom
slova, která obrátit se dle libosti dají

\rr

možná si k tobě někdo přisedne a
možná to bude zrovna muž, který
osobně znal egypťana sinuheta
dřevěnou nohou bude vyťukávat
do podlahy rytmus metronomu
který tady klepe od počátku světa
nebyli jsme nebudem a nebyli jsme
nebudem a co bude až nebudem jen
navezená mrva v boží stáji

\rr

žena doma pláče a děti doma
pláčí, pes potřebuje venčit a
stát potřebuje daň z přidané hodnoty
a ty si koupíš krejčovský
metr a pak nůžkama
odstříháváš pondělí, úterý, středy, čtvrtky, pátky, soboty
v neděli zajdeš do slezského
muzea podívat se na vitrínu
kterou tam pro tebe už mají

\rr



\song{peklo a ráj}{nohavica}

/a nuda a šeď, mlha v /d rezavé /E rýně
/C maková /E chuť, /d chromý kůň /E kope kopyty
/a kácí se zeď a /d ze staré /E skříně
/C vytéká /E rtuť, /d můj krevní /E obraz rozlitý

\R /a peklo a ráj, /d malá kropenatá /E vrána
   /a krákorá na plotě /d vedle mých /E vrat
   /a peklo a ráj, /d zavrzala stará /E brána
   /a poprvé v životě /d budu se /E bát **

formanský vůz plný hadrů a cárů
projíždí les, vozka má brýle beze skel
nuda a hnus, psi páří se v páru
nahoře bez, obrazy lásky nehezké

\R peklo a ráj, na žebříku kvete plíseň
   kabátek z mohéru, kolkolem puch
   peklo a ráj, zezdola i shora píseň
   procesí truvérů, pálí se vzduch **

kápy káp káp, černá infúze hrůzy
studený pot, spirála marných nadějí
jsem jako krab, mám už krabatou chůzi
sedám na schod, snad mi tu ještě nalejí

\R peklo a ráj, štamprlátka pimprlátka
   jak hejna komárů sajou mi krev
   peklo a ráj, herodova neviňátka
   ničí mou kytaru, ničí můj zpěv **



\song{tři sudičky}{nohavica}

/D tři sudičky /A u kolébky /G hoj hoj stály /D mi
/D podle tvaru /A mojí lebky /G hoj hoj /A věšti/D ly
/D já jsem zrovna /A dělal kakú /G hoj hoj do ple/D nek
/D nechápal jsem /A smysl jejich /G hoj hoj /A myšle/D nek

\R /D hoj hoj hola /e hoj /G budoucnost je /D boj
   /D hoj hoj hola /e hoj a /G tak se /A chlapče /D boj **

první babka byla mírně hoj hoj obézní
snad proto mi přivěštila ruku princezny
ať s ní strávím ve svém loži hoj hoj co to dá
čili jak to zpívá samson hoj hoj pohoda

\r hoj...

druhá babka byla zloduch hoj hoj od kosti
snad proto mi přivěštila samé starosti
jenom ať se hošík trochu hoj hoj potrápí
ať je jeho princeznička hoj hoj na chlapy

\r hoj...

třetí řekla holky vy snad hoj hoj blbnete
víte jaká nemoc vládne hoj hoj ve světě
ať zavládne v jeho žití hoj hoj idyla
vzala nůž a fik --- a věštby zrušila

\r hoj...

z toho plyne poučení hoj hoj pro všecky
každý problém dá se řešit hoj hoj vědecky
aby nekradlo se uřež lidem hoj hoj ručičky
aby nezdrhali za kopečky hoj hoj nožičky

\r hoj...



\song{hrdina nebo dezertér}{nohavica}

/e nad zemí zjizvenou zákopy létají vrány
žížaly po dešti vylezly z podzemních /a děr
mužové zalehli na místo
začal se okamžik /G nejis/H7 tot
/a kdo z nás je hrdina /H7 a kdo je /e dezertér \S

poručit srdci, ať nebouchá, ať jen tak cinká
čapku si ozdobit vějířem bojarských per
hlava, ať hlava nic netuší
že rukama sahám si na duši
jestli jsem hrdina anebo dezertér \s

v batůžku na zádech nosíme fotky svých blízkých
střelka jde na jih a my musíme na sever
nad hlavou ohňostroj zuří
ne ještě nechce se umřít
ani jak hrdina ani jak dezertér \s

prokleté období lásek a nenávistí
proč jsi mi fauste řekl sáhni a ber
po dlani žížala leze mi
a já chtěl dožít zde na zemi
trochu jak hrdina a trochu jak dezertér \s

sbírám svou odvahu ona je schoulená v blátě
malinký vrcholek obrovských ledových ker
oči mám bolestí podlité
kde je můj, kde je můj spasitel
který mi řekne jsi hrdina a ne dezertér



\song{sarajevo}{nohavica}

/e přes haličské pláně /a-Fis vane vítr zlý
to /H7 málo, co jsme měli, nám /e vody sebraly
jako tažní ptáci, /a-Fis jako rorýsi
/H7 letíme nad zemí, dva /e modré dopisy

\R  /e ještě hoří oheň a /a praská dřevo
    /D7-Fis ale už je čas jít /{G H7} spát
    /e tamhle za kopcem je /a sarajevo
    tam /H7 budeme se zítra ráno /e brát **

farář v kostele nás sváže navěky
věnec tamaryšku pak hodí do řeky
voda popluje zpátky do moře
my dva tady dole a nebe nahoře

\r  ještě...

postavím ti dům z bílého kamení
dubovými prkny on bude roubený
aby každý věděl, že jsem tě měl rád
postavím ho pevný, navěky bude stát

\R  ještě hoří oheň a praská dřevo
    ale už je čas jít spát
    tamhle za kopcem je sarajevo
    tam zítra budeme se, lásko, brát ... **



\song{pochod marodů}{nohavica}


kra/d bička cigaret a /F do kafe /C rum, /B rum, /d rum
dvě vodky a fernet a teď, /F doktore, /C čum, /B čum, /d čum
chra/g pot v hru/B dním ko/d ši, no /g to je /B záži/A tek
/d my jsme kámoši řidi/F čů sani/C tek, -/B tek, -/d tek \S

měli jsme ledviny, ale už jsou nadranc, -dranc, -dranc
i tělní dutiny už ztratily glanc, glanc, glanc
u srdce divný zvuk, co je to, nemám šajn
je to vlastně fuk, žijem fajn, žijem fajn, fajn, fajn

\R  /d cirhóza, /F trombóza, /C dávivý /F kašel
    /g tuberku/d lóza - /A jó, to je /d naše!
    neuróza, /F skleróza, /C ohnutá /F záda
    /g paraden/d tóza, no /A to je pa/d ráda!
    jsme /g slabí na tě/d le, ale /C silní na du/F chu
    /g žijem vese/d le, /A juchuchuchu/d chu! **

už kolem nás chodí pepka mrtvice, -ce, -ce
tak pozor, marodi, je zlá velice, -ce, -ce
zná naše adresy a je to čiperka
koho chce, najde si, ten natáhne perka, -rka, -rka \s

zítra nás odvezou, bude veselo, -lo, -lo
medici vylezou na naše tělo, -lo, -lo
budou nám řezati ty naše vnitřnosti
a přitom zpívati ze samé radosti, -sti, -sti

\R  zpívati: cirhóza, trombóza, dávivý kašel
    tuberkulóza, hele, já jsem to našel!
    neuróza, skleróza, křivičná záda
    paradentóza, no to je paráda!
    byli slabí na těle, ale silní na duchu
    žili vesele, než měli poruchu **



\song{divoké koně}{nohavica}

\[ /e já viděl divoké koně, /G běželi /a soumra/e kem, \]
   /a vzduch /e těžký /a byl a divně /e voněl /Adim tabáke/C m
   /a vzduch /e těžký /a byl a divně /e voněl /H7 tabá/e kem \S

\[ běželi, běželi bez uzdy a sedla krajinou řek a hor \]
\[ sper to čert, jaká touha je to vedla za obzor? \] \s

\[ snad vesmír nad vesmírem, snad lístek na věčnost \]
\[ naše touho, ještě neumírej, sil máme dost \] \s

\[ v nozdrách sládne zápach klisen na břehu jezera \]
\[ milování je divoká píseň večera \] \s

\[ stébla trávy sklání hlavu, staví se do šiku \]
\[ král s dvořany přijíždí na popravu zbojníků \] \s

\[ chtěl bych jak divoký kůň běžet běžet nemyslet na návrat \]
\[ s koňskými handlíři vyrazit dveře, to bych rád \] \s

já viděl divoké koně ...



\Song{petěrburg}{nohavica}

/a když se snáší noc na střechy petěrburgu, /F padá /E na mě ž/a al
zatoulaný pes nevzal si ani kůrku chl/F eba, kterou js/E em mu d/a al

\R  \[ /C lásku moji /d kníže i/E gor si bere
    /F nad sklenkou /{Disdim H7} vodky hraju si s /E revolverem
    /a havran usedá na střechy petěrburgu, /F čert a/E by to /a spral \] **

nad obzorem letí ptáci slepí v záři červánků
moje duše, široširá stepi, máš na kahánku

\R  \[ mému žalu na světě není rovno
    vy jste tím vinna, naděždo ivanovno
    vy jste tím vinna, až mě zítra najdou s dírou ve spánku \] **



\song{dokud se zpívá}{nohavica}

z /C těšína /e vyjíždí /d7 vlaky co /F čtvrthodi/{C e d7 G} nu
/C včera jsem /e nespal a /d7 ani dnes /F nespoči/{C e d7 G} nu
/F svatý me/G dard, můj pa/C tron, ťuká /a si na če/G lo
ale /F dokud se /G zpívá, /F ještě se /G neumře/C lo, /{e d7 G} hóhó \S

ve stánku koupím si housku a slané tyčky
srdce mám pro lásku a hlavu pro písničky
ze školy dobře vím, co by se dělat mělo
ale dokud se zpívá, ještě se neumřelo, hóhó \s

do alba jízdenek lepím si další jednu
vyjel jsem před chvílí, konec je v nedohlednu
za oknem míhá se život jak leporelo
ale dokud se zpívá, ještě se neumřelo, hóhó \s

stokrát jsem prohloupil a stokrát platil draze
houpe to, houpe to na housenkové dráze
i kdyby supi se slítali na mé tělo
tak dokud se zpívá, ještě se neumřelo, hóhó \s

z těšína vyjíždí vlaky až na kraj světa
zvedl jsem telefon a ptám se:``lidi, jste tam?''
a z veliké dálky do uší mi zaznělo
\[ že dokud se zpívá, ještě se neumřelo \]



\song{dál se háže kamením a píská}{nohavica}

/d ve městě jménem jeruzalém, v hlavním městě římské /Fdim kolonie
na sklonku /A7 velikonoc pilát si ruce my/d je
otrhaný ježíš stojí opodál, dav se mu smě/Fdim je zblízka
jak je to /A7 dál? dál se háže kamením a /{d g d A7 d} píská ...

\R  /d kříže se nemění, jen příjmení a jména
    když letí kamení, jde /Fdim kolem huby pěna
    /g potom se omluvíme, /d pomníky postavíme
    /A7 a housky posvačí/d me **

břevnovským klášterem jde dým, kříže se pálí
ve jménu kalicha oltáře rozsekali
a dobrý muž schovaný za portál křičí:``to ďábel spískal!''
jak je to dál? dál se háže kamením a píská ...

\R  kříže se nemění, jen příjmení a jména
    když letí kamení, jde kolem huby pěna
    potom se omluvíme, pomníky postavíme
    a hlavy ozdobíme **

kostel v arles se kácí dle žabí perspektivy
bláznivá kráso, vincent je jurodivý
profesor akademie předvádí lineál:``jak radno viděti, je to ryska,''
jak je to dál? dál se háže kamením a píská, na na na ...

\r  kříže...

malý muž s kytarou na pódiu otvírá vrátka
pravda je v proscéniu, krutá a bez pozlátka
a lidé opouštějí sál, zaujmou stanoviska
jak je to dál? dál se háže kamením a píská, na na na ...

\rr

/d ve městě jménem ..., to je vlastně jedno
na sklonku přítomnosti a /Fdim historie
miliarda /A7 pilátů zase si ruce my/d je ...



\song{kapr}{nohavica}

/C na rybníce jsou velké vl/d ny, dělají /G7 se od příro/C dy
kapr, ten je opatr/d ný, nevystr/G7 čí čumák z vo/C dy \S

\R  \[ /C šplouchy šplouchy, štika žere mou/d chy
    žbluňky žbluň/G7 ky, a kapr meruň/C ky \] **

nevylézej, kapře milý, je tam velké vlnobití
chlupy vodě se naježily, mohl by ses utopiti

\r šplouchy...



\Song{ukolébavka}{nohavica}

/G den už se seše/G\^F řil, /G\^E už jste si dost /G\^F uži/G li
tak /a hajdy do /a\^G peřin, a /a\^{FIS} ne, abyste tam moc /a\^G řádi/a li
/G zítra je taky /G\^F den, /G\^E slunko mi to dneska /G\^F slíbi/G lo
/a přejme si hezký /a\^G sen, a /a\^{FIS} kéž by se nám to /a\^G splni/a lo
/G na na /{e h e} na ..., aby hůř /a nebylo, to by nám /D7 stačilo

\R  /G hajduly, /C dajdu/G ly, aby víčka /D7 sklapnu/G ly
    hajduly /C daj/G dy, každý svou peřinu /{D7 G} najdi
    hajajajajaja/{D7 G} jaja, kuba, lenka, máma /{D7 G} a já
    zítra, dřív než slunce /C začne /G hřát ...,
    dobrou noc /{D7 G} a spát **

v noci někdy chodí strach, srdce náhle dělá buch-buchy
nebojte, já spím na dosah, když mě zavoláte, zbiju zlé duchy
zítra je taky den, slunko mi to dneska slíbilo
přejme si hezký sen, a kéž by se nám to splnilo
na na na ..., aby hůř nebylo, to by nám stačilo

\r + a spát, a spát ...



\song{lachtani}{nohavica}

\R  \[ /C lach lach /F jé /C jé, /a lach lach /G jé /C jé \] **

/C jedna lachtaní /F rodi/C na
/a rozhodla se, že si vyjde /G do ki/C na
jeli vlakem, metrem, lodí a pak /F tramva/C jí
a teď /a u kina vesmír /G lachta/C jí
/G lachtaní úspory /C dali dohromady
/G koupili si lístky /C do první /G řady
/C táta lachtan řekl:``nebudem t/F řít bí/C du''
a /a pro každého koupil pytlík /G araši/C dů, ó \S

\r  lach lach...

na jižním pólu je nehezky
a tak lachtani si vyjeli na grotesky
těšili se, jak bude veselo
když zazněl gong a v sále se setmělo
co to ale vidí jejich lachtaní zraky:
sníh a mráz a sněhové mraky
pro veliký úspěch změna programu
dnes dáváme film ze života lachtanů, ó

\rr

táta lachtan vyskočil ze sedadla
nevídaná zlost ho popadla:
``proto jsem se netrmácel přes celý svět
abych tady v kině mrznul jako turecký med
tady zima, doma zima, všude jen chlad
kde má chudák lachtan relaxovat?''
nedivte se té lachtaní rodině
že pak rozšlapala arašidy po kině, jé

\rr

/C tahle lachtaní /F rodi/C na
/a od té doby nechodí už /G do ki/C na, jé



\song{zatanči}{nohavica}

/e zatan/G či, má milá, /D zatanči /e pro mé oči
zatan/G či a vetkni /D nůž do mých za/e d
ať tvůj š/G at, má milá, /D ať tvůj šat n/e a zemi skončí
ať tvůj š/G at, má milá, /D rázem je sň/e at \S

\R  zatanči, jako se okolo ohně tančí
    zatanči jako na vodě loď
    zatanči jako to slunce mezi pomeranči
    zatanči, a pak ke mně pojď **

polož dlaň, má milá, polož dlaň na má prsa
polož dlaň nestoudně na moji hruď
obejmi, má milá, obejmi moje bedra
obejmi je pevně a mojí buď

\r zatanči...

nový den než začne, má milá, nežli začne
nový den než začne, nasyť můj hlad
zatanči, má milá, pro moje oči lačné
zatanči a já budu ti hrát

\rr



\Song{šnečí blues}{nohavica}

/G jednou /C7 jeden /{G D\^{FIS}} šnek --- /G šíle/C ně se l/{G D7} ek'
/G nikdo už dnes /G7 neví, z /C čeho se tak /c zjevil
že se /G dal hned /{C7 D7 G D7} na útěk \S

přes les a mýtinu rychlostí půl metru za hodinu
z ulity pára, ohnivá čára
měl cihlu na plynu \songgg

ale v jedné zatáčce, tam v mechu u svlačce
udělal šnek chybu, nevyhnul se hřibu
nevyhnul se bouračce \s

hned seběhl se celý les a dali šneka pod pařez
tam v tom lesním stínu, jestli nezahynul
tak leží ještě dnes \s

a kdyby použil vůz anebo autobus
\[ nebylo by nutné zpívat tohle smutné
   smutné šnečí blues \]



\Song{stodoly}{nohavica}

/D za kopci hořely stodoly
/ havrani kroužili nad poli
/e stařenky /G smály se do /D dlaní
/e navěky /G jsme spolu sváza/D ní \S

na seně stébla nás píchala
oblaka zrychleně dýchala
rozpálen sladkými přísliby
oči zavřel jsem, abych tě políbil \s

kristus pán sekerou tesaný
díval se z božích muk na stráni
svatou mši sloužili v kostele
a mé tělo stalo se tvým tělem \s

křičel jsem, lásko má, na hory, na doly
až jsem vyplašil ony havrany v údolí
vylétli, jako když vystřelí
a pak pod nebem stali se bílými anděly
/D pod nebem bílými anděly...



\song{těšínská}{nohavica}

/a kdybych se narodil /d před sto lety v /F tom/E hle /a měst/{d F E a} ě
u larischů na zahradě /d trhal bych květ/F y sv/E é ne/a věst/{d F E a} ě
/C moje nevěsta by /d byla dcera ševcova
z /F domu kamiňskich /C odněkud ze lvova
kochal bym ja i /d pieščil, /F chyb/E a lat /a dwiešči/{d F E a} e \S

bydleli bychom na sachsenbergu v domě u žida kohna
nejhezčí ze všech těšínských šperků byla by ona
mluvila by polsky a trochu česky
pár slov německy, a smála by se hezky
jednou za sto let zázrak se koná, zázrak se koná \s

kdybych se narodil před sto lety, byl bych vazačem knih
u prohazků dělal bych od pěti do pěti a sedm zlatek za to bral \rest{bych}
měl bych krásnou ženu a tři děti
zdraví bych měl a bylo by mi kolem třiceti
celý dlouhý život před sebou, celé krásné dvacáté století \s

kdybych se narodil před sto lety v jinačí době
u larischů na zahradě trhal bych květy, má lásko, tobě
tramvaj by jezdila přes řeku nahoru
slunce by zvedalo hraniční závoru
a z oken voněl by sváteční oběd \s

večer by zněla od mojzese melodie dávnověká
bylo by léto tisíc devět set deset, za domem by tekla řeka
vidím to jako dnes: šťastného sebe
ženu a děti a těšínské nebe
ještě že člověk nikdy neví, co ho čeká \s

na na /{a d F E a d F E a} na ...



\song{starý muž}{nohavica}

/G až budu starým mužem, /G\^H budu staré knihy /e číst
a mladé /C víno /D lisovat
/G až budu starým mužem, /G\^H budu si konečně /e jist
tím, /C koho chci /D milovat
/G koupím si pergamen a št/G\^H ětec a tuš
a jako /e čínský mudrc sednu na břeh /D řeky
a budu /G starý mu/{G\^H e C D} ž
/G starý mu/{G\^H e C D} ž \S

až budu starým mužem, pořídím si starý byt
a jedno staré rádio
až budu starým mužem, budu svoje místo mít
u okna kavárny avion
koupím si pergamen, štětec a tuš
a budu pozorovat lidi, kam jdou asi
a budu starý muž, a budu starý muž \s

/{C  C\^H a C\^H C C\^H a C\^H C C\^H a C\^H C C\^H D} na na na ... \S

až budu starým mužem, budu černý oblek mít
a šedou vázanku
až budu starým mužem, budu místo vody pít
lahodné víno ze džbánku
koupím si pergamen, štětec a tuš
a budu mlčet, jako mlčí ti, kdo vědí už
starý muž, starý muž

   

\song{alasdair, syn collův}{asonance}

/e alasdaire, oho
/ synu collův, o ho
/ do tvých rukou, o ho
/C vkládám osud /D skotské /e země \S

/ slavné činy
/ budeš konat
/ ty nás povedeš
/C vždyť jsi jediným /a synem veli/D kého /e krále \S

pán z ach-nam-breac, o ho
bude zabit, o ho
tvojí rukou, o ho
pohřběte jej do údolí \s

alasdaire, o ho
synu collův, o ho
zvedni hlavu, o ho
země čeká na tvá slova \s

nepřátelé, o ho
ve svém strachu, o ho
budou prchat, o ho
překvapíš je za svítání \s

zdálo se mi, o ho
že jsem viděl, o ho
dneska ráno, o ho
v rozvalinách celé město



\song{omnia vincit amor}{klíč}

F C d a B C d 
/d šel pocestný kol /C hospodských /d zdí
/F přisedl k nám a /C lokálem /F zní
/g pozdrav jak svaté /F přikázá/C ní
/d omnia /C vincit /d amor \s

hej, šenkýři, dej plný džbán
ať chasa ví, kdo k stolu je zván
se mnou ať zpívá, kdo za své vzal
omnia vincit amor \s

\R zlaťák /F pálí, /C nesleví /d nic
   štěstí v /F lásce /C znamená /F víc
   všechny /g pány /F ať vezme /{C A} ďas
   /d omnia /C vincit /d amor **

já viděl zem válkou se chvět
musel se bít a nenávidět
v plamenech pálit prosby a pláč
omnia vincit amor \s

zlý trubky troubí, vítězí zášť
nad lidskou láskou roztáhli plášť
v tom kdosi krví napsal ten vzkaz
omnia vincit amor \s

\r zlaťák...

já prošel každou z nejdelších cest
všude se ptal, co značí ta zvěst
až řekl moudrý, pochopíš sám
omnia vincit amor (všechno přemáhá láska) \s

\rr

teď s novou vírou obcházím svět
má hlava zšedla pod tíhou let
každého zdravím větou všech vět
omnia vincit amor



\song{čarodějnice z amesbury}{asonance}

zuzana /d byla dívka, /C která žila v /d amesbury
s jasnýma /F očima a /C řečmi pánům /d navzdory
souse/F dé o ní /C říkali, že /d temná kouzla /a zná
a /B že se lidem /a vyhýbá a s /B ďáblem /C pletky /d má \S

onoho léta náhle mor dobytek zachvátil
a pověrčivý lid se na pastora obrátil
že znají tu moc nečistou, jež krávy zabíjí
a odkud ta moc vychází, to každý dobře ví \s

tak zuzanu hned před tribunál předvést nechali
a když ji vedli městem, všichni kolem volali:
už konec je s tvým řáděním, už nám neuškodíš
teď na své cestě poslední do pekla poletíš! \s

dosvědčil jeden sedlák, že zná její umění
ďábelským kouzlem prý se v netopýra promění
a v noci nad krajinou létává pod černou oblohou
sedlákům krávy zabíjí tou mocí čarovnou \s

jiný zas na kříž přísahal, že její kouzla zná
v noci se v černou kočku mění dívka líbezná
je třeba jednou provždy ukončit ďábelské řádění
a všichni křičeli jak posedlí: na šibenici s ní! \s

spektrální důkazy pečlivě byly zváženy
pak z tribunálu povstal starý soudce vážený
je přece v knize psáno: nenecháš čarodějnici žít
a před ďáblovým učením budeš se na pozoru mít! \s

zuzana stála krásná s hlavou hrdě vztyčenou
a její slova zněla klenbou s tichou ozvěnou:
pohrdám vámi, neznáte nic nežli samou lež a klam
pro tvrdost vašich srdcí jen, jen pro ni umírám! \s

tak vzali zuzanu na kopec pod šibenici
a všude kolem ní se sběhly davy běsnící
a ona stála bezbranná, však s hlavou vztyčenou
zemřela tiše, samotná, pod letní oblohou



\song{lokaj}{klíč}

\R  /D kde zas vězí ten /G spící /D lokaj, /A kolo/D hnát
    omlouvám se /D panstvo, račte jen, /G tu je /D salón /A a tam /D sad
    /d zákusky b/C udou /F hned i vína /C je /F dost
    hněte se /F žán, kde /C je /F led, tohle vzácný /C je /A host
    a /D co je s muzikou /G a ten /D pejsek /A chtěl by /D kost
    bravo žán -- /D teď si smíte dát /G kvelbu /D sklenku /A pro ra/D dost **

/d chtěl jsem jí jednou jen lásko má /C říci
a ona na to: žán, tak vyměň tu /d svíci!

/d chřadnu, vadnu, /F láska je /C hořká jak /F blín
a ona: /C hlavně /a vyciď ten /d cín

/d povězte mi /F přátelé /C kdo tohle může /F snést
teď jdu venčit /C psa a /a pak musím \S

\r  kde zas vězí...

psal jsem jí psaníčko: krásko buď zdráva
a ona na to: žán, kdy bude ta káva?!
tuhle jsem jen hles: když chceš, tak si mě muč
prý: zapal ten krb a proboha mlč!

povězte mi přátelé kdo toto může snést
teď jdu utřít prach a pak musím

\R  kde zas vězí
    \ \vdots
    bravo žán -- teď si smíte dát kvelbu sklenku --- a pak dost! **




\song{kopyta a hříva}{klíč}

jsou /G kopy/H7 ta a /e hříva a /a čabraky a /D šporny
skrz ně /G kůň se /H7 na svět /e dívá a /a smíchy /D chtěl by /{G D (C G)} ržát \s

když v sedle stařík zpívá, že vede život vzorný
kůň moudře hlavou kývá a čeledín je mlád

\R  /H7 o tý holce, /e která v stáji /H7 dává koňům /e pít
    /D klisny tiše /G řehtají, že /a hříbě bude /H7 mít **

jsou jen kopyta a hříva a čabraky a šporny
a kůň se na svět dívá a smíchy chtěl by ržát \s

jsou kopyta a hříva a pán si v sedle mlaskne
jó, až to na něj praskne, to teprv začne cval \s

teď v slunci srst se svítí a přes potok si skáčem
už zejtra budem s pláčem sedlat opodál

\R  starý jezdec bez otěží do chomoutu vlít'
    o hřebce tu neběží, vždyť holka chce si žít **

budou prstýnky a hříva a čeledín je mlád
a proto kůň jen krkem kývá a smíchy chtěl by ržát \krat2



\song{prachskra}{klíč}

\R  /a dneska třu /E nouzi, /a zejtra bídu /E mám, prachsakra
    v /a cárech se /E vzbouzím, v /a hadrech usí/E nám
    /a a když jsem v /E tísni, /a a když /E smůlu /a mám
    tak natruc touhletou /E písní /a vztek si /E vylej/a vám **

/a dost vzduchu /E mám, ten /a sotva vydej/E chám, prachsakra
/a jít není /E kam, tak /a nikam nespě/E chám
/E7 lásku jsem měl, ta řekla: žij si /A sám
já natruc /E7 nejdřív jsem klel, a teď si /A zazpívám o tom, že:

\r dneska...

můj život zkrátka odvyk' dětskejm hrám, prachsakra
zaklapla vrátka, vyprodanej krám
co přijde pak, já zázrak nečekám
a natruc kytku si jen tak za klobouček dám

\rr




\song{víno}{klíč}

\R  /e víno teď /D nalej nám, /e ať si každej, /C co chce, /H7 zpívá
    /e vína /D plnej džbán, dnes /e do rána /H7 budeme /e pít
    víno nalej nám, po boji jen žízeň zbývá
    zítra někdo z nás třeba už nebude žít **

/e než přijde /D den, každý /C má /H7 času /e dost
zbývá jen /D zapít žal, /G klít /D a mít /G zlost
že zas je nás /D čím dál míň, /e víra je tu jen /H7 host
/e proč se /D máme bít /C druhejm pr/H7 o ra/e dost \S

\R  víno teď nalej nám, ať si každej, co chce, zpívá
    vína plnej džbán, dnes do rána budeme pít **

krátí se noc, v krčmě všem slábne hlas
ráno zavelí a kdekdo zláme vaz
\[ už zní rozkaz, dál bít se můžem, třeba i krást
ten, kdo přežije, tomu zbyde chlast \]

\R  víno teď nalej nám, ať si každej, co chce, zpívá
    vína plnej džbán, dnes do rána budeme pít
    víno nalej nám, po boji jen žízeň zbývá
    zítra někdo z nás třeba už nebude žít. /{emi d emi h7 emi} **


\R  /fis víno teď /E nalej nám, /fis ať si každej, /D co chce, zpívá
    /fis vína /E plnej džbán, dnes /fis do rána /C budeme /fis pít
    /a zpívej až /E do rána, /fis není každá bitva /D prohran/C á
    /fis dojde i n/e a pána, a /fis pak si bu/C deme /fis žít
    na na na ... **



\song{mince}{klíč}

u /a pasu v měšci potají, /G poslouchej ten /a zvuk
mince si tiše /C cinkají ``/E cinky-linky-/{a E} drnk''
/a jak koho dneska utěší, /G komu bude /a hej
děvče si rádo /C zahřeší, /E vždyť je na pro/{a C\^H} dej: \S

\R  /C šátek i srdce z perníku, /G čokoláda, víno
    /a mince se dají, smilníku, /E břinkavě do křiku, /F hej
    muziku /C chystají, k zlatníku /E spěj a své zlatky mu /a dej (2x) **

hned nato v prahu u cesty vedle bodláčí
z topolů listy zachřestí --- smích je roztančí
svůj život zpátky na úvěr nemůžeš si brát
tvůj měšec plný děr neví, že je listopad

\R  mince jak sny se ztrácejí, po zemi se válí
    tys v pozlacené aleji stál, spíš jak hadrák než král
    oči pálí a mince se válí a listí a hrob opodál **

náramek na tvém zápěstí láskou zazvoní
leskne se dukát pro štěstí v stínu mandloní
touha se hrozny červená jako vinohrad
dívka je předem ztracená, vždyť jí můžeš dát:

\R  šátek i srdce z perníku, čokoládu, víno
    mince se dají, smilníku, břinkavě do křiku, hej
    muziku chystají, k zlatníku spěj a své zlatky mu dej **

\R  je tu šátek i srdce z perníku, čokoláda, víno
    mince se dají, smilníku, břinkavě do křiku, hej
    muziku chystají, k zlatníku spěj a své zlatky mu dej **

\R  šátek i srdce z perníku, čokoláda, víno
    mince se dají, smilníku, břinkavě do křiku, hej
    muziku chystají, k zlatníku spěj a své zlatky mu dej (2x) **



\song{zum zum}{dobeš}

/G zpívají si o tom vrabci na ro/D7 kytě
že učenec je horší nežli /G dítě
se žábami hraje si pan /D7 galvani
archimédes potápí se /G do vany
a nepoučen událostmi /D7 ráje
isaac newton s jabkama si /G hraje

\R  /G zum zum zum /D7 zum
    / a nejde mi to do kebule
    /G zum zum zum /D7 zum
    a nejde mi to na ro/{G D7} zum **

bylo to jako výbuch jako salva
když se žárovkou přišel thomas alva
do pochodu vyhrávaly kapely
muži pili šampus, ženy šílely
jak když pustíš tygry do arény
začalo se dělat na tři směny

\r  zum...

kdyby naši předci vstali z ledu
podivili by se jak jsme vpředu
jak závazky předhánějí úkoly
einstein by se těžko dostal na školy
mozart by moh dneska u klavíru
jen těžko dělat do muziky díru \s

michelangelo by sebral sochy \pozn{C dur}
a vhodil by je všecky do macochy
lumiér by zčervenal jak malina
kdybyste ho vzali s sebou do kina
jen u elektrotechnického vesla
ještě ňákou dobu moh by sedět tesla \songgg

vědeckotechnická revoluce
uvolňuje lidem obě ruce
dnes má každý vědátor už od plínky
sunarku a digitální hodinky
s optimismem hledí k stratosféře
a brano samo zavírá mu dveře

\rr

kdyby starý thales nemoh čmárat
a kreslit si do písku podle nálad
mendělejev kdyby musel, vážení
periodicky vykazovat hlášení
a osm hodin zvedat telefony
svět by stál za pytlík bikarbony

\rr

nikdo z nás by doma neměl sony
dvakrát třicet wattů, čtyři ohmy
lidé by se hnali kamsi za hmotou
regály by nejspíš zely prázdnotou
neměli bychom šajn o opeře
a válčilo by se u sudoměře \s

zem by byla rovná jako deska \pozn{C dur}
nebyla by kulatá jak dneska
adam s evou nemuseli z ráje ven
giordano bruno by nebyl upálen
jen temno jak když vstoupíš do komory
škoda každé rány z aurory \s

ze všech zvířat archy noemovy
a ze všeho co můžem popsat slovy
jen balvany a lidé mají odvahu
urvat se od skály a padat dolů po svahu
a na světě který se furt mění
překonat co překonáno není

\R  zum zum zum zum protože to co nejde do kebule
    zum zum zum zum rádo leze na rozum
    zum zum zum zum hmmm....
    zum zum zum zum rádo leze na rozum **



\song{skupinové foto}{dobeš}

kdyby mi /C kdosi řeknul /G7 že postavi metr /C piv
abych se mu za to /G7 postavil před objek/{C C7} tiv
/F prvni bych si řeknul že ho /G dlabu
a /C potom bych ho roztrhnul jak /{C7 F} žabu
vím, že by se mi to vymstil/{C a} o
ale na /C take vtipy mě /G nikdy něužil/{C G} o\S

možna vam to budě připadat banalně
ale ja sem ztratil odvahu a malem sem se octnul v pakarně
odvahu sem ztratil a kaj sem se hnul
furt mě kusek chybělo a furt mě bylo pul
bo bez odvahy se těžko cosi robi
nic tak jak odvaha chlapa něozdobi \s

dva dni sem ju hledal, dva dni byla k nenalezeni
zlomeny šel sem hlasit ztratu na odděleni
křižem kražem probirali, kde sem všude byl
s kym sem se tam potkal, ja sem jim to vyklopil
připadu se chytli fachmani
za devět hodin bylo ukončeno patrani \s

za devět hodin bych tak maximalně poryl zahradu
kolikrat eště dele čekam než ryba zhltně navnadu
když za devět hodin na vysoke peci
zrobime tři odpichy, tak zpoceni sme všeci
motorista za tu dobu nězarobi na blatnik
a mala ručička neoběhne cifernik \s

posledni dvě hodiny už sem enom čekal v čekarně
na dřevěne lavici jak v prvotřidni vinarně
v rohu seděla banda s huslami
chovali se spontanně, jak by tam byli sami
předstiral sem čteni v kulturnim měsičniku
ale pamatuju enom tolik, že basa tvrdi muziku \songgg

za devět hodin byla odvaha zpatky na světě
ten co mi ju předaval pravi dobře si ju schovejtě
dejtě si ju pod dva a nebo pod tři zamky
ja mu na to pravim co tak do švycarske banky
tam by mi ju zamkli a ja by sem měl svaty klid
a zitra rano bych se s ňu moh naposledy nechat vyfotit \s

/F komu se to hodi a kdo ma čas ten može přiji/{C G a} t
sraz je /C zitra rano v osm /G7 jdem se nechat vyfo/C tit



\song{něco o lásce}{dobeš}

/C za ledovou /F horou a černými /C lesy
je stříbrná /F řeka a za ní /C kdesi
stojí /F domek bez ad/C resy a bez de/{d G7} chu
/F bydlí v něm nechci říkat /C víla
ale co /F na tom, i kdyby /C byla
/F před lidmi se trošku /C skryla
a /d víme o ní /G7 hlavně z dosle/C chu

\R  že lidi /d rozumné blbnout /G7 nutí
    a není /C na ni nej/e menší spoleh/a nutí
    /d co ji zrovna napad/G7 ne, to udě/{C e a} lá
    z puberťáků /d chlapy a z chlapů puber/G7 ťáky
    o ženských /C nemluvím /e tam to platí /a taky
    a /d urážlivá je a /G7 hořkosladkokyse/C lá **

genetičtí /d inženýři /G7 lámou její /C kód
/a po praze se /d o nich šíří /G7 že jezdí tramva/C jí
/a strkají /d hlavy /G7 po vodo/C vod
/a a pak i /d oni nakonec /G7 podléha/C jí \S

\R  a holubicím dál rostou křídla dravců
    družstevním rolníkům touha mořeplavců
    a lásce té potvoře sebevědomí
    že jednou bude vládnout světem
    tedy i nám a po nás našim dětem
    které na tom budou stejně špatně jako my ** 
\songgg{}

\R  když chlap zmagoří láskou utíká za ní
    platí i s úroky a nepočítá s daní
    u ženských je to přímo námět na horror
    papuče letí pod pohovku
    nákupní tašky padají na vozovku
    ať si tramvaj zvoní ať se zblázní semafor **

až vám ta potvora zastoupí cestu
sedněte na zadek a seďte jak z trestu
jen ať si táhne jak to dělají vandráci
láska se totiž, i když je prevít
nikomu dvakrát nemůže zjevit
láska se totiž, i když je prevít, nevrací \s

a nesmí vám to nikdy přijít líto
kupte si auto a cucejte chito
odreagujte se psychicky
protože jestli byste na ni měli myslet
to radši vstaňte a jděte z ní ihned
utíkejte než vám zmizí navždycky

\R  převrhněte stůl, opusťte dům
    fíkusy rozdejte sousedům
    nechte vanu vanou, ať si přeteče
    na světě není větší víra
    pro žádnou z nich se tolik neumírá
    ani v žádné jiné zemi na světě **

\r  hmmm...



\song{jarmila}{dobeš}

/G jarmila vždycky mi /h radila
abych /D7 pracovní dobu dodr/{G D7} žel
/G dneska mě ale /h náramně táhlo /D7 domů
a tak jsem prostě /{G D} šel
/e jarmila má totiž dneska narozeniny
/D7 proto jsem dnes přišel dříve o dvě hodiny
na stole /G sklenice, smích slyšet /h z ložnice
v předsíni /D7 stojí pánské střeví/{G D7} ce \S

vytahuji z aktovky květiny
uvažuji kdo asi přijel z rodiny
tipuji nejspíše na strýce
kdo jiný měl by přístup až do ložnice
kdo jiný kdo jiný než strejda z dědiny
vzpomenul si na jarmilu nejsem jediný
v ruce mám kytici, už stojím v ložnici
vidím že nevymřem po přeslici \s

kdepak jejda --- není to strejda
františku, ty máš boty úplně jak on
přičemž nechávám prostor úvahám
vyhledávám optimální tón
kterým bych oběma jednak pohanil
ale ani jediného slovem nezranil
takže jsem chvíli stál, pak říkám krucinál
tebe bych soudruhu tady nehledal \s

dodnes mě mrzí, že jsem byl drzý
a že jsem pracovní kázeň porušil
dřív než o hodinu vypnul jsem mašinu
a tím jsem rozdělanou práci přerušil
oba si mě postavili na kobereček
to, jak zle mi vyčinili, nedal jsem si za rámeček
z nevěry nedělám závěry
mrzí mě že jsem u nich pozbyl důvěry



\song{holubí dům}{}

/d zpí/C vám /B ptákům a /a zvlášť holu/d bům
stá/C val v /B údolí /a mém starý /d dům
/F ptá/C ků /F houf zalé/C tal ke kro/F vům
/d měl /C jsem /B rád holu/a bích křídel /d šum \S

vlídná dívka jim házela hrách
mávání perutí víří prach
ptáci krouží a neznají strach
měl jsem rád starý dům, jeho práh

\R  hledám /g dům holu/C bí, kdopak z /F vás cestu /d ví
    míval /g stáj roube/C nou, bílý /F štít
	kde je /g dům holu/C bí a ta /F dívka kde /d spí
	vždyť to /g ví, že jsem /a chtěl pro ni /d žít **

sdílný déšť vypráví okapům
bláhový kdo hledá tenhle dům
odrůstáš chlapeckým střevícům
neslyšíš holubích křídel šum



\song{mraveneček}{}

/G polámal se mraveneček
ví to /a celá obora
o půl/D noci zavolali
mraven/C čího dokto/G ra \S

doktor klepe na srdíčko
potom píše recepis
třikrát denně prášek cukru
bude chlapík jako rys \s

dali prášek podle rady
mraveneček stůně dál
celý den byl jako v ohni
celou noc jim proplakal \s

čtyři stáli u postýlky
pátý těšil: neplakej
pofoukám tě na bolístku
do rána ti bude hej \s

pofoukal ho na bolístku
pohladil ho po čele
ráno zdravý mraveneček
skáče dolů z postele



\song{děti jdou kam je pošlou}{spirituál kvintet}

děti /G jdou kam je pošlou, jdou kam je pošlou ó
/C a jak /D7 jdou tak /G melou svou, že prý: \S

deset bylo besed, kde rád si nikdo nesed'
kejt bylo devět a sněd je sám nedvěd
dál osmý stál ten král co se jen bál
sedum dědů k dědům se posadilo před dům
šest bylo měst, kde říkali ``čest!''
pět bylo vět, co podepsali hned
ven ze čtyř stěn kdo chodí je ctěn
tři byli páni dáni
dva chytli lva co couvá
sám zůstal sámo se svou slávou
jdou, /C jdou, /G jdou a /F melou /G svou...




\Song{beránek a vlk}{spirituál kvintet}

\R riu, riu, čiu, píseň se tu zpívá
   \[ před beránkem v hrůze zlý vlk, že se skrývá \] **

/d ve dne ukryt /F bývá, stá/C da v noci /d loupí
/d marně boha /F vzývá, kdo /C do cesty mu /d vstoupí
/d vale ať dá /F spáse, vlk /d ne/C zná /B slito/A vání
jen /d ovcím vyhý/F bá se, když /C beránek je /d chrání

\r riu, riu,...

čím víc strach se šíří, tím víc vlk si troufá
už i do vsí míří, kde v bázni lid si zoufá
nikdo nemá zdání, že na ty zuby dračí
není třeba zbraní, být s beránkem že stačí

\rr

jednou přijde chvíle, účet kdy se splácí
kdo věřil jen síle, ten sílu náhle ztrácí
vlci strachem vyjí, to jim teď zvoní hrana
že jim srazí šíji dnes od berana rána!



\song{margot}{spirituál kvintet}

\R \[ margot češe na vinici v koši- v koši- v košili
    margot češe na vinici jen tak \] **

\[ kdopak to jde po silnici, margot \]
šlapou si tu tři chasníci v koši- v koši- v košili
šlapou si tu tři chasníci jen tak \s

\r margot češe...

\[ dej nám vína do čepice, margot! \]
dám vám bičem na zadnice v koši- ... \s

\rr

\[ pročpak ztichla tvoje ústa, margot? \]
dva utekli, třetí zůstal, v koši- ... \s

\rr

\[ v sadě kvetou rozmarýnky, margot! \]
margot brečí pere plínky, v koši- ...




\Song{když jste nás pozvali}{spirituál kvintet}

přijďte k nám, přijďte k nám,... \s

/G když jste nás pozvali, /D7 tak nás tady /G mějte
my od vás nepudem, /D7 až nás vyže/G nete! \S

hajdy ven, hajdy ven,... \s

když nás vyženete, my vám venku zmoknem
když dveře zamknete, vrátíme se oknem!




\song{veličenstvo kat}{kryl}

/d v ponurém osvětlení /C gotického /d sálu
/F kupčíci vyděšení /g hledí do mi/A sálů
/d a houfec /B mordýřů si /C žádá pože/F hná/C ní
\[ /g vždyť první z /d rytířů je /A7 veličenstvo /d kat \] \S

kněz -- ďábel, co mši slouží, z oprátky má štolu
pod fialovou komží láhev vitriolu
pach síry z hmoždířů se valí k rudé kápi
\[ prvního z rytířů, hle: veličenstvo kat \]

\R  /F na korouhvi /C státu /B je emblém s /C gilotinou
    z /F ostnatýho /C drátu /B páchne to /C shnilotinou
    /g v kraji hnízdí hejno krkav/d čí
    /g lidu vládne mistr poprav/A čí **

král klečí před satanem na žezlo se těší
a lůza pod platanem radu moudrých věší
a zástup kacířů se raduje a jásá
\[ vždyť prvním z rytířů je veličenstvo kat \] \s

na rohu ulice vrah o morálce káže
před vraty věznice se procházejí stráže
z vojenských pancířů vstříc černý nápis hlásá
\[ že prvním z rytířů je veličenstvo kat \]

\R  nad palácem vlády ční prapor s gilotinou
    děti mají rády kornouty se zmrzlinou
    soudcové se na ně zlobili
    zmrzlináře dětem zabili **

byl hrozný tento stát, když musel jsi se dívat
jak zakázali psát a zakázali zpívat
a bylo jim to málo, poručili dětem
\[ modlit se jak si přálo veličenstvo kat \] \s

s úšklebkem ďábel viděl pro každého podíl
syn otce nenáviděl, bratr bratru škodil
jen motýl smrtihlas se nad tou zemí vznáší
\[ kde v kruhu tupých hlav dlí veličenstvo kat \]



\song{kužel}{nedvědi}

/d v kuželu světel půlnočních vlaků koleje les
tu a tam křídla zmatenejch ptáků přejetej /G pes
a v křoví /g u trati ještě pár tuláků co /d hledaj kde jak spát
kolik je /F hvězd, tolik je /G cest, když chceš si /d hrát \S

v kamenech tratí tu a tam květ, bláto a mech
s rachotem letí semafor vzhůru, jak si se lek
ke skále přitisknutý doufáš, že dveře žádný nebudou vlát
za vlakem vítr bude si s listím plakátů hrát

\R  \[ jak /G stejská se když /d tejden /g vleče se jak /d pluh
    když /G práce, spěch a /d spánek /g jsou jak jeden /d kruh \] **

v kuželu světel půlnočních vlaků koleje, les
tu a tam křídla zmatenejch ptáků, přejetej pes
a tichej netopýr nad hlavou přelítne --- možná usnul ... a spad
tolik je hvězd, tolik je cest, když chceš si hrát

\r jak stejská...



\song{safenat paneah fi.muni}{pár.studentů.fi.muni.cz}

\R  /d do prachu padněte tváří, přichází pán
    / chodbami fakulty blíží se brandejs sám
    /C ten kdo si na krku hlavu chce ponechat
    /d do prachu /C padne a /d rád **

/d hej ty chlapče za tou erinou
odloguj se a podej mi hned kartu /C svou
cvt hned tě vyprovo/G dí
/g dva týdny /B chládku ti /A neuško/d dí \S

kroky nohsledů učebnou zní
brandy se usmívá, celý září
to přece celý život si přál
konečně cítí se jak samotný král

\R  do prachu padněte tváří, přichází pán
    chodbami fakulty blíží se brandejs sám
    ten, kdo si účet svůj ještě chce ponechat
    kartu si připne a rád **

vládcem fakulty cítím se být
konečně tahám tu za každou nit
záviděl by mi zlatuška sám
teď jsem tu jediný, jediný pán!



\song{alison gross}{asonance}

když /d zapadlo /C slunce a /B vkradla se /A noc
a v /d šedivých /C mracích se /B ztrá/g cel /A den
a /d když síly /C zla ve tmě /F převzaly /A moc
tu /d alison /C gross vyšla z hradu /d ven \S

tiše se vplížila na můj dvůr
a jak oknem mým na mě pohlédla
tak jen kývla prstem a já musel jít
a do komnat svých si mě odvedla

\R  /d alison gross a černý /C hrad
    ze /d zlověstných /C skal jeho hradby /d ční
    alison gross, nejodporněj/C ší
    ze všech /F čaroděj/C nic v zemi sever/{G A} ní **

složila mou hlavu na svůj klín
a sladkého vína mi dala pít 
já můžu ti slávu i bohatství dát
jen kdybys mě chtěl za milenku mít \s

mlč a zmiz babo odporná!
slepý jak krtek bych musel být!
to radši bych na špalek hlavu chtěl dát
než alison gross za milenku mít!

\r  alison gross...

přinesla plášť celý z hedvábí
zlatem a stříbrem se celý skvěl
kdybys jen chtěl se mým milencem stát
tak dostal bys vše, co bys jenom chtěl \s

pak přinesla nádherný zlatý džbán
bílými perlami zářící
kdybys jen chtěl se mým milencem stát
těch darů bys měl plnou truhlici \s

stůj a mlč babo odporná!
slepý jak krtek bych musel být!
to radši bych na špalek hlavu chtěl dát
než milencem tvým na chvíli se stát! \songgg

\rr

tu k ohyzdným rtům zvedla černý roh
a natřikrát na ten roh troubila
a s každým tím tónem mně ubylo sil
až všechnu mou sílu mi sebrala \s

pak alison gross vzala čarovnou hůl
a nad mojí hlavou s ní kroužila
a podivná slova si zamumlala
a v slizkého hada mě zaklela

\rr

tak uplynul rok a uplynul den
a předvečer svátku všech svatých byl
a tehdy na místě, kde žil jsem jak had
se zjevila královna lesních víl \s

dotkla se mě třikrát rukou svou
a její hlas kletbu rozrazil
a tak mi zas vrátila podobu mou
že už jsem se dál v prachu neplazil

\rr



\song{můj pes}{nohavica}

/D můj p/A es je gr/h óf /A mezi ps/D y
co /e chlup to /A skvost, co zub to hrouda zlata
/D oca/A sem dě/h lá /A elipsy
/G když vrčí /A na štěňa/D ta \S

můj pes je psem z anglie
po otci lord po mámě lady
hodný na hodné lidi je
a zlý je na obejdy \s

/G můj /A pes má /D duši
/G čtyři nohy, /A ocas a /D uši
/G chlu/A paté /D tělo
/E krásné, chytré ale /A tvrdohlavé čelo

\R  /G jí jenom /A maso a /D kosti
    /G kouše /A debilní /D hosti
    /G jinak se /A chová /D vhodně
    /e já ho /A miluji /D hodně **

můj pes je psí prototyp
on modelem být moh psímu sousoší
od pondělí do soboty
prolenoší \s

můj pes je rek, hrdina
běda lumpovi jenž by na práh šláp
slupnut by byl jako malina
á ten chlap \s

můj pes má sílu
jak se má k jídlu tak i k dílu
pelech má u kredence
vysoké je yntelligence

\r jí jenom...

my dog is dog, number one, jé



\song{já se v tom nevyznám}{ebenové}

\R  \[ /d\^{podivné} já se v tom nevyznám
    já se v tom neorien/C tu/d ju
    já se v tom /a neorien/C tu/d ju \] **

už nevím, co je /d\^{ještě podivnější} vpravo
a co /g vlevo
kde /F má ta věc svou /C hlavu
a kde /A střevo \s

už se mi třesou ruce
potím se čůrkem
tohle je ještě horší
než vojna s turkem

\r já se v tom...

ke komu se mám přidat
když vyjdu z domu
vždyť já bych rád zaplatil
jen vědět komu \s

komu se podřizovat
a komu velet
kdo asi půjde k moci
kdo do pr...

\r říkám vám, že se v tom vůbec nevyznám...



\song{tichá domácnost}{ebenové}

/h7 není doma /E vždycky všechno /D\^E tak
/Amaj7 jak by si člověk /Dmaj7 představoval
/h7 někdy to de pr/E ávě nao/D\^E pak
s /Amaj7 tím bych vás nerad /Dmaj7 unavoval

\R  /h7 u nás se nekři/cis čí, /fis u nás se nespí/cis lá
    /h7 u nás je zvláštní /E idyla

    /A tichá, /D tichá /E naše /Dmaj7 domácnost je /A\^{Cis} tichá, /D tichá
    a s /E konverzací /D nikdo nepos/A píchá
    /D tichá, /E naše /Dmaj7 domácnost /A\^{Cis} tichá, /D tichá
    jen /E vodovodní /D kohout tiše vzd/cis ychá, /fis tichá
    je /G poz/D dě /G hon/D it /G bych/A a **

není doma vždycky všechno tak
jak by to žena měla v plánu
celý večer čekáte a pak:
on přijde o půl páté k ránu

\r u nás se nekřičí...



\song{soudný den}{spirituál kvintet}

/a zdál se mi sen, že se nebe hroutí
/G zdál se mi sen o poslední pouti
/a zdál se mi sen, že všechno seberou ti
v ten /E soudný /a den \S

kam běžet mám, slunce rychle chladne
kam běžet mám, měsíc na zem spadne
kam běžet mám, moře už je na dně
v ten soudný den \s

stůj, nechoď dál, času už je málo
stůj, nechoď dál, míň, než by se zdálo
stůj, nechoď dál, otevři se, skálo
v ten soudný den \s

pán tě zavolá, má pro každého místo
pán tě zavolá, jen kdo má duši čistou
pán tě zavolá, sám nedokázal bys to
v ten soudný den.

\R  /a soudí, soudí pány, slouhy
    /G soudí, soudí hříšné touhy
    /a soudí, soudí, výčet pouhý, áá/{d E} á...

    /a vtom se probudíš, to byl jen sen
    /F vtom se probudíš, to byl jen sen
    /d vtom se probudíš, to byl jen sen
    /a jen /E pouhý /a sen **

zdál se mi sen, že se nebe hroutí...\s

zdál se mi sen, já stojím na svém místě
zdál se mi sen, mé svědomí je čisté
zdál se mi sen, jen jedno vím jistě:
je  soudný den!


\song{poutník a dívka}{spirituál kvintet}

/A kráčel krajem poutník, šel sám
/D kráčel krajem poutník, šel /A sám
kráčel krajem poutník, kráčel /{Cis fis} sám
tu potkal /H dívku, nesla /H7 džbán, přistoupil k /E7 ní a pravil: \s

``ráchel, ráchel, žízeň mě zmáhá
ráchel, ráchel, žízeň mě zmáhá
ráchel, ráchel, žízeň mě zmáhá
tak přistup blíže, nehodná, a dej mi pít,'' a ona: \s

``kdo jsi, kdo jsi, že mi říkáš jménem
kdo jsi, kdo jsi, že mi říkáš jménem
kdo jsi, kdo jsi, že mi říkáš jménem
já tě vidím poprvé, odkud mě znáš?'' \s

``ráchel, ráchel, znám víc než jméno
ráchel, ráchel, znám víc než jméno
ráchel, ráchel, znám víc než jméno,''
pak se napil, ruku zdvih' a kráčel dál \s

ten džbán, ten džbán z nepálené hlíny
ten džbán, ten džbán z nepálené hlíny
ten džbán, ten džbán z nepálené hlíny
v onu chvíli zazářil kovem ryzím \s

kráčel krajem poutník, šel sám
kráčel krajem poutník, šel sám
kráčel krajem poutník, kráčel sám
ač byl /H chudý, nepoz/E nán, přece byl /{A D A} král



\song{darmoděj}{nohavica}

/a šel včera městem /e muž a šel po hlavní /{a e} třídě
/a šel včera městem /e muž a já ho z okna /{a e} viděl
/C na flétnu chorál /G hrál, znělo to jako /a zvon
a byl v tom všechen /e žal, ten krásný dlouhý /F tón
a já jsem náhle /Fisdim věděl: ano, to je /E7 on, to je /a on \S

vyběh' jsem do ulic jen v noční košili
v odpadcích z popelnic krysy se honily
a v teplých postelích lásky i nelásky
tiše se vrtěly rodinné obrázky
a já chtěl odpověď na svoje otázky, otázky

\R  /{a e C G a F Fisdim E7 a e C G a F Fisdim E7} na na na... **

dohnal jsem toho muže a chytl za kabát
měl kabát z hadí kůže, šel z něho divný chlad
a on se otočil, a oči plné vran
a jizvy u očí, celý byl pobodán
a já jsem náhle věděl, kdo je onen pán, onen pán \s

celý se strachem chvěl, když jsem tak k němu došel
a v ústech flétnu měl od hieronyma bosche
stál měsíc nad domy jak čírka ve vodě
jak moje svědomí, když zvrací v záchodě
a já jsem náhle věděl: to je darmoděj, můj darmoděj

\R  můj darmoděj, vagabund osudů a lásek
    jenž prochází všemi sny, ale dnům vyhýbá se
    můj darmoděj, krásné zlo, jed má pod jazykem
    když prodává po domech jehly se slovníkem **

šel včera městem muž, podomní obchodník
šel, ale nejde už, krev skápla na chodník
já jeho flétnu vzal a zněla jako zvon
a byl v tom všechen žal, ten krásný dlouhý tón
a já jsem náhle věděl: ano, já jsem on, já jsem on

\R  váš darmoděj, vagabund osudů a lásek
    jenž prochází všemi sny, ale dnům vyhýbá se
    váš darmoděj, krásné zlo, jed mám pod jazykem
    když prodávám po domech jehly se slovníkem **



\song{domov na zemi}{spirituál kvintet}

/D jak léta jdou, svět /G pro mě ztrácí /D glanc
všichni se rvou a /A7 duši dávaj' všanc
a /D za pár šestáků vás /G prodaj', věřte /D mi
už víc nechci mít /G domov /D svůj /A7 na ze/D mi! \S

\R  čas žádá svý a mně se krátí dech
    když před kaplí tu zpívám na schodech
    svou píseň vo nebi, kde bude blaze mi
    už víc nechci mít domov svůj na zemi! **

po jmění netoužím, jsme tu jen nakrátko
i sláva je jak dým, jak prázdný pozlátko
já koukám do voblak, až anděl kejvne mi
už víc nechci mít domov svůj na zemi!

\r čas žádá...

teď říkám ``good-bye'' světskýmu veselí
těm, co si užívaj', nechci lízt do zelí
jsem hříšná nádoba, však spása kyne mi
už víc nechci mít domov svůj na zemi!

\rr

v určenej čas kytara dohraje
zmlkne můj hlas na cestě do ráje
vo tomhle špacíru noc co noc zdá se mi
už víc nechci mít domov svůj na zemi!

\rr



\song{v 9 hodin 25}{samson lenk}

\R  /{a D F E a E a D F E a E} uap tadap ... **

v /a devět hodin dvacet pět mě /D opustilo štěstí
ten /F vlak, co jsem jím měl jet, na koleji /E dávno /E7 nestál
v /a devět hodin dvacet pět /D jako bych dostal pěstí
já /F za hodinu na náměstí měl jsem /E stát, ale v /E7 jiným městě \S

tvá /A7 zpráva zněla prostě a byla tak krátká
že /d stavíš se jen na skok, že nechalas' mi vrátka
/G zadní otevřená, /E zadní otevřená
já /A7 naposled tě viděl, když ti bylo dvacet
to /d jsi tenkrát řekla, že se nechceš vracet
/G že jsi unavená, /E ze mě unavená

\r uap tadap...

já čekala jsem, hlavu jako střep, a zdálo se, že dlouho
může za to vinný sklep, že člověk často sleví
já čekala jsem, hlavu jako střep, s podvědomou touhou
já čekala jsem dobu dlouhou, víc než dost, kolik přesně, nevím \s

pak jedenáctá bila a už to bylo passé
já dřív jsem měla vědět, že vidět chci tě zase
láska nerezaví, láska nerezaví
ten list, co jsem ti psala, byl dozajista hloupý
byl odměřený moc, na vlídný slovo skoupý
už to nenapravím, už to nenapravím

\rr



\song{mám jednu ruku dlouhou}{buty}

/E najdem si /cis místo, /gis kde se dobře /fis kouří
/E kde horké /cis slunce d/gis o nápojů /fis nepíchá
/H kde vítr /H7 snáší /E žmolky ptačích /A hovínek
/H okolo /A nás a /H7 říká \S

můžeme zkoušet co nám nejlíp zachutná
a klidně se dívat jestli někdo nejde
někdo kdo ví, že už tady sedíme
a řekne nazdar kluci

\R  \[ /E mám /cis jednu ruku /A dlou/E hou \] **

/A posaď se k /fis nám, /cis necháme tě /h vymluvit
/A a vzpome/fis nout si /cis na ty naše /h úkoly
/E tu ruku nám /h dej a /A odpočívej v /D pokoj/E i
/A tam na tom /fis místě, /cis kde se dobře /h nen/E7 í \S

\R  /A na /fis na na na na /D ná /A ná \krat6
    /a na na na na na /D ná /A ná
    /A mám /fis jednu ruku /D dlou/A hou \krat8 **



\song{amerika}{lucie}

/G nandej mi /D do hlavy tvý /a brouky
a bůh nám seber /G beznaděj
v duši zbylo /D světlo z jedný /a holky
tak mi teď za to /G vynadej \S

zima a /D promarněný /a touhy
do vrásek stromům padá /G déšť
zbejvaj roky, /D asi ne moc /a dlouhý
a do vlasů mi zabrou/C kej:

\R  pá /G pa pá /e pá
    pá, pá, pá, /G pá; pá, pá, pá, /e pá \krat 2 **

tvoje voči jenom řvavý tóny
dotek slunce zapadá
horkej vítr rozezní mý zvony
do vlasů ti zabroukám:

\r  pá pa pá pá...

na obloze křídla tažnejch ptáků
tak už na svý bráchy zavolej
na tváře ti padaj slzy z hrachů
a bůh nám sebral beznaděj \s

v duši zbylo světlo z jedný holky
do vrásek stromům padá déšť
poslední dny, hodiny a roky
do vlasů ti zabrouká:

\rr



\song{sametová}{žlutý pes}

/G vzpomínám, když tehdá /C před léty /G začaly lítat /C rakety
/G zdál se to bejt /C docela dobrej /D nápad
/G saxofony hrály /C unyle, /G frčely švédský /C košile
/G a někdo se moh' /e docela dobře flák/{C D} at \S

když tam stál rohatej u školy a my neměli podepsaný úkoly
už tenkrát rozhazoval svoje sítě
poučen z předchozích nezdarů sestrojil elektrickou kytaru
a rock'n'roll byl zrovna narozený dítě

\R  /G vzpomínáš, takys' tu /D žila a /e nedělej, že jsi /C jiná
    taková /G malá pilná /D včela, taková /C celá /D sameto/G vá **

přišel čas a jako náhoda byla tu bigbítová pohoda
kytičky a úsmevy sekretárok
sousedovic bejby milena je celá blbá z boba dylana
ale to nevadí, já mám taky nárok \s

starý, mladý nebo pitomí, mlátili do toho jako my
hlavu plnou londýna nad temží
a starej dobrej satanáš hraje u nás v hospodě mariáš
a pazoura se mu trumfama jenom hemží

\R  vzpomínáš, už je to jinak, a jde z toho na mě zima
    ty jsi, holka, tehdá byla taková celá sametová **

a do toho tenhle gorbačov, co ho znal celej dlabačov
kopyta měl jako z arizony
přišel a zase odešel a nikdo se kvůli tomu nevěšel
a po něm tu zbyly samý volný zóny

\R  \[ vzpomínáš, jak jsi se měla, když jsi nic nevěděla,
    byla to taková krásná cela a byla celá (sametová) ... \] **



\song{milenci v texaskách}{starci na chmelu}

/C chodili spolu z čisté /a lásky /F a sedmnáct jim bylo /{a G} let
/C a do té lásky bez nad/a sázky /F se vešel celý širý /{C B C} svět
/F ten svět v nich ale viděl /e pásky, /d a jak by mohl /{D7 G F Fis G} nevidět
/C vždyť horovali pro te/a xasky /F a sedmnáct jim bylo /{a G} let \S

a v jedné zvláště slabé chvíli za noci silných úkladů
ti dva se spolu oženili bez požehnání úřadů
ať vám to je či není milé, měla ho ráda, měl ji rád
odpusťte dívce provinilé, jestli vám o to bude stát \s

ať vám to je či není milé, měla ho ráda, měl ji rád
a bylo by moc pošetilé pro život hledat jízdní řád
tak jeden mladík s jednou slečnou se spolu octli na trati
kéž dojedou až na konečnou, kéž na trati se neztratí...



\Song{mašinka}{semtex}

\R  /C jede jede mašinka kouří /F se jí z komín/C ka
    \[ jede /F jede do dá/G li, veze /F samý ožra/C lý \] **

neusínej nechoď spát, neusínej nechoď spát
\[ až do rána bílýho my budem zpívat a budem hrát \] \s

lásko má jsi jediná, do třinácti nevinná
\[ vyhrnu si rukávy, praštím s tebou do trávy \]

\r  jede jede...

přijde ke mně průvodčí, kleštičkama zatočí
\[ lístky prosím, nemám prosím, jak to prosím, to máte tak \]

\rr

jel jsem jednou tramvají a pod sedačkou potají
\[ přistoupila starší dáma přisedla mi dařbujána \]

\rr



\song{dej mi víc své lásky}{olympic}

/a vymyslel jsem spoustu napadů, a/C ů
co /a podporujou dobrou nála/G du, a/E ů
/a hodit klíče do kanálu, /D sjet po zadku /d holou skálu
/a v noci chodit /E strašit do hra/a du, a/(G) ů \S

dám si dvoje housle pod bradu, aů
v bílé plachtě chodím pozadu, aů
úplně melancholicky, s citem pro věc jako vždycky
vyrábím tu hradní záhadu, aů

\R  /C má drahá, dej mi víc, /E má drahá, dej mi víc
    /a má drahá, /F dej mi víc své /C lásky, a/G ů
    /C já nechci skoro nic, /E já nechci skoro nic
    /a já chci jen /F pohladit tvé /C vlásky, a/E ů **

nejlepší z těch divnejch nápadů, aů
mi dokonale zvednul náladu, aů
natrhám ti sedmikrásky, tebe celou s tvými vlásky
zamknu si na sedm západů, aů

\r  má drahá...

vymyslel jsem...



\song{rána v trávě}{žalman}


\R  /a každý ráno boty /G zouval,  /a orosil si nohy v /G trávě
    /a že se lidi mají rádi, /G doufal, /a a pro/e citli /a právě
    /a každý ráno dlouze /G zíval, /a utřel čelo do /G rukávu
    /a a při chůzi tělem sem-tam /G kýval, /a před se/e bou sta /a sáhů **

/C poznal /G mora/F věnku /C krásnou
/a a ví/G nečko /C ze zlata
v čechách /G slávu /F muzi/C kantů 
/a uma/e zanou /a od bláta

\r  každý ráno...

toužil najít studánečku
a do ní se podívat
by mu řekla: proč, holečku
musíš světem chodívat \s

studánečka promluvila: 
to ses' musel nachodit
abych já ti pravdu řekla
měl ses' jindy narodit

\rr



\song{nezacházej, slunce}{žalman}

/G nezacházej, slunce, neza/a cházej ještě
/D já mám potěšení /C na da/D lekej cestě
/G já mám potě/{h e} šení    /a na da/D lekej ces/G tě \S

já má potěšení mezi hory-doly
\[ žádnej neuvěří, co je mezi námi \] \s

mezi náma dvouma láska nejstálejší
\[ a ta musí trvat do smrti nejdelší \] \s

trvej, lásko, trvej, nepřestávaj trvat
\[ až budou skřivánci o půlnoci zpívat \] \s

skřivánci zpívali, můj milej nepřišel
\[ on se na mě hněvá, nebo za jinou šel \]



\Song{hvězda na vrbě}{karel mareš -- jiří štaidl}

kdo se /a večer /e hájem /a vrací, /F ten ať /e klop/G7 í zra/{C e} ky
ať /G je /a nikd/e y neo/a brac/d í k vr/G7 bě, /e křivola/{E A C e} ký
jin/G ak /a jeho /e oči /a zjistí, /F i když /e se to /G7 nezd/{C e} á
že /G na /a vrbě, /e kromě /a listí, /d visí /F malá hvěz/a da \S

viděli /C jsme jednou v /F lukách, plakat /C na tý vrbě /F kluka
který /d pevně věřil /B tomu, že ji /D7 sundá z toho /{G7 e} stromu \S

kdo o hvězdy jeví zájem, zem, když večer hladne
ať jde klidně někdy hájem, hvězda někde spadne
ať se pro ni rosou brodí a pak vrbu najde si
a pro ty, co kolem chodí, na tu větev zavěsí



\song{všech vandráků múza}{žalman}

přišla k /G nám znenadání, /Fmaj7 hubená až /G hrůza
a /Fmaj7 řekla, že je /G múza /Fmaj7 všech vandráků z /G čech
to nebyl hřích po nocích, než /Fmaj7 po kytaře /G sáhla
/Fmaj7 tak nám chleba /G kradla a /Fmaj7 čmárala po /G zdech \S

\R  /e ha/D lel/G uja, /C zavírá se /G brána
    my /C zpívat chcem do /G rána
    /C než napadá /G sníh
    /e ha/D lel/G uja, /C kteroupak si /G dáme
    /C než skončíme s /G ránem
    na hřbi/C tově obu/G tých **

na moravě z demižónu dobré víno pila
pak mezi námi žila spoustu hezkejch dní
až jeden pán v limuzíně začal po ní toužit
budeš mi, holka, sloužit, a oženil se s ní

\r haleluja...

každou noc po milování skládala mu hity
a kašlala na city, na obyčejnej lid
za pár dní tahle můza, dneska služka mici
skončila na ulici a pod mostem má byt

\rr

vraťte /G nám, vraťte nám tu /Fmaj7 všech vandráků /G múzu
má /Fmaj7 roztrhanou /G blůzu, /Fmaj7 hm hm /G hm...



\song{já, písnička}{pavlína jíšová -- žalman}

to /d já, zrozená z nemo/C cných básníků
z tichých /a vět po nocích utka/d ná
to já, zrozená, nen/C í mi do smíchu
když mám /a být pod cenou proda/d ná

\R  kdo /F vás obléká do šat/C ů svatebních
    holky m/d ý ztracený, ztracen/a ý
    ženich sp/d í na růži, až něk/C am pod kůži
    se můj str/a ach do klína schová/d vá **

to já, zrozená v hospodách na tácku
raněných ospalých milenců
to já, zrozená, vyrytá na placku
směju se po stěnách cizincům

\r kdo vás obléká...

to já, písnička, ležím tu před vámi
jenom tak, spoutaná tolikrát
to já, písnička, provdaná za pány
kteří pro lásku smí o mně hrát

\rr



\song{zachraňte koně}{kamelot}

/e peklo byl ráj, když hořela stáj, /a7 příteli
/C věř mi, koně pl/D áčou, poví/{G C H7} dám
/e to byla půlnoc, v tom křik o pomoc, už /a7 letěly
/C hejna kohout/H7 ů, a bůhví /e kam

\R  /G zachraňte koně, kř/h ičel jsem tisíc/C krát
    /G žil jsem jen pro ně, br/h ánil je nejvíc/C krát
    než přišla chv/a íle, kdy hřívy /C bílé
    pročesal pl/a amen, spálil na /{H7 D7} troud **

ohrady a stáj, a v plamenech kraj už nedýchal
já viděl, jak to hříbě umírá
klisna u něj a smuteční děj se odbývá
jak tiše pláče, oči přivírá

\r zachraňte koně...

\R  /G zachraňte koně, k/h řičel jsem tisíc/C krát
    /G žil jsem jen pro ně, br/h ánil je nejvíc/C krát
    /G zachraňte koně, kř/h ičel jsem tisíc/C krát
    zachraňte /e koně... **



\song{jarošovský pivovar}{argema}

/C léta tam /G stál, /F stojí tam /C dál
pivovar /G u cesty, /F kdekdo ho /G znal
/C léta tam /G stál, /F stát bude /C dál
ten kdo zná /G jarošov, /F zná pivo/C var

\R  /F bílá /G pěna, lá/C hev oro/a sená
    /F chmelový /G nektar já zná/C m
    /F jen jsem to /G zkusil a /C jednou se /a napil
    /F od těch dob /G žízeň má/C m **

bída a hlad, kolem šel strach
když bylo piva dost, mohl ses smát
tři sta let stál, stát bude dál
ten kdo zná jarošov, zná pivovar

\r  bílá pěna...

ná na na ná...

\rr

\rr



\song{hudsonské šífy}{wabi daněk}

ten, kdo /a nezná hukot vody lopat/C kama vířený
jako /G já, jako /C já
kdo hud/a sonský slapy nezná sírou /G pekla sířený
ať se /a na hudsonský /G šífy najmout /a dá --- /G joho/a ho \S

ten, kdo nepřekládal uhlí, šíf, když na mělčinu vjel
málo zná, málo zná
ten, kdo neměl tělo ztuhlý, až se nočním chladem chvěl
ať se na hudsonský šífy najmout dá --- johoho

\R  a/F hoj, páru tam /a hoď
    ať /G do pekla se dříve dohra/a bem
    /G joho/a ho, /G joho/a ho **

ten, kdo nezná noční zpěvy zarostených lodníků
jako já, jako já
ten, kdo cejtí se bejt chlapem, umí dělat rotyku
ať se na hudsonský šífy najmout dá --- johoho \s

ten, kdo má na bradě mlíko, kdo se rumem neopil
málo zná, málo zná
kdo necejtil hrůzu z vody, kdo se málem utopil
ať se na hudsonský šífy najmout dá --- johoho

\r ahoj, páru...

kdo má roztrhaný boty, kdo má pořád jenom hlad
jako já, jako já
kdo chce celý noci čuchat pekelnýho kouře smrad
ať se na hudsonský šífy najmout dá --- johoho \s

kdo chce zhebnout třeba zejtra, komu je to všecho fuk
kdo je sám, jako já
kdo má srdce v správným místě, kdo je prostě príma kluk
ať se na hudsonský šífy najmout dá --- johoho

\rr



\song{máma má rýmu}{těžkej pokondr}

/G venku je zima veliká
/h a kdekdo do práce utík/b á
/a já nemusím nikam vstávat
/D já budu mámě /C první pomoc /D dávat \S

mamina byla včera pařit s náma
potkala na baru jednoho pána
jen co si stačila panáka loknout
to zvíře na ni otevřelo vokno

\R  /{G e G D C D...} mamama máma má rýmu má... **

za chvíli si mě pošle do lékárny
šumivý aspirín by nebyl marný
do skříně hodím její sáčko
a v práci nahlásím výjimečně áčko \s

otevřu pivo a ona řekne díky
a potom budu žehlit kapesníky
musím jí dát aspoň malé množství lásky
aby zas neměla ze života vrásky \s

ta žena není žádná kuriozita
takhle to prostě moje máma má
jak někde zůstane o trochu déle
tak musí ráno rychle do postele

\r mamama...



\song{na kolena}{ivan hlas --- šakalí léta}

táhněte /G do háje a všichni /e pryč
chtěl jsem jít /G do ráje a nemám /e klíč
jak si tu /G můžete takhle /e žrát
ztratil jsem /C holku, co ji mám /D rád \S

napravo, nalevo nebudu mít klid
dala mi najevo, že mě nechce mít
zbitej a špinavej tancuju sám
váš pohled káravej už dobře znám

\R  pořád jen /C na kolena, na kolena
    na kolena, na kolena /G jé, jé, jé 
    pořád jen /C na kolena, na kolena
    na kolena, na kolena /G jé, jé, jé
    pořád jen /C na kolena, na kolena
    na kolena, na kolena /G je to /e tak
    a vaše /C saka vám posere /D pták **

cigáro do koutku si klidně dám
tuhle tu pochoutku vychutnám sám
kašlu vám na bonton, výmysly chytrejch hlav
sere mě tichej don a ten váš tupej dav

\R  pořád jen
    \ \vdots
    a tenhle barák vám posere pták **



\song{tata}{buty}

/g musíme zajet na chatu
/d musím se podívat na tatu
/c chtěl bych se za ním podívat
/g tralalalálálá ... \S

/{g d c g g d c g} na, na... \S

/B tata, tata, /d tata tata tata /c tata 
/B tata, tata, /d tata tata tata /c tata \S

dneska ho stáhnem na pivo
jenom se rýpe do hlíny
má tam ty svoje okurky
tralalalálálá ... \s

nasadí staré tepláky
všecko co umím je od taty
musíme zajet na chatu
za mojim tatu \s

musíme zajet na chatu
musím se podívat na tatu
chtěl bych se za ním podívat
tralalalálálálalala ... \s

nasadí staré tepláky
všecko co umím je od taty
musíme zajet na chatu
za mojim tatu \s

jo ..tata, tata,tata, tata........



\song{voda se neutopí}{bratři ebenové}

/d napsal jsem ti písničku, /G abys už /C neplaka/{C\^H F G} la
/d napsal jsem ti písničku, /G abys by/C la vese/{C\^H F G} lá
/F kdy/d7 by  /C jsi /G se /d7 pousm/C ál/G a, /B to /F by /C jsi /G mě /d7 potěši/G la

\R  vždyť víš, že /C voda se /D neuto/G pí, /a plamínek /Fj7 nespá/G lí
    /C lávky jdou /D přes příko/G py, /a proč bychom /Fj7 se  bá/G li
    /a vítr se /e nezadu/g sí, dálka se /F nevzdálí
    jsem vždycky /F6 tam, kde ty /C jsi
    \[ proč bychom /Fj7 se bá/C li? \] **

napsal jsem ti písničku, aby sis byla jistá
napsal jsem ti písničku, že jsi moje jediná
kdyby jsi se pousmála, to by jsi mě potěšila

\r vždyť víš...

\R  voda se neutopí, plamínek nespálí
    \ \vdots
    \[ proč bychom se báli, \] \krat3 **



\song{za malou chvíli}{bratři ebenové}

pojďte si /{G C G} povídat, beze slov po/{C G} vídat
tak, aby nikdo /F kromě /C\^E nás /a7 neměl /F zdá/G ní
že si lze po/C víd/G at, beze slov po/C víd/G at
že jsou to rozho/F vory /C\^E bez /a7 poví/F dá/G ní

\R  jen tak mi v /Es myšlenkách /F pošli pár /G vět
    /Es za malou chvíli /F máš tu odpo/G věď
    pojďte si po/C víd/G at, beze slov po/C víd/G at
    jen přivřít oči /F a nic /C\^E víc, /a7 co vám /F brá/G ní? **

pojďte si posílat, dopisy posílat
dopisy bez obálek, bez adresátů
pojďte si posílat, dopisy posílat
kolik je lidí, tolik je cizích států

\R  jen tak mi v myšlenkách pošli pár vět
    za malou chvíli máš tu odpověď
    pojďte si posílat, dopisy posílat
    kolik je lidí, tolik je cizích států **

\R  jen tak mi v myšlenkách pošli pár vět
    za malou chvíli máš tu odpověď
    pojďte si posílat, dopisy posílat
    adresu není třeba psát --- ty tam, já tu **



\song{keramická}{bratři ebenové}

/d když je smutno /a malíři, /g pozve si /a model/d ky
maluje jim /a profily, /g anfasy /a i cel/d ky
/B děvčata /F jsou vese/B lá s překrás/F nými tě/C ly
/d malíři však /a pod štětcem /g pláčou ak/a vare/d ly \S

když je smutno sochaři, tak si vezme dláto
formu z dřeva udělá, kdepak, nezvorá to
pak si z kovu odlije ňákou pěknou sošku
jenomže je studená, teplá ani trošku \s

když je smutný keramik, to je jiná píseň
ten si v peci zatopí a už netrápí se
nejdřív spálí dopisy od svý první lásky \s

potom skripta z umprumu, jsou jich čtyři svazky
potom přijdou na řadu dva oddací listy
pak monografie polského symbolisty
z komína valí se kysličník uhličitý
když hoří akryly, lepty a linoryty
mezitím, co pálí veškeré svoje šaty
opravuji: jde o kysličník uhelnatý \s
  
v peci oheň plápolá, umělec usíná
když je smutný keramik, je to vždycky hlína



\song{jak to dělaj kosmonauti}{bratři ebenové}

/E dneska jsem s/A e u sní/H daně /A zase dusi/{E A H A} l
/E bylo ňák/A ý suchý /H těsto - j/A á vám zkusi/{E A H} l
někdo kl/cis eje, a tím h/Fis řeší proti b/E\^{Gis} ohu
já však hl/A edím /A\^H na obl/E oh/{A H A} u \S

\R  jak to /E dělaj/A ', /H no /A řekněte mi
    /E jak to děl/A aj' kosmon/H auti/Cis ? **

/Fis všude samý monitory, /E budíky a stopky
/H a mezi tím poletujou v /E kabině ty /Fis drobky
ať mi nikdo neříká, že /E to tam takhle chodí
/H to by měli pěknej bordel v t/D ý kosmický lodi

\R  (ale) jak to /E dělaj/A ', /H no /A řekněte mi
    /E jak to děl/A aj' kosmon/H auti
    /h vím, co dělaj' /A pionýři, /D vím, co dělaj' sk/E auti
    ale /fis jak to dělaj /E kosmonau/H ti? **

tuhle mi zas moje žena hnula žlučí:
mačkat pastu od konce se nenaučí
tak jsem aspoň na chviličku vypad' z bytu
a hned jsem byl na orbitu

\r jak to dělaj...

co když ti daj' do posádky ňákou tetu
čumíš na ni vod obletu do obletu
navíc: praštit dveřma, to tam taky nejde
to bys musel do skafandru, a to tě to přejde

\r jak to dělaj...
\r vím co dělaj...

\ \songgg

existuje jedna místnost v každým bytě
kde je člověk v absolutní intimitě
spousta lidí, pokud já vím, tam čte knížky
já však hledím zas do výšky

\r jak to dělaj...

jestli mají v kombinéze ňákej pytlík
nebo prostě něco, kam by to hned chytli
v časopisech odborných se marně hrabu
jako kdyby tyhle věci byly tabu

\r jak to dělaj...
\r vím co dělaj...



\Song{čert}{bratři ebenové}

/F když má čert /C smutek, /d když má čert /{B C} splín
/C\^E když, na co /F sáh/C\^E ne, /d vždycky /F trefí /{G\^H G} vedle
/C když má čert /d smut/G\^H ek, /G když má čert /{F\^A C} splín
/C\^E tak /F nechce /C\^E se mu /{d F} ani   /G\^H bleble, /G ani /F\^A blebleble
ani /B blebleble/F\^A ble   /g7 udělat, /B\^C když má /{F B\^F} splín \S

když má čert smutek, když má čert splín
a když ho ještě navíc hrozně bolí škeble
když má čert smutek, když má čert splín
tak nechce se mu ani bleble, ani blebleble
ani bleblebleble udělat, když má splín \s

když má čert smutek, když má čert splín
a když mu nefungujou basy ani treble
když má čert smutek, když má čert splín
tak nechce se mu ani bleble, ani blebleble
ani bleblebleble udělat, když má splín


\song{prkenná hymna}

/a ((g\^{H(m)t} y\^{rt}) /G mod p) /a mod q = 
/a ((g\^{H(m)t} [g\^x]\^{rt}) /G mod p) /a mod q

\R protože /C y = g\^x /G mod /a p **

((g\^{H(m)t} y\^{rt}) mod p) mod q = 
(g\^{[H(m) + xr]t} mod p) mod q

\R protože s = k\^{-1}(H(m) + xr) mod q **

((g\^{H(m)t} y\^{rt}) mod p) mod q = 
(g\^{[ks]t} mod p) mod q

\R protože t = s\^{-1} mod q **

((g\^{H(m)t} y\^{rt}) mod p) mod q = 
(g\^k mod p) mod q

\R protože r = (g\^k mod p) mod q **

((g\^{H(m)t} y\^{rt}) mod p) mod q = r



\song{báječnej bál}

/A bá/D ječnej /{A D A} bál \quad us/D pořá/{A D A} dal
králů /{cis fis} král všechny pozval /E7 dál
ten největší /{A D A} sál \quad na to /D obsta/{A D A} ral
neboť se /fis bál, aby všem moh /H7 říct
pojďte /A dál začíná /cis bál
/E7 bálů /{A D A D A D} bál \S

ó pane můj, proč právě dnes
v plánu mám zrovna jinej ples
ze všech pozvaných mluvil náhle strach
že se připraví zas o jeden tah
když ale ten bál byl mimo plán
mimo plán \s

král však ten bál neodvolal
zvát šel dál než by kdo čekal
ani žádná mříž ba ani ten hřích
nezabrání tomu kdo vejít
chtěl by na ten bál báječnej bál
bálů bál


\song{něco za něco}

/e učíme se /h7 vidět velkej /C vůz
/e najít ještě /h7 účinnější /C lék
/e nečekat že /h7 ze světových /C stran
/D dostaví se /{e h7 C} vděk \S

přátelé už stárnou jako já
pozlacený z překroucených pravd
nechci dělat jen to co se má
nechci pořád lhát

\R  něco /G za něco a /D příště /C nic
    srdce /G naruby a /D na rtech /C líc
    něco /G za něco než /D přijde /a7 klam
    /h7 ze světových /{e h7 a7 e h7 C} stran **

někdo ze své dýmka bafá dýl
jiný ani nezapálí troud
nevidím, že by tu někdo zbyl
není se kam hnout

\r  něco za něco...

naučit se, jaký letí pták
nechat si zdát obyčejnej sen
pozdravit se s někým jenom tak
že je hezkej den

\rr

učíme se vidět...


\song{v širém poli}

/C v širém poli /G hruška stojí /F a v zahrádce /C dvě
/C pěstovala /G mamka cérku /F ale ne /C sobě
/G ej žol, žol, /C žol mi buďe /G až mi ju /C vezmú luďe
/C potěšení /D mo/G je \S

ona ide do kostela jako leluja
všeci luďe sa dívajú aká je šumná
ej žol, žol, žol mi buďe až mi ju vezmú luďe
potěšení moje


\song{kluziště}{plíhal}

strejček kovář chytil kleště, uštíp' z noční oblohy
jednu malou kapku deště, ta mu spadla pod nohy,
nejdřív ale chytil slinu, tak šáh' kamsi pro pivo,
pak přitáhl kovadlinu a obrovský kladivo.

zatím tři bílé vrány pěkně za sebou
kolem jdou, někam jdou, do rytmu se kývají,
tyhle tři bílé vrány pěkně za sebou
kolem jdou, někam jdou, nedojdou, nedojdou.

vydal z hrdla mocný pokřik ztichlým letním večerem,
pak tu kapku všude rozstřík' jedním mocným úderem,
celej svět byl náhle v kapce a vysoko nad námi
na obrovské mucholapce visí nebe s hvězdami.

zpod víček mi vytrysk' pramen na zmačkané polštáře,
kdosi mě vzal kolem ramen a políbil na tváře,
kdesi v dálce rozmazaně strejda kovář odchází,
do kalhot si čistí ruce umazané od sazí.



\konec{obsah}{abecedně}
\bye

dopsat:
-------
neco za neco
boure
chybi konec montgomery
bridge over troubled water
zitra rano v pet
dva havrani
tri bile vrany
pisek
hej clovece bozi
bratricku
bon soir
ezop
jasna zprava
kluziste (3 bile vrany)
kometa
