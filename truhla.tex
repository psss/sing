\input cantar.sty
\zacatek{zpěvník}{tajemství staré truhly}



\song{omnia vincit amor}{klíč}

F C d a B C d
/d šel pocestný kol /C hospodských /d zdí
/F přisedl k nám a /C lokálem /F zní
/g pozdrav jak svaté /F přikázá/C ní
/d omnia /C vincit /d amor \s

hej, šenkýři, dej plný džbán
ať chasa ví, kdo k stolu je zván
se mnou ať zpívá, kdo za své vzal
omnia vincit amor \s

\R zlaťák /F pálí, /C nesleví /d nic
   štěstí v /F lásce /C znamená /F víc
   všechny /g pány /F ať vezme /{C A} ďas
   /d omnia /C vincit /d amor **

já viděl zemi válkou se chvět
musel se bít a nenávidět
v plamenech pálit prosby a pláč
omnia vincit amor \s

zlý trubky troubí, vítězí zášť
nad lidskou láskou roztáhli plášť
v tom kdosi krví napsal ten vzkaz
omnia vincit amor \s

\r zlaťák...

já prošel každou z nejdelších cest
všude se ptal, co značí ta zvěst
až řekl moudrý, pochopíš sám
omnia vincit amor (všechno přemáhá láska) \s

\rr

teď s novou vírou obcházím svět
má hlava zšedla pod tíhou let
každého zdravím větou všech vět
omnia vincit amor



\song{alison gross}{asonance}

když /d zapadlo /C slunce a /B vkradla se /A noc
a v /d šedivých /C mracích se /B ztrá/g cel /A den
a /d když síly /C zla ve tmě /F převzaly /A moc
tu /d alison /C gross vyšla z hradu /d ven \S

tiše se vplížila na můj dvůr
a jak oknem mým na mě pohlédla
tak jen kývla prstem a já musel jít
a do komnat svých si mě odvedla

\R  /d alison gross a černý /C hrad
    ze /d zlověstných /C skal jeho hradby /d ční
    alison gross, nejodporněj/C ší
    ze všech /F čaroděj/C nic v zemi sever/{G A} ní **

složila mou hlavu na svůj klín
a sladkého vína mi dala pít
já můžu ti slávu i bohatství dát
jen kdybys mě chtěl za milenku mít \s

mlč a zmiz babo odporná!
slepý jak krtek bych musel být!
to radši bych na špalek hlavu chtěl dát
než alison gross za milenku mít!

\r  alison gross...

přinesla plášť celý z hedvábí
zlatem a stříbrem se celý skvěl
kdybys jen chtěl se mým milencem stát
tak dostal bys vše, co bys jenom chtěl \s

pak přinesla nádherný zlatý džbán
bílými perlami zářící
kdybys jen chtěl se mým milencem stát
těch darů bys měl plnou truhlici \songgg

stůj a mlč babo odporná!
slepý jak krtek bych musel být!
to radši bych na špalek hlavu chtěl dát
než milencem tvým na chvíli se stát! \s

\rr

tu k ohyzdným rtům zvedla černý roh
a natřikrát na ten roh troubila
a s každým tím tónem mně ubylo sil
až všechnu mou sílu mi sebrala \s

pak alison gross vzala čarovnou hůl
a nad mojí hlavou s ní kroužila
a podivná slova si zamumlala
a v slizkého hada mě zaklela

\rr

tak uplynul rok a uplynul den
a předvečer svátku všech svatých byl
a tehdy na místě, kde žil jsem jak had
se zjevila královna lesních víl \s

dotkla se mě třikrát rukou svou
a její hlas kletbu rozrazil
a tak mi zas vrátila podobu mou
že už jsem se dál v prachu neplazil

\rr



\song{barbora píše z tábora}{jaroslav uhlíř, zdeněk svěrák}

/D maminko, tatínku, posílám vám /A vzpomínku
/{ } z letního tábora, jistě víte, že vám píše
vaše dcera barbo/D ra \s

strava se nedá jíst, dneska byl jen zelný list
polívka studená, co v ní plavou místo nudlí
číslice a písmena \s

myslela jsem prostě, že budou různé soutěže
slíbili bojovku, pak jsme hráli vybíjendu
na ovce a na schovku \s

štefan, hlavní vedoucí, chodí s naší vedoucí
která je příšera, scházejí se, líbají se
u totemu za šera

\R  proč jsem se /G nenarodila /A o pár let /{D h} dřív
    dneska bych /e krásně chodila, /A7 se štefanem, co /D říká si stev/D7 e
    proč jste mě /G prostě neměli /A o pár let /{D h} dříve
    řekla bych: /e nebuď nesmělý, /A7 líbej mě /D steeve **

závěrem dopisu ještě trochu popisu
ta bréca vedoucí, je tlustá jak dvě normální
oddílové vedoucí \s

když běží po lese, všechno na ní třese se
užívá make-upu, co na ní ten štefan vidí
to já prostě nechápu

\r proč jsem se nenarodila...

p.s. pošlete dvě tři sta, jsem bez peněz dočista
pojedem do písku, máme v plánu zastavit se
v jitexovém středisku \s

když jsme se koupali, všichni na mě koukali
je to tím, že možná, v jednodílných plavkách už jsem
už jsem prostě nemožná

\R  proč jsem se /G nenarodila, /A nenarodila /D dřív **



\song{bouře}{nezmaři}

/d slyším bití vln a příboje
moře /d\^C zuří, bouře blízko je
maják /B\^{maj7} bílým světlem varuje
osa/C mělé námořníky \s

vítr proniká až pod kosti
jeho chlad se v duši rozhostí
tam, kde není žádné milosti
jenom rozbouřené moře

\R /d já mám /B přečkat bouře /a čas, ano pane
   /d já mám /B přečkat bouře /a čas, přečkat bouře /d čas **

odpusť, pane, o čem sním
proč vždy, když ti to předložím
já sebe znovu uvidím
jak pluju mořem sama \s

teď tvou ruku blízko mám
tvou velkou lásku poznávám
jen nevím, proč se obávám
plout bouří, když jsi se mnou

\r já mám přečkat...

až vzdálenost nás rozdělí
a pohltí mě čas
věř, nakonec se uvidí
že i ten mě nese k tobě \s

zář slunce nebe vyjasní
to světlo můj strach rozpustí
zas můžu plakat radostí
láskou, kterou dals mi

\R já mám přečkat bouře čas, ano pane
   já mám přečkat bouře čas, ano pane
   já chci v bouři vytrvat, s tebou pane
   já chci v bouři vytrvat, s tebou vytrvat **



\song{černá díra}{karel plíhal}

/D mívali jsme /A dědečka, /G starého už /D pána
stalo se to v /A červenci /G jednou časně /A zrá/D na
/h šel do sklepa /G pro vidle, /E aby seno /A sklízel
/D už se ale /A nevrátil, /G prostě někam /A zmi/D zel \S

máme doma ve sklepě malou černou díru
na co přijde, sežere, v ničem nezná míru
nechoď, babi, pro uhlí, sežere i tebe
už tě nikdy nenajdou příslušníci vb \s

přišli vědci zdaleka, přišli vědci zblízka
babička je nervózní a nás, děti, tříská
sama musí poklízet, běhat kolem plotny
a děda je ve sklepě nekonečně hmotný \s

hele, babi, nezoufej, moje žena vaří
a jídlo se jí většinou nikdy nepodaří
půjdu díru nakrmit zbytky od oběda
díra všechno vyvrhne, i našeho děda \s

tak jsem díru nakrmil zbytky od oběda
díra všechno vyvrhla, i našeho děda
potom jsem ji rozkrájel motorovou pilou
opět člověk zvítězil nad neznámou silou \s

/E dědeček se /H raduje, /A že je zase v /E penzi
teď je naše /H písnička /A zralá pro re/H cen/E zi \songgg



\song{dva roky prázdnin}{karel černoch}

/a vzhůru na palubu, /D dálky /a volají
/a vítr už příhodný /G vane /a nám
/a tajemné příběhy /D nás teď /a čekají
/a tvým domovem bude /G oce/a án

\R  /F v lanoví plachty /C vítr nadouvá
    /F žene loď v širou /C dál
    /d kolébá boky /a plachetnice
    /d jak by si s ní jenom /E hrál
    /F posádku ani /C škuner neleká
    /F bouře ni ura/C gán
    /d přítomnost země /a oznámí nám
    /d příletem /E kormo/a rán **

náš ostrov vzdálený, z vln se vynoří
z příboje snů našich, pustý kraj
zátoku písčitou úsvit odhalí
háj palem, útesy bílých skal

\R  příď krájí vlny i tvůj čas
    srdce tvé tluče rázně
    nástrahy moře, nebezpečí
    s přáteli zvládneš vždy snáz
    v přátelství najdeš pevnou hráz
    zbaví tě smutku, bázně
    zítra až naše cesta skončí
    staneš se jedním z nás **



\song{grónská písnička}{jaromír nohavica}

/D daleko /e na severu /A je grónská /D zem
žije tam /e eskymačka s /A eskymá/D kem
\[ /D my bychom /e umrzli jim /G není zi/D ma
snídají /e nanuky /A a eskym/D a \]\S

mají se bezvadně vyspí se moc
půl roku trvá tam polární noc
\[ na jaře vzbudí se a vyběhnou ven
půl roku trvá tam polární den \]\s

když sněhu napadne nad kotníky
hrávají s medvědy na četníky
\[ medvědi těžko jsou k poražení
neboť medvědy ve sněhu vidět není \]\s

pokaždé ve středu přesně ve dvě
zaklepe na íglů hlavní medvěd
\[ dobrý den mohu dál na vteřinu
já nesu vám trochu ryb na svačinu \]\s

v kotlíku bublá čaj, kamna hřejí
psi venku hlídají před zloději
\[ smíchem se otřásá celé íglů
medvěd jim předvádí spoustu fíglů \]\s

tak žijou vesele na severu
srandu si dělají z teploměrů
\[ my bychom umrzli jim není zima
neboť jsou doma a mezi svýma \]



\song{hlídač krav}{jaromír nohavica}

/D když jsem byl malý říkali mi naši
/ dobře se uč a jez chytrou kaši
/G až jednou vyrosteš /A budeš doktorem /D práv \S

takový doktor sedí pěkně v suchu
bere velký peníze a škrábe se v uchu
já jim ale na to řek chci být hlídačem krav \s

já chci mít čapku s bambulí nahoře
jíst kaštany mýt se v lavoře
od rána po celý den
zpívat si jen
zpívat si pam pam padam pam ... \s

k vánocům mi kupovali hromady knih
co jsem ale vědět chtěl to nevyčet jsem z nich
nikde jsem se nedozvěděl jak se hlídají krávy \s

ptal jsem se starších a ptal jsem se všech
každý na mě hleděl jako na pytel blech
každý se mě opatrně tázal na moje zdraví \s

já chci mít čapku s bambulí nahoře
jíst kaštany mýt se v lavoře
od rána po celý den
zpívat si jen
zpívat si pam pam padam pam ... \s

dnes už jsem starší a vím co vím
mnohé věci nemůžu a mnohé smím
a když je mi velmi smutno lehnu do mokré trávy \s

s nohama křížem a rukama za hlavou
koukám nahoru na oblohu modravou
kde se mezi mraky honí moje strakaté krávy \s

já chci mít čapku s bambulí nahoře
jíst kaštany mýt se v lavoře
od rána po celý den
zpívat si jen
zpívat si pam pam padam pam...



\song{kapr}{jaromír nohavica}

/C na rybníce jsou velké vl/d ny, dělají /G7 se od příro/C dy
kapr, ten je opatr/d ný, nevystr/G7 čí čumák z vo/C dy

\R  \[ /C šplouchy šplouchy, štika žere mou/d chy
    žbluňky žbluň/G7 ky, a kapr meruň/C ky \] **

nevylézej, kapře milý, je tam velké vlnobití
chlupy vodě se naježily, mohl by ses utopiti

\r šplouchy...



\song{kluziště}{karel plíhal}

/C strejček /C\^H kovář /C\^A chytil /C\^G kleště, /F\^{maj7} uštíp' z /C noční /{F\^{maj7} G} oblohy
/C jednu /C\^H malou /C\^A kapku /C\^G deště /F\^{maj7} a ta mu /C spadla /{F\^{maj7} G} pod nohy
/C nejdřív /C\^H ale /C\^A chytil /C\^G slinu, /F\^{maj7} tak šáh' /C kamsi /{F\^{maj7} G} pro pivo
/C pak při/C\^H táhl /C\^A kova/C\^G dlinu /{F\^{maj7} C} a obrovský /{F\^{maj7} G} kladivo

\R  zatím /C tři bílé /C\^H vrány /C\^A pěkně za se/C\^G bou
    kolem /F\^{maj7} jdou, někam /C jdou, do ry/D7 tmu se kýva/G jí
    tyhle /C tři bílé /C\^H vrány /C\^A pěkně za se/C\^G bou
    kolem /F\^{maj7} jdou, někam /C jdou, nedo/F\^{maj7} jdou, nedo/C jdou **

vydal z hrdla mocný pokřik ztichlým letním večerem
pak tu kapku všude rozstřík' jedním mocným úderem
celej svět byl náhle v kapce a vysoko nad námi
na obrovské mucholapce visí nebe s hvězdami

\r zatím tři bílé vrány...

zpod víček mi vytrysk' pramen na zmačkané polštáře
kdosi mě vzal kolem ramen a políbil na tváře
kdesi v dálce rozmazaně strejda kovář odchází
do kalhot si čistí dlaně umazané od sazí

\r zatím...



\song{královské reggae}{dagmar patrasová}

/D umbabaú, umbabaumbaumba
/h umbabaú, umbabaumbaumba \s

/D někdo ráčí kráčet v brnění, má na něm rádoby /A7 er
/D když není urozen, /G  rychle je prozrazen
/D když ne/A7 říká /D krásné er
tak ten si /A7 říká /D o malér \s

moudrý král, princeznin bratranec
když vdával svých pět dcer
řek, to dál nesnesu a prchnul do lesů:
žádný ženich nemá er
kdo nemá er, ztrácí charakter

\R /G královské reggae, /D královské reggae
   /G královské reggae, /A7 královské reggae **

umbabaú... \s

ten, kdo chce mít pravou princeznu
musí být z vyšších sfér
kdo prosí o ruku, musí mít záruku
tou je jeho ryčné er
pouze ten může být premiér
a kdo nemá er, tak je amatér \s

mějte, prosím, trochu strpení
rozhodně všechno je fér
tak proč ty trampoty, příšerné drahoty
přece slyšíš to mé er
které mě pasovalo nad průměr

\r královské reggae...

umbabaú...



\song{lachtani}{jaromír nohavica}

\R  \[ /C lach lach /F jé /C jé, /a lach lach /G jé /C jé \] **

/C jedna lachtaní /F rodi/C na
/a rozhodla se, že si vyjde /G do ki/C na
jeli vlakem, metrem, lodí a pak /F tramva/C jí
a teď /a u kina vesmír /G lachta/C jí
/G lachtaní úspory /C dali dohromady
/G koupili si lístky /C do první /G řady
/C táta lachtan řekl:``nebudem t/F řít bí/C du''
a /a pro každého koupil pytlík /G araši/C dů, ó \S

\r  lach lach...

na jižním pólu je nehezky
a tak lachtani si vyjeli na grotesky
těšili se, jak bude veselo
když zazněl gong a v sále se setmělo
co to ale vidí jejich lachtaní zraky:
sníh a mráz a sněhové mraky
pro veliký úspěch změna programu
dnes dáváme film ze života lachtanů, ó

\rr

táta lachtan vyskočil ze sedadla
nevídaná zlost ho popadla:
``proto jsem se netrmácel přes celý svět
abych tady v kině mrznul jako turecký med
tady zima, doma zima, všude jen chlad
kde má chudák lachtan relaxovat?''
nedivte se té lachtaní rodině
že pak rozšlapala arašidy po kině, jé

\rr

/C tahle lachtaní /F rodi/C na
/a od té doby nechodí už /G do ki/C na, jé



\song{metro pro krtky}{jaromír nohavica}

/G prvá druhá /C třetí /D čtvrt/G á
/G na zahradě /C krtek /D vrt/G á
/e drápy má jak /C vývrtky, óó/G ó
/a vrtá metro /C pro krt/D ky, óó/G ó

\R rrrrrrrrrrrrrrrrrrrrrrrrrrr
   rrrrrrrrrrrrrrrrrrrrrrrrrrr
   rrrrrrrrrrrrrrrrrrrrrrrrrrr
   rrrrrrrrrrrrrrrrrrrrrrrrrrr **

každý kdo si zaplatí, óóó
smí se projet po trati, óóó
od okurek po macešky, jé
dál už musí každý pěšky, jé

\r rrr...



\song{mlýny}{spirituál kvintet}

\R /G slyším mlýnský kámen, jak se otá/G7 čí
   /C slyším mlýnský kámen, jak se otá/G čí
   já /G slyším mlýnský kámen, /H7 jak se otá/e čí
   /C ot/D áč/G í, otá/D čí, otá/C čí **

ty mlýny /G melou celou /C noc, melou /G celý den
melou /C7 bez výhod a melou /G stejně všem
melou doleva, /C melou /G doprava
melou /A pravdu i lež, když zrovna /D vyhrává
melou /G otrok/C áře, melou /G otroky
melou /C7 na minuty, na hodiny, /G na roky
melou /H7 pomalu a jistě, ale /e melou /C včas
já už /G slyším /D7 jejich /G hlas \S

\r slyším...

ó, já chtěl bych aspoň na chvíli být mlynářem
pane já bych mlel, až by se chvěla zem
to mi věřte, uměl bych dobře mlít
já bych věděl, komu ubrat, komu přitlačit
ty mlýny čekají někde nad námi
až zdola zazní naše volání
až zazní jeden lidský hlas
no tak už melte, je čas! \s

\r slyším... \songgg



\song{na políčku v jetelíčku}{petr skoumal, zdeněk svěrák}

/D na /A po/D líčku v /A je/D te/G líč/D ku
/A tam /D kde /A potok /D prame/A ní
/D do/A bře /D je nám /A po /D tě/G líč/D ku
/A zá/D ko/A nem jsme /D chráně/A ni

\R  /D sosá/G čka/D ma /G so/D sá/A me
    /D bez pře/G stá/D ní /G to /D sa/A mé
    /D do pě/A ti, /D do pě/A ti, /D do pě/A ti, /D do pě/A ti
    /D musíme /G být /D u mamin/G ky v /D dou/A pě/D ti **

čmelá, čmelá, čmeláčiska
to jsou kluci veselí
jestli se vám děti stýská
najdete nás v jeteli

\r sosáčkama sosáme...



\song{není nutno}{jaroslav uhlíř, zdeněk svěrák}

\R  /D není nutno, není nutno, aby bylo přímo vese/e lo
    /A7 hlavně nesmí býti smutno, natož aby se breče/{D A7} lo \s
    chceš-li trap se, že ti v kapse zlaté mince nechřestí
    nemít žádné kamarády, tomu já říkám neštěstí **

nemít /h prachy --- /D nevadí
nemít /h srdce --- /D vadí
zažít /h krachy --- /D nevadí
zažít /h nudu --- j/G ó, to /A vadí \S

\r není nutno...



\Song{pramen zdraví z posázaví}{jaroslav uhlíř, zdeněk svěrák}

/G každý /C den, každý den, k svači/a ně jedině
/F jedině /G pramen zdraví z /C posázaví
/G chcete-/C li prospěti dítě/a ti zdravému
/F kupte mu /G pramen zdraví z /C posázaví

\R  /G výrobky /C mléčné to je /a marné
    jsou blaho/F dárné /G a /C věčné **



\song{proklatej vůz}{greenhorns}

\R /C čtyři /E bytelný /a kola /F má náš proklatej /G vůz
   tak /C ještě p/a ár /F dlouhejch /G mil /F zbejvá /G dál /F tam je c/G íl
   a tak /C zpívej ó /F san/G ta /{C F C C7} cruz
   /F polykej whisky a /C zvířenej prach
   /G nesmí nás /C porazit stra/C7 ch
   až /F přejedem támhleten /C pískovej plác
   pak /G nemusíš se /G7 už rudochů /C bát **

george, už jsem celá roztřesená, zastav
zalez zpátky do vozu, ženo!

\r jen tři bytelný kola...

tatínku, tatínku, už mám plnej nočníček
synku, to není nočníček, to je soudek s prachem!!!

\r už jen dvě bytelný kola...

seno, seno, wiki, jedeme jak s hnojem
jó, koho jsem si naložil, toho vezu

\r už jen jediný kolo...

george, když já se tak strašně bojím indiánů!
zatáhni za sebou plachtu, ženo, a mlč!!!!!!!

\r už ani jediný kolo nemá náš bytelnej vůz... \songgg



\song{skřítkové, tesaři}{jaroslav uhlíř, zdeněk svěrák}

\[ /D skřítkové /G tesaři, /D vylezte z /A mechu \]
\[ /D chopte se /D7 náčiní, /G postavte /A střechu \] \S

\[ skřítkové ze skal a skřítkové z lesa \]
\[ pila ať pracuje, sekera tesá \]

\R  /D šupy, dupy, dup, /h raz, dva, tři, hej rup
    /G pidimužík /A pracuje
    /D šupy, dupy, dup, /h raz, dva, tři, hej rup
    /G pidimu/A žík /D kutá
    /D šupy, dupy, dup, /h jedli nebo dub
    /G všechno krásně /A zpracuje
    /D šupy, dupy, dup, /h raz, dva, tři, hej rup
    /G nebozíz/A kem /D vrtá **

/G ať mistr šindelář koná své dílo
/D střechu nám pokryje kdyby snad lilo
/G držte si klobouky, nasaďte helmy
/E7 mistr to provede /A7 způsobem střelným

\r šupy dupy dup...

\[ skřítkové zedníci, zanechte spánku \]
\[ vezměte kladívko, lžíci a fanku \] \s

\[ písek se červená, vápno se bělá \]
\[ brána je zřícená, ať stojí celá \]

\R  \[ šupy, dupy, dup, raz, dva, tři, hej rup
    už to máme pod střechou
    šupy, dupy, dup, raz, dva, tři, hej rup
    už to máme v suchu \] **


\song{statistika}{jaroslav uhlíř, zdeněk svěrák}

je /D statisticky dokázáno
že /h slunce vyjde každé ráno
a /D i když je tma jako v ranci
noc /h nemá celkem žádnou šanci

\R  stati/G stika nuda /A je
    má však /Fis cenné úda/h je
    nekle/G sejme na my/A sli
    ona /G nám to vyčí/D slí **

když drak si z nosu síru pouští
a honza na něj číhá v houští
pak statistika předpovídá
že nestvůra už neposnídá

\r statistika...

tak vyřiďte to ctěné sani
že záleží to čistě na ni
když nepustí ji choutky dračí
pak bude o hlavičky kratší

\r statistika...



\song{trpasličí svatba}{jaroslav uhlíř, zdeněk svěrák}

/G v lese, jó v lese /C na jehlič/G í
koná se svatba /A7 trpaslič/D7 í
/G žádná divná /C věc to nen/H7 í
/C šmudla /G\^{dim} už má /G po voj/e ně a /C tak se /{D7 G} žení \s

malou má ženu, malinkatou
s malým věnem, malou chatou
už jim choděj' telegrámky
už jim hrajou mendelssohna na varhánky

\R  /G protože /D láska, láska, láska v každém /G srdci klíč/D í
    protože láskou hoří i to srdce /G trpaslič/D í
    a kdo se v téhle věci jednoduše /G neopič/D í
    ten ať se dívá, co se děje v lese /G na jehlič/{D D7} í **

mají tam lásku, jako trámek
pláče tchýňka, pláče tchánek
štěstíčko přejou mladí, staří
v papinově hrníčku se maso vaří \s

pijou tam pivo popovický, (ale jen malý)
šmudla se polil, jako vždycky
kejchal kejchá na profouse
jedí hrášek s uzeným a nafouknou se

\r protože láska...

v lese, jó v lese na jehličí
koná se svatba trpasličí
žádná divná věc to není
šmudla už má po vojně a tak se žení



\song{tři čuníci}{jaromír nohavica}

/C v řadě za sebou tři čuníci jdou
ťápají si v blátě cestou nece/a stou
/F kufry nemají, cestu nezna/G jí
/d vyšli prostě do světa a /G vesele si zpívají: \S

ui ui ui ui uí... \s

levá pravá teď přední zadní už
tři čuníci jdou jdou jako jeden muž
žito chřoupají ušima bimbají
vyšli prostě do světa a vesele si zpívají... \s

auta jezdí tam náklaďáky sem
tři čuníci jdou jdou rovnou za nosem
lidé zírají důvod neznají
proč ti malí čuníci tak vesele si zpívají... \s

když se spustí déšť roztrhne se mrak
k sobě přitulí se čumák na čumák
blesky blýskají kapky pleskají
oni v dešti v nepohodě vesele si zpívají... \s

když kopýtka pálí když jim dojde dech
sednou ke studánce na vysoký břeh
do vody koukají kopýtka máchají
chvilinku si odpočinou a pak dál se vydají... \s

za tu spoustu let co je světem svět
přešli zeměkouli třikrát tam a zpět
v řadě za sebou hele támhle jdou
pojďme s nima zazpívat si jejich píseň veselou...



\song{tři kříže}{hop trop}

/d dávám sbohem /C břehům prokla/a tejm
který v /d drápech má /C ďábel /d sám
/d bílou přídí /C šalupa ``my /a grave''
míří k /d útesům, /C který /d znám

\R jen tři /F kříže z bí/C lýho kame/a ní
   někdo /d do písku /C posklá/d dal
   slzy v /F očích měl a v /C ruce znave/a ný
   lodní /d deník, co /C sám do něj /d psal **

první kříž má pod sebou jen hřích
samý pití a rvačka jen
chřestot nožů, při kterým přejde smích
srdce kámen a jméno ``sten'' \s

já, bob green, mám tváře zjizvený
štěkot psa zněl, když jsem se smál
druhej kříž mám a spím pod zemí
že jsem falešný karty hrál \s

třetí kříž snad vyvolá jen vztek
katty rogers těm dvoum život vzal
svědomí měl, vedle nich si klek... \s

{\i recitativ:
  vím, trestat je lidský, ale odpouštět
     božský, ať mi tedy bůh odpustí}

\R jen tři kříže z bílýho kamení
   jsem jim do písku poskládal
   slzy v očích měl a v ruce znavený
   lodní deník a v něm, co jsem psal **



\song{ukolébavka}{jaromír nohavica}

/C den už se seše/C\^H řil, /C\^A už jste si dost u/{C\^H C} žili
tak /d hajdy do /d\^C peřin, a /d\^H ne, abyste tam moc řá/{d\^C d} dili
/C zítra je taky /C\^H den, /C\^A slunko mi to dneska slí/{C\^H C} bilo
/d přejme si hezký /d\^C sen, a /d\^H kéž by se nám to spl/{d\^C d} nilo
/C na na /{a e a} na ..., aby hůř /d nebylo, to by nám /G7 stačilo

\R  /C hajduly, /F dajdu/C ly, /a aby víčka /G sklapnu/C ly
    hajduly /F daj/C dy, /a každý svou peřinu /G naj/C di
    hajajajajaja/{G C} jaja, /a kuba, lenka, máma /G a /C já
    zítra, dřív než slunce /F začne /C hřát...
    /a dobrou noc /G a /C spát **

v noci někdy chodí strach, srdce náhle dělá buch-buchy
nebojte, já spím na dosah, když mě zavoláte, zbiju zlé duchy
zítra je taky den, slunko mi to dneska slíbilo
přejme si hezký sen, a kéž by se nám to splnilo
na na na ..., aby hůř nebylo, to by nám stačilo

\r + a spát, a spát...



\song{zahrada ticha}{jakub smolík}

/D je tam brána zdobená
cestu oteví/e rá
zahradu zele/C nou
/G všechno připomí/D ná \s

jako dým závojů
mlhou upředených
vstupuješ do ticha
cestou vyvolených \s

je to březový háj
je to borový les
je to anglický park
je to hluboký vřes \s

je to samota dnů
kdy jsi pomalu zrál
v zahradě zelený
kde sis za dětství hrál \s

kolik chceš, tolik máš
očí otevřených
tam venku za branou
leží studený sníh \s

z počátku uslyšíš
vítr a ptačí hlas
v zahradě zelený
přejdou do ticha zas \s

světlo připomíná
rána slunečných dnů
v zahradě zelený
v zahradě beze snů \s

uprostřed závratí
sluncem prosvícených
vstupuješ do ticha
cestou vyvolených \songgg



\konec{obsah}
\bye
