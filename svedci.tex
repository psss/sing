\input cantar.sty
\zacatek{zpěvník}{}



\song{tvoje jméno vyznávám}{rick founds, karel řežábek}

/G tvoje /C jméno vyzná/{D C} vám
/G rád ti /C zpívám svoji /{D C} chválu
/G jsem tak /C vděčný, že tě /{D C} znám
/G jsem tak /C vděčný za tvou /{D C} spásu

\R  /G přišel jsi k /C nám z nebe/D s, /C ukázat smě/G r
    dát v oběť /C sám /D sebe, jak /C bůh to /G chtěl
    z kříže /C byls do hrobu /D dán
    třetí /H den jsi z mrtvých /e vstal
    tvoje /a7 jméno /D vyzná/G vám **



\song{ať smolařům je hej}{jaroslav uhlíř, zdeněk svěrák}

/D tolik lidí chodí s hlavou svěšenou \hfill cap III.
/e o štěstí si léta jenom sní
/fis jsou to děti odložené štěstěnou
a smůla /G má je za /A vlastní

\R  dej, pane bože, d/{D G} ej
    ať smolařům je h/{D G} ej
    a protože jsi s/D pravedlivý
    tak /h dělej divy
    a k /e zázrakům se /A měj
    dej, pane bože /{D G} dej
    ať smolařům je /{D G} hej
    dej, ať jsou z toho /D udivení
    že ž/h ivot není jen /e trápení a /A žal
    to prosím, abys /{D G D G} dal **

tolik lidí pátek má i v úterý
nešťastný je v týdnu každý den
splň jim, pane, ze snů aspoň některý
dej znamení, jak z toho ven

\R  dej, pane bože...
    /G to prosím, abys /D dal (3x) **




\song{buchet je spousta}{kouzla králů}

/D z pekárny /G světa kraj za /h láskou /A svou
/D nejednu /G překonal jsem /h vůli /A zlou
/E a moje /A princezna /Fis ta jistě /h rozezná
/G že ji mám /A vážně /D rád \S

a v jeho blízkosti se náhle zdá
že létat v oblacích se málem dá
dát ruku na ruku netřeba záruku
/G jeho se přece nelze /A vůbec bát

\R  /D oslava /G veliká ať /h už se chys/A tá
    /Fis buchet je /h spousta, /E lahodná /A sousta
    /D ať naše /G láska je vždy /h stejně čis/A tá
    a /G pevná jako /A brány /D trám **

hezká jak obrázek a zábavná
a pýchu princezen tak málo zná
vyrostla v dostatku jak holka ze statku
umí za práci brát \s

konečně kluk, co o mě zájem má
přitažlivost je jistě vzájemná
a srdce z rubínu má barvu po vínu
starý král novému vždy má jej dát

\R  ta naše dvojice je překvapivá
    a dle všech známek pekárna - zámek
    je trasa která někdy krátce trvá
    a /G já ji dobře /A lásko /{G A} znám \s

    oslava veliká ať už se chystá
    buchet je spousta, lahodná sousta
    ať naše láska je vždy stejně čistá
    a pevná jako brány trám **


\song{babička mary}{jan werich, jiří voskovec}

/d štěchovická laguna když dřímá v zadumaném stínu kordylér
/g pirát zkrva/d venou šerpu ždímá, /E7 šerif si láduje revol/A ver
/d pikovická rýžoviště zlata čeří se v příboji sázavy
/g ale za to /d krčmářova chata /g křepčí rykem /A7 chlapské zába/d vy \s

/C7 když tu náhle, co se /F děje, /C7 divný šelest houštím /F spěje
/C7 plch, skunk, /F vše utíká /B po stráni /{H\dim} od mední/A ka
/d krčmář zhasne, kovbojové ztichnou, pirát zděšen tvář si zakryje
/G rudé squaw se /d chvějí a pak vzdychnou:
/E7 blíží se k nám postrach préri/{A C7} e

\R  /F mary, babička /G7 mary,
    dva kolťá/C7 ky za pasem, nad hlavou /F točí lasem
    /C7 stoletá /F mary, babička /G7 mary,
    ta zkrotí /C7 křepce hřebce ať chce či nech/{F A7} ce **

žádné zuby, z jelenice sukně, ale za to tvrdé bicepsy
mary má vždy slivovici v putně, toma mixe strčí do kapsy
klika cvakla, v krčmě dveře letí a babička vchází do dveří
"pintu ginu, lumpové prokletí!" bezzubou dásní zaláteří \s

vypiju to jen ve stoje, jdu do volebního boje
zřím zas město drahý, jedu volit do prahy
dopila a aby se neřeklo, putykáře změní v mrtvolu
za zády má štěchovické peklo s šlajsnou svatojánských atolů

\R  mary, babička mary,
    pádluje bez námahy po proudu až do prahy
    stoletá mary, babička mary,
    jde do volebního boje za kovboje ** \songgg

ledva v praze kotvu vyhodila, pro babičku nastal hrozný čas
neboť hned každá strana tvrdila, že jí náleží babiččin hlas
malá stejně jako velká strana psala, že bude mít o hlas víc
že ta druhá strana je nahraná, oni že maj hlas ze štěchovic \s

stoletý věk prý nevadí, na předáka je to mládí
ze všech nejvíce, volala ji polnice
tak babičku, pro kterou vždy byla válka s lidojedy legrace
tu babičku za pár dni zabila, volební agitace

\R  mary, bojovná mary,
    už nesedává v sedle, ve volbách byla vedle
    stoletou mary, babičku mary,
    volbama zabitou vzal k sobě manitou **



\song{amazonka}{hop trop}

byly krásný naše /G plány \hfill cap II.
byla jsi můj celej /{h b a} svět
čas je vzal a nechal /G rány
/a starší jsme jen o pár /D let \S

tenkrát byly děti malý
ale život utíká
už na "táto" slyší jinej
i když si tak neříká

\R  nebe modrý zrca/G dlí se
    v /E7 řece, která všechno /a ví
    stejnou barvu jako /G měly
    /C tvoje oči džíno/D vý **

kluci tenkrát, co tě znali
všude, kde jsem s tebou byl
"amazonka" říkávali
a já hrdě přisvědčil \s

tvoje strachy, že ti mládí
pod rukama utíká
vedly k tomu, že ti nikdo
"amazonka" neříká

\r nebe modrý...

zlatý kráse cingrlátek
jak sis časem myslela
vadil možná trampskej šátek
nosit dáls' ho nechtěla

\R  teď jsi víla z paneláku
    samá dečka, samej krám
    já si přál jen, abys byla
    /C pořád stejná, přísa/G hám, /a pořád stejná, přísa/G hám **




\song{i když jsme plešatý}{pokáč}

/C i když jsme plešatý /a ty a já, já a ty \hfill cap I.
/e nic si z toho nedě/F láme
/C ač máme starostí /a občas víc než dosti
/e aspoň lehčí hlavu /F máme

\R  /d i když se kdekdo nejdřív /G lekne, když nás vidí
    /C nemusí/C\^H te nás lito/a vat, jsme taky lidi
    /d a ždímem každej ze svejch /G dní
    jak měl by bejt ten posle/C dní **

i když jsme plešatý ty a já, já a ty
z maličkostí radost máme
v zimě nám čepice nerozcuchaj kštice
vždy když si je sundáváme

\R  jestli vám naše sestřihy připadaj k smíchu
    tak super smích je lepsí než se nudit v tichu
    i když nám osud ránu dal,
    tak my jsme vstali a jdem dál **

hm/{d G C a d G C} m... \S

i když jsme plešatý ty a já, já a ty
v lecčems jsme dost zvýhodněný
kloubouky nám seknou velice líp snášíme hice
plus ušetříme za fény

\R  a i když nesplňujem stereotyp krásy
    aspoň nemáme strach, že vypadaj nám vlasy
    můžou se horší věci stát,
    třeba se bez boje vzdát **

hm/{d G C a d G C} m... \S




\song{pod dubem, za dubem}{lotrando a zubejda}

\R  /a pod dubem, za dubem - tam si na tě počíháme
    /E pod dubem, za dubem - tam tě vošku/{a G7} bem **

/C loupežníci z povolání, to jsou páni, to jsou páni
/G loupežníci z profese, nejlepší jsou v okrese
/C my řekneme: "ruce vzhůru" a hned máme peněz fůru
/G žádné jiné řemeslo nikdy tolik nene/{C E} slo

\r pod dubem...

hloupý koupí, chytrák loupí, dej sem cukr, dej sem kroupy
seď, formánku, na houni, přepadli tě vrahouni!
loupežník je nesmlouvavý, loupení ho strašně baví
co šlohne, to nevydá, jelikož je nelida

\rr

špatná věc se podařila, hrubá síla zvítězila
tak to chodí na světě, na vzdory vší osvětě
loupežníci z povolání, to jsou páni, to jsou páni
loupežníci z profese, nejlepší jsou v okrese



\song{měla bába trnku}{javory}

/G měla bába trnku, vařila ji v /c hrnku
vařila ji /G dlouho, /f trnka zmize/G la
na dně toho hrnku, co vařila /c trnku
bylo něco k /G pití a /f tak to vypi/G la \S

/f pročpak jsi to pila, ty /G bábo hloupá
/f vždyť se celý svět teď s /G tebou houpá \S

/C švestka bývá slíva, ale taky /f trnka
takhle se /C zpívá v /b zemi valaš/C ské
ve vodě jsou žáby a kafe pijou /f báby
a všichni mají /C rádi /b trnky teku/C té \S

/f vizovice, vsetín, /G karolinka, karlovice
/f ve valašských horách, /G tam je pohoda
/f borovička, krkovice, /G tekelice, slivovice
/f ten kdo chce být zdravý, /G ten ať si ji dá!





\song{to ta heľpa}{lidová}

/d to ta hel/g pa, /d to ta hel/g pa, /d to je /A7 pekné /d mesto
/d a v tej hel/g pe, /d a v tej hel/g pe /d švarných /A7 chlapcov /d je sto.
\[ /B koho je /C sto, /F toho je sto, /g nie po mojej /{F A7} vóli,
/d len za jed/g nym, /d len za jed/g nym /d srdie/A7 čko ma /d boli \] \S

za janíčkom, za palíčkom krok by něspravila,
za ďuríčkom, za mišíčkom dunaj preskočila.
\[ dunaj, dunaj, dunaj, dunaj, aj to širé pole,
len za jedním, len za jedním, počešenie moje. \]



\song{pohár a kalich}{klíč}

/e zvedněte poháry, /D pánové, už jen /G poslední /D přípitek /e zbývá /{cap II.} ~
pohleďte, nad polem /D špitálským /G vrcholek /D roubení /G skrývá
\[ za chvíli /D zbyde tam popel a /e tráva
    nás čeká /D vítězství, bohatství, /e sláva. \] /{D e D e} ~\S

ryšavý panovník jen se smál, sruby prý dobře se boří
palcem k zemi ukázal, ať další hranice hoří
\[ ta malá země už nemá co získat
teď bude tancovat, jak budem pískat. \]

\R  /a náhle se pozvedl /D vítr a mocně /G vál
    /a odněkud přinesl /D nápěv, sám si ho /H7 hrál, /{a G e H7} hm...
    /e zvedněte poháry, /D pánové, večer z /e kalichu /D budeme /e pít **

nad sruby korouhev zavlála, to však neznačí, že je tam sláva
všem věrným pokaždé nesmírnou sílu a jednotu dává
\[ ve znaku kalich, v kalichu víra
jen pravda vítězí, v pravdě je síla. \] \s

"modli se, pracuj a poslouchej," kázali po celá léta
mistr jan cestu ukázal proti všem úkladům světa
\[ být rovný s rovnými, muž jako žena
dávat všem věcem ta pravá jména. \] \s

do ticha zazněla přísaha - ani krok z tohoto místa
se zbraní každý vyčkává, dobře ví, co se chystá
\[ nad nimi stojí muž, přes oko páska
slyšet je dunění a dřevo praská. \]

\R  /a náhle se pozvedl /D vítr a mocně /G vál
    /a odněkud přinesl /D nápěv, sám si ho /H7 hrál:
    /a kdož sú boží bo/G jovní/e ci a zákona /{H7 e} jeho
    /C pros/e tež od /a bo/C ha /a pomo/e ci a /a úfajte v /G něho... **



\song{sáro}{traband}

\R /a sáro, /e sáro, v /F noci se mi zdá/C lo
   že /F tři andělé (boží) k /C nám /F přišli na ob/G ěd
   /a sáro, /e sáro, jak /F moc a nebo /C málo
   mi /F chybí abych /C tvojí duši /F mohl rozum/G ět? **

sbor /a kajícných /e mnichů jde /F krajinou v /C tichu
a pro /F všechnu lidskou /C pýchu má jen /F přezíravý sm/G ích
z /a prohraných /e válek se /F vojska domů /C vrací
však /F zbraně stále /C burácí a /F bitva zuří v /G nich

\r sáro...

vévoda v zámku čeká na balkóně
až přivedou mu koně, pak mává na pozdrav
a srdcová dáma má v každé ruce růže
tak snadno poplést může sto urozených hlav

\rr

královnin šašek s pusou od povidel
sbírá zbytky jídel a myslí na útěk
a v podzemí skrytí slepí alchymisté
už objevili jistě proti povinnosti lék

\rr

páv pod tvým oknem zpívá sotva procit
o tajemstvích noci ve tvých zahradách
a já - potulný kejklíř, co svázali mu ruce
teď hraju o tvé srdce a chci mít tě nadosah

\R  sáro, sáro, pomalu a líně
    s hlavou na tvém klíně chci se probouzet
    sáro, sáro, sáro, rosa padá ráno
    a v poledne už možná bude jiný svět
    /F sáro, /C sáro, /F vstávej, milá /C sáro!
    /F andělé k nám /d přišli na ob/C\^{maj7} ěd **



\song{lano, co k nebi nás poutá}{traband}


já /F sedával v přístavu, /g popíjel kořalu, s /F holkama laško/C val
a /F bylo mi fuk, co je, /g hlavně když fajfka mi /F doutn/C á
co /F bylo už není, /g všechno mý jmění jsem /A7 dávno rozfofro/d val
jsme /F silný jak silný je /g lano, co k nebi nás /{F C F} poutá \S

ale najednou zmatek, když vešel ten chlápek, na mou duši!
objedná si drink a sedne si vedle do kouta
pak se nakloní ke mně a povídá jemně: matouši!
jsme silný jak silný je lano, co k nebi nás poutá \s

já povídám: pane, odkud se známe? esli se nemýlíte?
a co je vám do mě, starýho mrchožrouta?
on na to: pojď, dej se na moji loď, má jméno eternité
jsme silný jak silný je lano, co k nebi nás poutá \s

d A7 d A7 d
d C B A7 d C d A7 d\S

/G ty jeho slova se /a zařízly do mě /G jako bys břitvou /D šmik
/G jako když po ránu /a vzbudí tě křik ko/G hout/D a
tak /G povídám: jdem! a /a ještě ten den stal se /H7 ze mě námoř/e ník
jsme /G silný jak silný je /a lano, co k nebi nás /{G D G} poutá \S

tak zvedněme kotvy a napněme plachty, vítr začíná vát!
černý myšlenky vymeťme někam do kouta
hudba ať hraje o dobytí ráje, teď není čeho se bát
jsme silný jak silný je lano, co k nebi nás poutá
jsme silný jak silný je lano, co k nebi nás poutá \s

e D e D e D e D e
G D G D G D G D G


\song{mám doma kočku}{pokáč}

/G jednou mi drahá říká: /e miláčku, já chci kočku
/C já na to, že chci spíš psa /D a jeli jsme pro kočku \S

v útulku v měcholupech maj dobrej výběr, zdá se
kočky jsou zcela zdarma jen očkování za patnáct set \s

po dlouhým vybírání diskuzích a dvou hádkách
jedno z koťátek jemně sevře můj prst ve svých drápkách \s

tak volba byla jasná má drahá slzy suší
co si však vezem domů to ani jeden z nás netuší

\R mám doma kočku, mám ji rád je to skvělý kamarád
   jen od tý doby, co ji znám mám ruce nohy samej šrám
   dávno už se nemazlí žít s ní je pro zdraví fakt zlý
   každý den projdu terorem mám nejspíš kočku s errorem **

po pár dnech soužití vše vypadá nevinně
koťátko v náručí spí a pak zkadí půlku kuchyně \s

prostě nic, co nečekáš když útulek zaručí
že kočička se za pár týdnů na záchůdek naučí \s

jenže pár měsíců je pryč záchod zeje prázdnotou
bytem čpí příšernej smrad za kočkou lítám s lopatou \s

když jí zkusím pomoct sám na záchod ji přemístit
zvládne mi dlaně rozdrásat a ještě potřísnit

\r mám doma kočku...

za noc nejmíň pětkrát jí okno otevřít jdu
pak práci šéf se zeptá proč vždy tak ospalý přijdu \s

že občas hrdě nám donese zakouslou myš
poznáš až ráno když se s torzem myši v ústech vzbouzíš \s

tak jsme si dobrovolně pořídili katastrofu
stálo nás to patnáct stovek a jednu zbrusu novou sofu \s

kouše, škrábe, drásá no prostě je to krása
tváří se tak roztomile zvlášť, když se mi rejpe v žíle

\rr \songgg

ráda /e dlouze sedá v /C okně a/D nebo ve dveř/e ích
bůh ví, proč nejradši v zimě když venku padá sníh

když pak v /e bytě máme /C mínus dvacet /D tak nás zahří/G vá
aspoň pocit, že v tu chvíli naši krev neprolívá
ou!

\rr

mňau



\song{díky za každé nové ráno}{javory}

/F díky /d za každé /C nové ráno, /F díky /d za naděj /g nových dn/C ů
/F díky /d za to, že v /g dálce /C mizí /d tíseň /C špatných /F snů \S

/F díky /d za každou /C novou píseň, /F díky /d za to, co /g nadějí zn/C í
/F díky /d za to co /g dávno v/C í se, /d že je /C krásné /F žít

\R dík /g za všechno to dobr/F é, co nás /C potkalo
   dík /g za všechno to zl/F é, co se nám /C nestalo **

díky, za každé nové ráno, díky za víru, za ideál
díky za cestu darovanou, kterou půjdem dál


\song{modlitba}{javory}

pane, /C ať jsi stéblo trávy nebo /G obyčejný /C list
prosím dej, ať aspoň trochu umím /F ve tvých vzkazech čí/C st
\[ prosím /F dej, ať řeči /G stromů aspoň /a trochu rozumí/F m
ať vědí, /C že se učím /G a že nic neum/C ím \] \S

dej, ať zlomím svoji pýchu, dej mi hledat pokoru
když se trápím zbytečnostmi, ať pohlédnu nahoru
\[ ať mi stačí dohlédnout na obzor, který jsi mi dal
ať se smířím se vším, co jsi mi kdy vzal \] \s

a dej mi sílu snášet pokorně, co změnit nemám sil
odvahu abych to, nač stačím, na tomhle světě pozměnil
\[ a také prostý rozum, který vždycky správně rozezná
co se změnit nedá a co se změnit dá \]


\song{kdysi před léty}{javory}

/G kdysi před léty si /D lidi občas /G zazpívali
kdysi před léty /D byl prý na to /G klid
dneska je to jinačí, /D co bychom si /G namlouvali
ale říkat si to nestačí, /D pojďte vstát a /G jít
jít krajinou a /D učit se zas /G slyšet slova
těch starých písní, /D které člověk /G znal
zpívat a hrát a /D sem-tam si i /G pamatovat
kdo jen chvíli stál, že /D stojí /G opodál



\song{mezi horami}{čechomor}

\[ /e mezi /D hora/e mi, /G lipka /D zelen/G á \]
\[ /G zabili janka, /D janíčka, /e janka, miesto /h jeleň/e a \] \S

\[ keď ho zabili, zamordovali \]
\[ na jeho hrobě, na jeho hrobě, kříž postavili \] \s

\[ ej křížu, křížu, ukřižovaný \]
\[ zde leží janík, janíček, janík, zamordovaný \] \s


\[ tu šla anička, plakat janíčka \]
\[ hned na hrob padla a viac nevstala, dobrá anička \]


\song{v 9 hodin 25}{samson lenk}

\R  /{a D F E a E a D F E a E} uap tadap ... **

v /a devět hodin dvacet pět mě /D opustilo štěstí
ten /F vlak, co jsem jím měl jet, na koleji /E dávno /E7 nestál
v /a devět hodin dvacet pět /D jako bych dostal pěstí
já /F za hodinu na náměstí měl jsem /E stát, ale v /E7 jiným městě \S

tvá /A7 zpráva zněla prostě a byla tak krátká
že /d stavíš se jen na skok, že nechalas' mi vrátka
/G zadní otevřená, /E zadní otevřená
já /A7 naposled tě viděl, když ti bylo dvacet
to /d jsi tenkrát řekla, že se nechceš vracet
/G že jsi unavená, /E ze mě unavená

\r uap tadap...

já čekala jsem, hlavu jako střep, a zdálo se, že dlouho
může za to vinný sklep, že člověk často sleví
já čekala jsem, hlavu jako střep, s podvědomou touhou
já čekala jsem dobu dlouhou, víc než dost, kolik přesně, nevím \s

pak jedenáctá bila a už to bylo passé
já dřív jsem měla vědět, že vidět chci tě zase
láska nerezaví, láska nerezaví
ten list, co jsem ti psala, byl dozajista hloupý
byl odměřený moc, na vlídný slovo skoupý
už to nenapravím, už to nenapravím

\rr



\song{sbohem galánečko}{}

/D sbohem galá/h nečko, /e já už musím /{A D} jíti
/A sbohem galá/fis nečko, /h já už musím /{E A} jíti
\[ /e kyselé ví/A nečko, /D kyselé /G ví/D neč/A ko
/D podala´s /G mě k /{D A D} pití \]\S

\[ ač bylo kyselé, preca som sa opil \]
\[ eště včíl sa stydím \]
co jsem všecko tropil \s

\[ sbohem galánečko, rozlúčme sa v pánu \]
\[ kyselé vínečko \]
podala´s mě v džbánu \s

\[ ale sa něhněvám, že´s mňa ošidila \]
\[ to ta moja žízeň \]
ta to zavinila



\song{poutník a dívka}{spirituál kvintet}

/A kráčel krajem poutník, šel sám
/D kráčel krajem poutník, šel /A sám
kráčel krajem poutník, kráčel /{Cis fis} sám
tu potkal /H dívku, nesla /H7 džbán, přistoupil k /E7 ní a pravil: \s

``ráchel, ráchel, žízeň mě zmáhá
ráchel, ráchel, žízeň mě zmáhá
ráchel, ráchel, žízeň mě zmáhá
tak přistup blíže, nehodná, a dej mi pít,'' a ona: \s

``kdo jsi, kdo jsi, že mi říkáš jménem
kdo jsi, kdo jsi, že mi říkáš jménem
kdo jsi, kdo jsi, že mi říkáš jménem
já tě vidím poprvé, odkud mě znáš?'' \s

``ráchel, ráchel, znám víc než jméno
ráchel, ráchel, znám víc než jméno
ráchel, ráchel, znám víc než jméno,''
pak se napil, ruku zdvih' a kráčel dál \s

ten džbán, ten džbán z nepálené hlíny
ten džbán, ten džbán z nepálené hlíny
ten džbán, ten džbán z nepálené hlíny
v onu chvíli zazářil kovem ryzím \s

kráčel krajem poutník, šel sám
kráčel krajem poutník, šel sám
kráčel krajem poutník, kráčel sám
ač byl /H chudý, nepoz/E nán, přece byl /{A D A} král



\song{domov na zemi}{spirituál kvintet}

/D jak léta jdou, svět /G pro mě ztrácí /D glanc
všichni se rvou a /A7 duši dávaj' všanc
a /D za pár šestáků vás /G prodaj', věřte /D mi
už víc nechci mít /G domov /D svůj /A7 na ze/D mi! \S

\R  čas žádá svý a mně se krátí dech
    když před kaplí tu zpívám na schodech
    svou píseň vo nebi, kde bude blaze mi
    už víc nechci mít domov svůj na zemi! **

po jmění netoužím, jsme tu jen nakrátko
i sláva je jak dým, jak prázdný pozlátko
já koukám do voblak, až anděl kejvne mi
už víc nechci mít domov svůj na zemi!

\r čas žádá...

teď říkám ``good-bye'' světskýmu veselí
těm, co si užívaj', nechci lízt do zelí
jsem hříšná nádoba, však spása kyne mi
už víc nechci mít domov svůj na zemi!

\rr

v určenej čas kytara dohraje
zmlkne můj hlas na cestě do ráje
vo tomhle špacíru noc co noc zdá se mi
už víc nechci mít domov svůj na zemi!

\rr



\song{až to se mnu sekne}{nohavica}

/d až obuju si rano /A černe papirove /d boty
/F až i moje stara pocho/C pi, že nejdu do ro/F boty
/g až vyjde dluhy pruvod smutečnich hostu
na /d slezsku ostravu od sykorova mostu
/A až to se mnu sekne
to bude /d pěkne, /g pěkne, fajne a /d pěkne
/A až to se mnu definitivně /d sekn/{A d A} e \S

aby všeckym bylo jasne, že mě lidi měli rádi
ať je gulaš silny, baby smutne, muzika ať ladi
bo jak sem nesnašel šlendrijan ve vyrobě
nebudu ho trpět ani co sem v hrobě
to bude pěkne... \s

s někerym to seka, že až neviš, co se robi
jestli pomohla by deka nebo teplo mlade roby
kdybych si moh vybrat chtěl bych hned a honem
ať to se mnu šlahne tak jak ze starym magdonem
to bude pěkne... \s

jedine, co nevim, esi startku nebo spartu
bo bych tam nahoře v nebi nerad trhal partu
na každy pad s sebu beru bandasku s rumem
bo rum nemuže uškodit, když pije se s rozumem
to bude pěkne... \s

já vim, že bože nejsi, ale kdybys třeba byl, tak
hoď mě na cimru, kde leži stary lojza miltak
s lojzu chodili sme do orlove na zakladni školu
farali sme dolu tak už doklepem to spolu
až to se mnu sekne
pěkne, to bude pěkne
až to se mnu definitivně sekne \s

až obuju si rano černe papirove boty
až i moje stara pochopi, že nejdu do roboty
kdybych co chtěl dělal všechno malo platne
mohlo to byt horši nebylo to špatne
až to se mnu sekne
kdybych co chtěl dělal všechno malo platne
mohlo to byt horši nebylo to špatne
až to se mnu \s

na na na...



\Song{válka růží}{}

už /d rozplynul se /G hustý dým, /d derry down, hej /A down a down
/d nad ztichlým polem /g válečným, derry /{d A} down
jen /F ticho stojí /C kol ko/A lem a /d vítěz plení /B vlastní z/A em
je válka /d růží --- /g derry derry /A derry down, /d down \S

nečekej soucit od rváče, derry down, hej down a down
kdo zabíjí, ten nepláče, derry down
na těle mrtvé krajiny se mečem píšou dějiny
je válka růží, derry derry derry down, down \s

dva erby, dvojí korouhev, derry down hej down a down
dva rody živí jeden hněv derry down
kdo změří, kam se nahnul trůn, zda k yorkům nebo \rest{k lancasterům}
je válka růží, derry derry derry down, down \s

dva rody, dvojí korouhev, derry down, hej down a down
však hlína pije jednu krev, derry down
ať ten či onen přežije, vždy nejvíc ztratí anglie
je válka růží, derry derry derry down down



\song{už se nám čas krátí}{}

/a kde je, kde je /C sláva říma
/G kde je, kde je /C sláva /C7 říma
/F v dávných troskách /a dávno /e dřímá
/a tam, /C kde i /a světlo /G mizí v /{C E7} tmách \S

\[ kudy bloudí prázdná slova? \]
zjeví se a ztratí znova
tam, kde i světlo mizí v tmách \s

\R \[ už se, už se nám čas krátí. \]
   přijde den a půjdem spáti
   tam, kde i světlo mizí v tmách **

\[ kam se, kam se ztratí činy? \]
mrtvé činy z naší viny
tam, kde i světlo mizí v tmách \s

\[ kam se, kam se ztrácí touha? \]
únava nám zbývá pouhá
tam, kde i světlo mizí v tmách \s

\r už se...

\[ kde jsou, kde jsou velké lásky? \]
půvabné jak sedmikrásky
tam, kde i světlo mizí v tmách \s

\[ kde se, kde se skrývá pravda? \]
pálená i mrazem zvadlá
dál vábně voní věčnou tmou \s

už se, už se nám čas krátí
kde se, kde se skrývá pravda
přijde den a půjdem spáti
dál vábně voní věčnou tmou



\song{telegrafní cesta}{poutníci}

(to) /G tenkrát dávno šel /e pustinou muž
a /D na pravým místě tam /C vykácel buš
z klád /C postavil d/D ům že měl /G sílu jak b/D ýk
a /C rozoral zem jako /a válečník \S

jenže za ním jdou další a ti umí víc
vázaný krovy a zdi z vepřovic
a do zlatejch polí a bučení krav
po ňáký době zní telegraf \s

a už je tu kostel a konečně most
a železná ruda a zločinnost
a okresní město má okresní soud
a ta stará trať jméno telegraph r/e oad \S

těžký jdou časy teď znova a znova
skončila válka a chystá se nová
rozmoklou stezkou, co prošel ten skaut
v deseti proudech jdou provazy aut
jak /(C?) dravá ř/G eka \s

a /a rádio hlásí, že v noci byl mráz
/e lidi jdou z práce a nemají čas
jenže /D vlak domů má /h velký zpoždě/e ní \S

už nemůžu dělat, co kde bych kdy chtěl
třeba v pralesích kácel --- to bohužel
můžu jen sklízet, co zasil jsem sám
a zaplatit všechno, co komu kde mám \s

ti šedaví ptáci na drátech z mědi
vo tomhletom kódu už ledacos vědí
ti můžou letět a zapomenout
na celej řád týhle telegraph road \songgg

víš, že bývaly časy, kdy bylo to zlý
kdy spali jsme v dešti a promrzlí
teď když mi říkáš --- tak vem si co chceš
moc dobře cítím, že už je to lež \s

ale důvěřuj ve mě, dej kočímu bič
já vezmu tě s sebou a odvedu pryč
ode všech temnot a vysokejch zdí
od všeho strachu, co v ulicích spí
já prošel jsem pamětí kdekterej kout
a viděl jen smutek jak pýcha se dmout
a chtěl bych zapomenout
na všechny ty zákazy vjezdu, který jsou rozsetý
po celý telegraph road



\song{rána v trávě}{žalman}


\R  /a každý ráno boty /G zouval,  /a orosil si nohy v /G trávě
    /a že se lidi mají rádi, /G doufal, /a a pro/e citli /a právě
    /a každý ráno dlouze /G zíval, /a utřel čelo do /G rukávu
    /a a při chůzi tělem sem-tam /G kýval, /a před se/e bou sta /a sáhů **

/C poznal /G mora/F věnku /C krásnou
/a a ví/G nečko /C ze zlata
v čechách /G slávu /F muzi/C kantů
/a uma/e zanou /a od bláta

\r  každý ráno...

toužil najít studánečku
a do ní se podívat
by mu řekla: proč, holečku
musíš světem chodívat \s

studánečka promluvila:
to ses' musel nachodit
abych já ti pravdu řekla
měl ses' jindy narodit

\rr



\song{johanka}{}

/a s hlavou /e skloně/d nou lidí zástup se tu /a dívá
nebe /e nad hla/d vou, slyšíš /G dětskej /a pláč
jenom s /e vírou /d svou stojí dívka plavá, /a bílá
oheň /e nad se/d bou, jenom s /G pravdou d/a ál \S

\R /E hej, muži, přidej /a oheň spí, vždyť /E páni se nudí /a jen
   ať /E plameny nesou /F zprávu zlou, jak /C skončil
   soudnej /G den
   s johan/a kou **

dík tvůj dal ti král, celá francie si zpívá
to se osud smál, smutek utíká
s naší johankou ke štěstí se země dívá
vítr zprávu vál, že se dýchat dá \s

\r hej, muži...

s hlavou skloněnou lidí zástup se tu dívá
nebe nad hlavou slyšíš dětskej pláč
popel s vánkem ví, co se v dívčím srdci skrývá
hra se zastaví, jiná začíná \s

\r hej, muži...



\song{filimi}{}

/e čert aby vzal už tuhle trať
kdo /G nemáš práci tak se ztrať
/e že nemáš prachy?! --- no tak ať!
jó, tak se /h na to /e dívám! \S

\R /e filimi jori jůri ej, /G filimi jori jůri ej
   /e filimi jori jůri ej, vo tom /h si teď /e zpívám **

jen pražec chop a kolej suň
chyť lano, táhni jako kůň
po tíhou jako medvěd fuň
jó, tak se na to dívám! \s

z kůže se loupeš jako had
je vedro, že by jeden pad´
na vodu smíš jen vzpomínat
jó, tak se na to dívám! \s

když konečně máš vody dost
určitě přes ni stavíš most
kláda ti ráda zlomí kost
jó, tak se na to dívám \s

na rukách už jsem potěžkal
většinu těch okolních skal
ještě to cejtí každej sval
jó, tak se na to dívám! \s



\song{david a goliáš}{J+V+W}

/D lidi na li/h di jsou jako /A saně
/D člověk na člo/h věka jako /A kat
/D podívejte /H se na ně, /e musíte /A7 naří/{D Cis7} kat
/fis obr do pidimužíka /Cis7 mydlí
/fis domnívaje se, že vyhra/Cis7 je
/A klidně /Fis7 seďme /h na ži/E7 dli
/A čtěme bibli, tam to všechno je: \S

/D samuelova kniha nám /e poví/A7 dá
/D jak na žida přišla veli/e ká bí/A7 da
/D jak ti /D7 bídní /h filiš/D7 tíni %
/G válku vést ne/g byli líní
/D až potkali /B7 davi/A7 da \S

david šel do války volky nevolky
z velké dálky nesl bratrům homolky
v pochodu se cvičil v hodu %
dal si pro strýčka příhodu
tři šutry do /{B7 A7 D} tobolky. /\ D7 hej \S

/B hej --- kam se /D valej vždyť jsou /E7 malej!
takhle /A7 goliáš ho /Fis7 provokuje
/e7 david slušně /A7 salutuje
/D když mu ale obr plivnul /e do o/A7 čí
/D david se otočí, prakem /e zato/A7 čí
/D když za/D7 čínáš, /h no tak /D7 tu máš
/G byl jsi velkej, /g já měl kuráž %
/D a jakej byl /{E7 A7 D A7 D} goliáš



\song{autobusy přijíždějí}{}

/C autobusy přijíždějí, autobusy přijíždě/G jí
/C možná, že už zítra, /e možná, že už /a zítra
/F pojedeme /C za na/G děj/C í \S

\[ známe zemi vyvolenou \]
\[ do vysněné brány \]
nevejdeme po kolenou \s

\[ pojedeme staří, mladí, \]
\[ že tu zůstat nechcem \]
našim pánům nejvíc vadí. \s

\[ policajti s tváří bledou \]
\[ marně do nás buší \]
stejně s námi nepojedou. \s

\[ pod jednou či podobojí \]
\[ spojme svoje srdce, \]
svoboda, ta za to stojí. \s

\[ povedeme život nový \]
\[ nepobrali všechny \]
snad se vozy vrátí, kdo ví \s



\song{žízeň}{}

když /C kapky deště buší na /F rozpá/a le/G nou /C zem
já toužím celou duší dát /F živou vodu /C všem
už v knize knih je psáno: bez /F vody /a nel/G ze /C žít
však ne každému je dáno z /F řeky pravdy /C pít \S

\R já mám /G žízeň, věčnou /{C C7} žízeň
   stačí ř/F íct, kde najdu vlá/C hu a zchladím /F žá/C hu páli/G vou
   já mám žízeň, věčnou /{C C7} žízeň
   stačí ř/F íct, kde najdu vlá/C hu a zmizí /{F C} žízeň **

stokrát víc než slova hladká jeden čin znamená
však musíš zadní vrátka nechat zavřená
mně čistá voda schází, mně chybí její třpyt
vždyť z moře lží a frází se voda nedá pít \s

jak vytékají říčky zpod úbočí hor
tak pod očními víčky se ukrývá můj vzdor
ten pramen vody živé má v sobě každý z nás
a vytryskne jak gejzír, až přijde jeho čas



\Song{život je jen náhoda}{}

/G život je jen /C7 náhoda, /G jednou si dole jednou /G7 nahoře
/C život plyne /c jako voda a /G smrt je /D jako mo/G ře \S

každý k moři dopluje, někdo dříve a někdo později
kdo v životě miluje, ať neztrácí naději \s

/C až uvidí v životě /G zázraky, /C které jenom láska u/G mí
/A7 zlaté ryby vyletí nad mraky, /D pak po/{D\dim D7} rozumí \S

že je život jak voda, kterou láska ve víno promění
láska že je náhoda a bez ní štěstí není



\song{za svou pravdou stát}{}

máš /a všechny trumfy mládí a /G ruce čistý /a máš
jen /d na tobě teď /a záleží na /E jakou hru se d/a áš --- /E musíš \S

\R /a za svou pravdou /E stát
   za svou pravdou /a stát
   musíš /a za svou pravdou /E stát
   za svou pravdou /a stát **

už víš, kolik co stojí, už víš, co bys rád měl
už ocenil jsi kompromis a párkrát zapomněls \s

\r že máš za svou pravdou stát...

už nejsi žádný elév, co prvně do hry vpad
už víš, jak s králem ustoupit a jak s ním dávat mat \s

\r tak hleď za svou pravdou stát...

teď přichází tvá chvíle, teď nahrává ti čas
tvůj  sok poslušně neuhnul a ty  mu zlámeš vaz \s

\r neměl za svou pravdou stát...

tvůj potomek ctí tátu, ty vštěpuješ mu rád
to heslo, který dobře znáš z dob, kdy jsi bejval mlád \s

máš všechny trumfy mládí a ruce čistý máš
jen na tobě teď záleží na jakou hru se dáš \s

\r musíš za svou pravdou stát...



\song{vlaštovka}{nohavica}

/C vlaštovko leť /a přes čínskou zeď
/F přes písek /C pouště go/G bi
/C oblétni zem /a přileť až sem
/F jen ať se /C císař zlo/G bí
/e dnes v noci zdál se mi /a sen
/F že ti zrní nasypal /G ludvig van beethoven
/C vlaštovko leť, /a nás chudé veď /{F C} {\ } \S

zeptej se ryb, kde je jim líp
zeptej se plameňáků
kdo závidí nic nevidí
z té krásy z pod oblaků
až spatříš nad sebou stín
věz, že ti mává sám pan jurin gagarin
vlaštovko leť, nás chudé veď \s

vlaštovko leť rychle a teď
nesu tři zlaté groše
první je můj, druhý je tvůj
třetí pro světlonoše
až budeš unavená
pírka ti pofouká máří magdaléna
vlaštovko leť, nás chudé veď \s



\song{tichá domácnost}{ebenové}

/h7 není doma /E vždycky všechno /D\^E tak
/Amaj7 jak by si člověk /Dmaj7 představoval
/h7 někdy to de pr/E ávě nao/D\^E pak
s /Amaj7 tím bych vás nerad /Dmaj7 unavoval

\R  /h7 u nás se nekři/cis čí, /fis u nás se nespí/cis lá
    /h7 u nás je zvláštní /E idyla

    /A tichá, /D tichá /E naše /Dmaj7 domácnost je /A\^{Cis} tichá, /D tichá
    a s /E konverzací /D nikdo nepos/A píchá
    /D tichá, /E naše /Dmaj7 domácnost /A\^{Cis} tichá, /D tichá
    jen /E vodovodní /D kohout tiše vzd/cis ychá, /fis tichá
    je /G poz/D dě /G hon/D it /G bych/A a **

není doma vždycky všechno tak
jak by to žena měla v plánu
celý večer čekáte a pak:
on přijde o půl páté k ránu

\r u nás se nekřičí...



\song{pískající cikán}{}

/G dívka /a loudá se /G vini/D cí
/G tam, kde z/a ídka je /h níz/D7 ká
/G tam, kde str/a áň končí /h voní/C cí
si /G písni/C čku někdo /{G C D} píská \S

ohlédne se a ``pro pána!''
v stínu, kde stojí líska
švarného vidí cikána
jak leží, písničku píská \s

chvíli tam stojí potichu
písnička si ji získá
domů jdou spolu do sklípku
je slyšet cikán jak píská \s

jenže tatík jak vidí cikána
pěstí do stolu tříská
ať táhne pryč vesta odraná
groš nemá, něco ti zpíská \s

teď smutnou dceru má u vrátek
jen bůh ví jak se jí stýská
``kéž vrátí se mi zas nazpátek
ten, který v dálce si píská!''\s

a šídla honí se na louce
v trávě rosa se blýská
cikán v rozmarným klobouce
jde dál a písničku píská \s

na závěr zbývá už jenom říct
v čem je ten kousek štístka
peníze často nejsou nic
má víc, kdo po svém si píská



\song{nezacházej, slunce}{žalman}

/G nezacházej, slunce, neza/a cházej ještě
/D já mám potěšení /C na da/D lekej cestě
/G já mám potě/{h e} šení    /a na da/D lekej ces/G tě \S

já má potěšení mezi hory-doly
\[ žádnej neuvěří, co je mezi námi \] \s

mezi náma dvouma láska nejstálejší
\[ a ta musí trvat do smrti nejdelší \] \s

trvej, lásko, trvej, nepřestávaj trvat
\[ až budou skřivánci o půlnoci zpívat \] \s

skřivánci zpívali, můj milej nepřišel
\[ on se na mě hněvá, nebo za jinou šel \]



\Song{hvězda na vrbě}{karel mareš -- jiří štaidl}

kdo se /a večer /e hájem /a vrací, /F ten ať /e klop/G7 í zra/{C e} ky
ať /G je /a nikd/e y neo/a brac/d í k vr/G7 bě, /e křivola/{E A C e} ký
jin/G ak /a jeho /e oči /a zjistí, /F i když /e se to /G7 nezd/{C e} á
že /G na /a vrbě, /e kromě /a listí, /d visí /F malá hvěz/a da \S

viděli /C jsme jednou v /F lukách, plakat /C na tý vrbě /F kluka
který /d pevně věřil /B tomu, že ji /D7 sundá z toho /{G7 e} stromu \S

kdo o hvězdy jeví zájem, zem, když večer hladne
ať jde klidně někdy hájem, hvězda někde spadne
ať se pro ni rosou brodí a pak vrbu najde si
a pro ty, co kolem chodí, na tu větev zavěsí



\song{husličky}{}

\[ /D čiže jste husličky, /{G D} čije
/e kdo vás tu /D zane/A chal \]
/e na trávě /A pová/D lané, /e na trávě /A pová/{D fis} lané
/e u paty /D oře/A ch/{e D A} a \S

\[ kdože tu trávu tak zválal
aj modré fialy \]
\[ že ste husličky samé \]
na trávě ostaly \s

kery tu muzikant usnul
co sa mu přišlo zdát
kery tu mládenec usnul
co sa mu přišlo zdát
co sa mu enem zdálo
bože co sa mu v noci zdálo
že už dál nechtěl hrát \s

\[ zahrajte husličky samy
zahrajte zvesela \]
až sa tá bude trápit
bože až sa tá bude trápit
\[ kerá ho nechtěla \]



\song{farao}{}

jen /D žhavý písek žhavá poušť a /G žhavý vzduch tam /D byl
pod žhavým sluncem chudý lid a /G řeka jménem /D nil
ti lidé byli otroci, nad /G nimi /C kara/G báč /D stál
nad karabáčem velekněz a /h nade /A všemi /D král \S

říkali mu farao a byl tak vysoko, že nevrhal ani stín
zato měl sýpky, sklepy, zlato a lesklá těla otrokyň \s
říkali mu farao a ten když rukou hnul duněly bubny k~obřadům
tančily kněžky hlavy padaly to aby nepad jeho trůn \s

\R \[ svůj /D trůn měl farao rád, měl /A rád, měl /D rád
   nechtěl ho nikomu dát to na žá/A dnej /D pád. \]**

říkali mu farao a musel všechno mít to víme z dávných knih
jeho vojska by nikdo nepřehlíd, kdyby šel týden kolem nich \s
říkali mu farao a z knížat pozemských ten nejmocnější byl
až jeden člověk hlavu zdvih a jeho vůli se postavil \s
říkali mu mojžíš a měl tu výhodu, že se nebál o svou moc
svému lidu slíbil svobodu a k útěku zvolil noc \s

\R \[ měl mojžíš svobodu rád, měl rád, měl rád
   nechtěl se svobody vzdát to na žádnej pád.\]**

říkali mu mojžíš a jako lodivod vedl k moři národ svůj
za ním se vojsko dalo na pochod a farao volal stůj! \s
co udělal mojžíš? holí udeřil, vlny se zvedly v mocný val
na jednom břehu mojžíš byl, na druhém zuřil král \s

\r svůj trůn měl farao rád...

jen žhavý písek žhavá poušť a žhavý vzduch tam byl
pod žhavým sluncem chudý lid a řeka jménem nil \s
ten příběh dávno odvál čas, jen krále neodvál
kus mojžíše je v každém z nás, tak jak to bude dál?



\song{zpívám a meč svůj v ruce mám}{}

\R \[ /C zpívám /G a meč /F svůj v /G ruce /{C (a)} mám. \](4x) **

mou /C píseň se uč a /a nesmíš se bát
/F já víc takových /C znám
ten, kdo s druhým zpívá, /E není /a sám
já meč v ruce má/F m \S

bezmocné, bídné, ponížené
zástupy z nemanic
svůj hněv křičte s námi z plných plic
a bude nás víc \s

stará jak lidstvo je naše víra
den účtování vin
kdy náš věčný hněv se změní v čin
můj meč vrhne stín \s



\song{trh ve scarborough}{}

/d příteli máš do /C scarborough /d jít
/F dobře /d vím, že půj/(G) deš tam /d rád
tam dívku /F najdi na market /C street
/d co chtěla /G dřív /B mou /C ženou se /d stát \S

vzkaž jí, ať šátek začne mi šít
za jehlu rýč však smí jenom brát
a místo příze měsíční svit
bude-li chtít mou ženou se stát \s

až přijde máj a zavoní zem
šátek v písku přikaž jí prát
a ždímat v kvítku jabloňovém
bude-li chtít mou ženou se stát \s

z vrkočů svých ať uplete člun
v něm se může na cestu dát
s tím šátkem pak ať vejde v můj dům
bude-li chtít mou ženou se stát \s

kde útes ční nad přívaly vln
zorej dva sáhy pro růží sad
za pluh ať slouží šípkový trn
budeš-li chtít mým mužem se stát \s

osej ten sad a slzou jej skrop
choď těm růžím na loutnu hrát
až začnou kvést tak srpu se chop
budeš-li chtít mým mužem se stát \s

z trní si lůžko zhotovit dej
druhé z růží pro mě nech stlát
jen pýchy své a boha se ptej
proč nechci víc tvou ženou se stát



\song{dokud se zpívá}{nohavica}

z /C těšína /e vyjíždí /d7 vlaky co /F čtvrthodi/{C e d7 G} nu
/C včera jsem /e nespal a /d7 ani dnes /F nespoči/{C e d7 G} nu
/F svatý me/G dard, můj pa/C tron, ťuká /a si na če/G lo
ale /F dokud se /G zpívá, /F ještě se /G neumře/C lo, /{e d7 G} hóhó \S

ve stánku koupím si housku a slané tyčky
srdce mám pro lásku a hlavu pro písničky
ze školy dobře vím, co by se dělat mělo
ale dokud se zpívá, ještě se neumřelo, hóhó \s

do alba jízdenek lepím si další jednu
vyjel jsem před chvílí, konec je v nedohlednu
za oknem míhá se život jak leporelo
ale dokud se zpívá, ještě se neumřelo, hóhó \s

stokrát jsem prohloupil a stokrát platil draze
houpe to, houpe to na housenkové dráze
i kdyby supi se slítali na mé tělo
tak dokud se zpívá, ještě se neumřelo, hóhó \s

z těšína vyjíždí vlaky až na kraj světa
zvedl jsem telefon a ptám se:``lidi, jste tam?''
a z veliké dálky do uší mi zaznělo
\[ že dokud se zpívá, ještě se neumřelo \]



\song{omnia vincit amor}{klíč}

F C d a B C d
/d šel pocestný kol /C hospodských /d zdí
/F přisedl k nám a /C lokálem /F zní
/g pozdrav jak svaté /F přikázá/C ní
/d omnia /C vincit /d amor \s

hej, šenkýři, dej plný džbán
ať chasa ví, kdo k stolu je zván
se mnou ať zpívá, kdo za své vzal
omnia vincit amor \s

\R zlaťák /F pálí, /C nesleví /d nic
   štěstí v /F lásce /C znamená /F víc
   všechny /g pány /F ať vezme /{C A} ďas
   /d omnia /C vincit /d amor **

já viděl zemi válkou se chvět
musel se bít a nenávidět
v plamenech pálit prosby a pláč
omnia vincit amor \s

zlý trubky troubí, vítězí zášť
nad lidskou láskou roztáhli plášť
v tom kdosi krví napsal ten vzkaz
omnia vincit amor \s

\r zlaťák...

já prošel každou z nejdelších cest
všude se ptal, co značí ta zvěst
až řekl moudrý, pochopíš sám
omnia vincit amor (všechno přemáhá láska) \s

\rr

teď s novou vírou obcházím svět
má hlava zšedla pod tíhou let
každého zdravím větou všech vět
omnia vincit amor



\song{dva roky prázdnin}{karel černoch}

/a vzhůru na palubu, /D dálky /a volají
/a vítr už příhodný /G vane /a nám
/a tajemné příběhy /D nás teď /a čekají
/a tvým domovem bude /G oce/a án

\R  /F v ráhnoví plachty /C vítr nadouvá
    /F žene loď v širou /C dál
    /d kolébá boky /a plachetnice
    /d jak by si s ní jenom /E hrál
    /F posádku ani /C škuner neleká
    /F bouře ni ura/C gán
    /d přítomnost země /a oznámí nám
    /d příletem /E kormo/a rán **

náš ostrov vzdálený, z vln se vynoří
z příboje snů našich, pustý kraj
zátoku písčitou úsvit odhalí
háj palem, útesy bílých skal

\R  příď krájí vlny i tvůj čas
    srdce tvé tluče rázně
    nástrahy moře, nebezpečí
    s přáteli zvládneš vždy snáz
    v přátelství najdeš pevnou hráz
    zbaví tě smutku, bázně
    zítra až naše cesta skončí
    staneš se jedním z nás **



\song{grónská písnička}{jaromír nohavica}

/D daleko /e na severu /A je grónská /D zem
žije tam /e eskymačka s /A eskymá/D kem
\[ /D my bychom /e umrzli jim /G není zi/D ma
snídají /e nanuky /A a eskym/D a \]\S

mají se bezvadně vyspí se moc
půl roku trvá tam polární noc
\[ na jaře vzbudí se a vyběhnou ven
půl roku trvá tam polární den \]\s

když sněhu napadne nad kotníky
hrávají s medvědy na četníky
\[ medvědi těžko jsou k poražení
neboť medvědy ve sněhu vidět není \]\s

pokaždé ve středu přesně ve dvě
zaklepe na íglů hlavní medvěd
\[ dobrý den mohu dál na vteřinu
já nesu vám trochu ryb na svačinu \]\s

v kotlíku bublá čaj, kamna hřejí
psi venku hlídají před zloději
\[ smíchem se otřásá celé íglů
medvěd jim předvádí spoustu fíglů \]\s

tak žijou vesele na severu
srandu si dělají z teploměrů
\[ my bychom umrzli jim není zima
neboť jsou doma a mezi svýma \]



\song{hlídač krav}{jaromír nohavica}

/D když jsem byl malý říkali mi naši
/ dobře se uč a jez chytrou kaši
/G až jednou vyrosteš /A budeš doktorem /D práv \S

takový doktor sedí pěkně v suchu
bere velký peníze a škrábe se v uchu
já jim ale na to řek chci být hlídačem krav \s

já chci mít čapku s bambulí nahoře
jíst kaštany mýt se v lavoře
od rána po celý den
zpívat si jen
zpívat si pam pam padam pam ... \s

k vánocům mi kupovali hromady knih
co jsem ale vědět chtěl to nevyčet jsem z nich
nikde jsem se nedozvěděl jak se hlídají krávy \s

ptal jsem se starších a ptal jsem se všech
každý na mě hleděl jako na pytel blech
každý se mě opatrně tázal na moje zdraví \s

já chci mít čapku s bambulí nahoře
jíst kaštany mýt se v lavoře
od rána po celý den
zpívat si jen
zpívat si pam pam padam pam ... \s

dnes už jsem starší a vím co vím
mnohé věci nemůžu a mnohé smím
a když je mi velmi smutno lehnu do mokré trávy \s

s nohama křížem a rukama za hlavou
koukám nahoru na oblohu modravou
kde se mezi mraky honí moje strakaté krávy \s

já chci mít čapku s bambulí nahoře
jíst kaštany mýt se v lavoře
od rána po celý den
zpívat si jen
zpívat si pam pam padam pam...



\song{kluziště}{karel plíhal}

/C strejček /C\^H kovář /C\^A chytil /C\^G kleště, /F\^{maj7} uštíp' z /C noční /{F\^{maj7} G} oblohy
/C jednu /C\^H malou /C\^A kapku /C\^G deště /F\^{maj7} a ta mu /C spadla /{F\^{maj7} G} pod nohy
/C nejdřív /C\^H ale /C\^A chytil /C\^G slinu, /F\^{maj7} tak šáh' /C kamsi /{F\^{maj7} G} pro pivo
/C pak při/C\^H táhl /C\^A kova/C\^G dlinu /{F\^{maj7} C} a obrovský /{F\^{maj7} G} kladivo

\R  zatím /C tři bílé /C\^H vrány /C\^A pěkně za se/C\^G bou
    kolem /F\^{maj7} jdou, někam /C jdou, do ry/D7 tmu se kýva/G jí
    tyhle /C tři bílé /C\^H vrány /C\^A pěkně za se/C\^G bou
    kolem /F\^{maj7} jdou, někam /C jdou, nedo/F\^{maj7} jdou, nedo/C jdou **

vydal z hrdla mocný pokřik ztichlým letním večerem
pak tu kapku všude rozstřík' jedním mocným úderem
celej svět byl náhle v kapce a vysoko nad námi
na obrovské mucholapce visí nebe s hvězdami

\r zatím tři bílé vrány...

zpod víček mi vytrysk' pramen na zmačkané polštáře
kdosi mě vzal kolem ramen a políbil na tváře
kdesi v dálce rozmazaně strejda kovář odchází
do kalhot si čistí dlaně umazané od sazí

\r zatím...



\song{lachtani}{jaromír nohavica}

\R  \[ /C lach lach /F jé /C jé, /a lach lach /G jé /C jé \] **

/C jedna lachtaní /F rodi/C na
/a rozhodla se, že si vyjde /G do ki/C na
jeli vlakem, metrem, lodí a pak /F tramva/C jí
a teď /a u kina vesmír /G lachta/C jí
/G lachtaní úspory /C dali dohromady
/G koupili si lístky /C do první /G řady
/C táta lachtan řekl:``nebudem t/F řít bí/C du''
a /a pro každého koupil pytlík /G araši/C dů, ó \S

\r  lach lach...

na jižním pólu je nehezky
a tak lachtani si vyjeli na grotesky
těšili se, jak bude veselo
když zazněl gong a v sále se setmělo
co to ale vidí jejich lachtaní zraky:
sníh a mráz a sněhové mraky
pro veliký úspěch změna programu
dnes dáváme film ze života lachtanů, ó

\rr

táta lachtan vyskočil ze sedadla
nevídaná zlost ho popadla:
``proto jsem se netrmácel přes celý svět
abych tady v kině mrznul jako turecký med
tady zima, doma zima, všude jen chlad
kde má chudák lachtan relaxovat?''
nedivte se té lachtaní rodině
že pak rozšlapala arašidy po kině, jé

\rr

/C tahle lachtaní /F rodi/C na
/a od té doby nechodí už /G do ki/C na, jé



\song{mlýny}{spirituál kvintet}

\R /G slyším mlýnský kámen, jak se otá/G7 čí
   /C slyším mlýnský kámen, jak se otá/G čí
   já /G slyším mlýnský kámen, /H7 jak se otá/e čí
   /C ot/D áč/G í, otá/D čí, otá/C čí **

ty mlýny /G melou celou /C noc, melou /G celý den
melou /C7 bez výhod a melou /G stejně všem
melou doleva, /C melou /G doprava
melou /A pravdu i lež, když zrovna /D vyhrává
melou /G otrok/C áře, melou /G otroky
melou /C7 na minuty, na hodiny, /G na roky
melou /H7 pomalu a jistě, ale /e melou /C včas
já už /G slyším /D7 jejich /G hlas \S

\r slyším...

ó, já chtěl bych aspoň na chvíli být mlynářem
pane já bych mlel, až by se chvěla zem
to mi věřte, uměl bych dobře mlít
já bych věděl, komu ubrat, komu přitlačit
ty mlýny čekají někde nad námi
až zdola zazní naše volání
až zazní jeden lidský hlas
no tak už melte, je čas! \s

\r slyším...



\song{není nutno}{jaroslav uhlíř, zdeněk svěrák}

\R  /D není nutno, není nutno, aby bylo přímo vese/e lo
    /A7 hlavně nesmí býti smutno, natož aby se breče/{D A7} lo \s
    chceš-li trap se, že ti v kapse zlaté mince nechřestí
    nemít žádné kamarády, tomu já říkám neštěstí **

nemít /h prachy --- /D nevadí
nemít /h srdce --- /D vadí
zažít /h krachy --- /D nevadí
zažít /h nudu --- j/G ó, to /A vadí \S

\r není nutno...



\song{zahrada ticha}{jakub smolík}

/D je tam brána zdobená
cestu oteví/e rá
zahradu zele/C nou
/G všechno připomí/D ná \s

jako dým závojů
mlhou upředených
vstupuješ do ticha
cestou vyvolených \s

je to březový háj
je to borový les
je to anglický park
je to hluboký vřes \s

je to samota dnů
kdy jsi pomalu zrál
v zahradě zelený
kde sis za dětství hrál \s

kolik chceš, tolik máš
očí otevřených
tam venku za branou
leží studený sníh \s

z počátku uslyšíš
vítr a ptačí hlas
v zahradě zelený
přejdou do ticha zas \s

světlo připomíná
rána slunečných dnů
v zahradě zelený
v zahradě beze snů \s

uprostřed závratí
sluncem prosvícených
vstupuješ do ticha
cestou vyvolených



\song{the sound of silence}{simon and garfunkel}

/h hello, darkness, my old /A friend
i've come to talk with you /h again
because a vision softl/G y cree/D ping
left its seeds while i w/G as slee/D ping
and the /G vision that was planted in my /D brain
still re/h mains, /D within the /A sounds of /h silence \S

in restless dreams i walked alone
narrow streets of cobblestone
'neath the halo of a street lamp
i turned my collar to the cold and damp
when my eyes were stabbed by the flash of a neon light
that split the night and touched the sound of silence \s

and in the naked light i saw
ten thousand people maybe more
people talking without speaking
people hearing without listening
people writing songs that voices never shared
and no one dare disturb the sound of silence \s

``fools,'' said i, ``you do not know
silence like a cancer grows.''
``hear my words that i might teach you
take my arms that i might reach you.''
but my words like silent rain-drops fell
and echoed in the wells of silence \s

and the people bowed and prayed
to the neon god they made
and the sign flashed out its warning
in the words that it was forming
and the sign said: ``the words of the prophets are written on \rest{the subway walls}
and tenement halls
and whispered in the sounds of silence.''



\song{batalion}{}

\R  /e víno /G máš a /D marky/e tánku
    /G dlouhá noc /D se /e pro/h hý/e ří
   /e víno /G máš a /D chvilku /e spánku
    /G díky, dí/D ky /e ver/h bí/e ři **

/e dříve než se rozední, kapitán k /G osedlání /D rozkaz /e dá/h vá
/e ostruhami do slabin /{G D} koně /e po/h há/e ní
/e tam na straně polední čekají /G ženy zlaťá/D ky a /e slá/h va
/e do výstřelů karabin /G zvon /D už /e vy/h zvá/e ní \S

\R víno na kuráž a pomilovat markytánku
   zítra do burgund batalion zamíří
   víno na kuráž a k ránu dvě hodiny spánku
   díky, díky vám královští verbíři **

rozprášen je batalion, poslední vojáci se k zemi hroutí
na polštáři z kopretin budou věčně spát
neplač sladká marion, verbíři nové chlapce přivedou ti
za královský hermelín padne každý rád \s

\r víno na kuráž...

\r víno máš...



\konec{obsah}
\bye
