\input cantar.sty
\zacatek{zpěvník}{}



\song{tichá domácnost}{ebenové}

/h7 není doma /E vždycky všechno /D\^E tak
/Amaj7 jak by si člověk /Dmaj7 představoval
/h7 někdy to de pr/E ávě nao/D\^E pak
s /Amaj7 tím bych vás nerad /Dmaj7 unavoval

\R  /h7 u nás se nekři/cis čí, /fis u nás se nespí/cis lá
    /h7 u nás je zvláštní /E idyla

    /A tichá, /D tichá /E naše /Dmaj7 domácnost je /A\^{Cis} tichá, /D tichá
    a s /E konverzací /D nikdo nepos/A píchá
    /D tichá, /E naše /Dmaj7 domácnost /A\^{Cis} tichá, /D tichá
    jen /E vodovodní /D kohout tiše vzd/cis ychá, /fis tichá
    je /G poz/D dě /G hon/D it /G bych/A a **

není doma vždycky všechno tak
jak by to žena měla v plánu
celý večer čekáte a pak:
on přijde o půl páté k ránu

\r u nás se nekřičí...



\song{pískající cikán}{}

/G dívka /a loudá se /G vini/D cí
/G tam, kde z/a ídka je /h níz/D7 ká
/G tam, kde str/a áň končí /h voní/C cí
si /G písni/C čku někdo /{G C D} píská \S

ohlédne se a ``pro pána!''
v stínu, kde stojí líska
švarného vidí cikána
jak leží, písničku píská \s

chvíli tam stojí potichu
písnička si ji získá
domů jdou spolu do sklípku
je slyšet cikán jak píská \s

jenže tatík jak vidí cikána
pěstí do stolu tříská
ať táhne pryč vesta odraná
groš nemá, něco ti zpíská \s

teď smutnou dceru má u vrátek
jen bůh ví jak se jí stýská
``kéž vrátí se mi zas nazpátek
ten, který v dálce si píská!''\s

a šídla honí se na louce
v trávě rosa se blýská
cikán v rozmarným klobouce
jde dál a písničku píská \s

na závěr zbývá už jenom říct
v čem je ten kousek štístka
peníze často nejsou nic
má víc, kdo po svém si píská



\song{nezacházej, slunce}{žalman}

/G nezacházej, slunce, neza/a cházej ještě
/D já mám potěšení /C na da/D lekej cestě
/G já mám potě/{h e} šení    /a na da/D lekej ces/G tě \S

já má potěšení mezi hory-doly
\[ žádnej neuvěří, co je mezi námi \] \s

mezi náma dvouma láska nejstálejší
\[ a ta musí trvat do smrti nejdelší \] \s

trvej, lásko, trvej, nepřestávaj trvat
\[ až budou skřivánci o půlnoci zpívat \] \s

skřivánci zpívali, můj milej nepřišel
\[ on se na mě hněvá, nebo za jinou šel \]



\Song{hvězda na vrbě}{karel mareš -- jiří štaidl}

kdo se /a večer /e hájem /a vrací, /F ten ať /e klop/G7 í zra/{C e} ky
ať /G je /a nikd/e y neo/a brac/d í k vr/G7 bě, /e křivola/{E A C e} ký
jin/G ak /a jeho /e oči /a zjistí, /F i když /e se to /G7 nezd/{C e} á
že /G na /a vrbě, /e kromě /a listí, /d visí /F malá hvěz/a da \S

viděli /C jsme jednou v /F lukách, plakat /C na tý vrbě /F kluka
který /d pevně věřil /B tomu, že ji /D7 sundá z toho /{G7 e} stromu \S

kdo o hvězdy jeví zájem, zem, když večer hladne
ať jde klidně někdy hájem, hvězda někde spadne
ať se pro ni rosou brodí a pak vrbu najde si
a pro ty, co kolem chodí, na tu větev zavěsí



\song{husličky}{}

\[ /D čiže jste husličky, /{G D} čije
/e kdo vás tu /D zane/A chal \]
/e na trávě /A pová/D lané, /e na trávě /A pová/{D fis} lané
/e u paty /D oře/A ch/{e D A} a \S

\[ kdože tu trávu tak zválal
aj modré fialy \]
\[ že ste husličky samé \]
na trávě ostaly \s

kery tu muzikant usnul
co sa mu přišlo zdát
kery tu mládenec usnul
co sa mu přišlo zdát
co sa mu enem zdálo
bože co sa mu v noci zdálo
že už dál nechtěl hrát \s

\[ zahrajte husličky samy
zahrajte zvesela \]
až sa tá bude trápit
bože až sa tá bude trápit
\[ kerá ho nechtěla \]



\song{farao}{}

jen /D žhavý písek žhavá poušť a /G žhavý vzduch tam /D byl
pod žhavým sluncem chudý lid a /G řeka jménem /D nil
ti lidé byli otroci, nad /G nimi /C kara/G báč /D stál
nad karabáčem velekněz a /h nade /A všemi /D král \S

říkali mu farao a byl tak vysoko, že nevrhal ani stín
zato měl sýpky, sklepy, zlato a lesklá těla otrokyň \s
říkali mu farao a ten když rukou hnul duněly bubny k~obřadům
tančily kněžky hlavy padaly to aby nepad jeho trůn \s

\R \[ svůj /D trůn měl farao rád, měl /A rád, měl /D rád
   nechtěl ho nikomu dát to na žá/A dnej /D pád. \]**

říkali mu farao a musel všechno mít to víme z dávných knih
jeho vojska by nikdo nepřehlíd, kdyby šel týden kolem nich \s
říkali mu farao a z knížat pozemských ten nejmocnější byl
až jeden člověk hlavu zdvih a jeho vůli se postavil \s
říkali mu mojžíš a měl tu výhodu, že se nebál o svou moc
svému lidu slíbil svobodu a k útěku zvolil noc \s

\R \[ měl mojžíš svobodu rád, měl rád, měl rád
   nechtěl se svobody vzdát to na žádnej pád.\]**

říkali mu mojžíš a jako lodivod vedl k moři národ svůj
za ním se vojsko dalo na pochod a farao volal stůj! \s
co udělal mojžíš? holí udeřil, vlny se zvedly v mocný val
na jednom břehu mojžíš byl, na druhém zuřil král \s

\r svůj trůn měl farao rád...

jen žhavý písek žhavá poušť a žhavý vzduch tam byl
pod žhavým sluncem chudý lid a řeka jménem nil \s
ten příběh dávno odvál čas, jen krále neodvál
kus mojžíše je v každém z nás, tak jak to bude dál?



\song{zpívám a meč svůj v ruce mám}{}

\R \[ /C zpívám /G a meč /F svůj v /G ruce /{C (a)} mám. \](4x) **

mou /C píseň se uč a /a nesmíš se bát
/F já víc takových /C znám
ten, kdo s druhým zpívá, /E není /a sám
já meč v ruce má/F m \S

bezmocné, bídné, ponížené
zástupy z nemanic
svůj hněv křičte s námi z plných plic
a bude nás víc \s

stará jak lidstvo je naše víra
den účtování vin
kdy náš věčný hněv se změní v čin
můj meč vrhne stín \s



\song{trh ve scarborough}{}

/d příteli máš do /C scarborough /d jít
/F dobře /d vím, že půj/(G) deš tam /d rád
tam dívku /F najdi na market /C street
/d co chtěla /G dřív /B mou /C ženou se /d stát \S

vzkaž jí, ať šátek začne mi šít
za jehlu rýč však smí jenom brát
a místo příze měsíční svit
bude-li chtít mou ženou se stát \s

až přijde máj a zavoní zem
šátek v písku přikaž jí prát
a ždímat v kvítku jabloňovém
bude-li chtít mou ženou se stát \s

z vrkočů svých ať uplete člun
v něm se může na cestu dát
s tím šátkem pak ať vejde v můj dům
bude-li chtít mou ženou se stát \s

kde útes ční nad přívaly vln
zorej dva sáhy pro růží sad
za pluh ať slouží šípkový trn
budeš-li chtít mým mužem se stát \s

osej ten sad a slzou jej skrop
choď těm růžím na loutnu hrát
až začnou kvést tak srpu se chop
budeš-li chtít mým mužem se stát \s

z trní si lůžko zhotovit dej
druhé z růží pro mě nech stlát
jen pýchy své a boha se ptej
proč nechci víc tvou ženou se stát



\song{dokud se zpívá}{nohavica}

z /C těšína /e vyjíždí /d7 vlaky co /F čtvrthodi/{C e d7 G} nu
/C včera jsem /e nespal a /d7 ani dnes /F nespoči/{C e d7 G} nu
/F svatý me/G dard, můj pa/C tron, ťuká /a si na če/G lo
ale /F dokud se /G zpívá, /F ještě se /G neumře/C lo, /{e d7 G} hóhó \S

ve stánku koupím si housku a slané tyčky
srdce mám pro lásku a hlavu pro písničky
ze školy dobře vím, co by se dělat mělo
ale dokud se zpívá, ještě se neumřelo, hóhó \s

do alba jízdenek lepím si další jednu
vyjel jsem před chvílí, konec je v nedohlednu
za oknem míhá se život jak leporelo
ale dokud se zpívá, ještě se neumřelo, hóhó \s

stokrát jsem prohloupil a stokrát platil draze
houpe to, houpe to na housenkové dráze
i kdyby supi se slítali na mé tělo
tak dokud se zpívá, ještě se neumřelo, hóhó \s

z těšína vyjíždí vlaky až na kraj světa
zvedl jsem telefon a ptám se:``lidi, jste tam?''
a z veliké dálky do uší mi zaznělo
\[ že dokud se zpívá, ještě se neumřelo \]



\song{omnia vincit amor}{klíč}

F C d a B C d
/d šel pocestný kol /C hospodských /d zdí
/F přisedl k nám a /C lokálem /F zní
/g pozdrav jak svaté /F přikázá/C ní
/d omnia /C vincit /d amor \s

hej, šenkýři, dej plný džbán
ať chasa ví, kdo k stolu je zván
se mnou ať zpívá, kdo za své vzal
omnia vincit amor \s

\R zlaťák /F pálí, /C nesleví /d nic
   štěstí v /F lásce /C znamená /F víc
   všechny /g pány /F ať vezme /{C A} ďas
   /d omnia /C vincit /d amor **

já viděl zemi válkou se chvět
musel se bít a nenávidět
v plamenech pálit prosby a pláč
omnia vincit amor \s

zlý trubky troubí, vítězí zášť
nad lidskou láskou roztáhli plášť
v tom kdosi krví napsal ten vzkaz
omnia vincit amor \s

\r zlaťák...

já prošel každou z nejdelších cest
všude se ptal, co značí ta zvěst
až řekl moudrý, pochopíš sám
omnia vincit amor (všechno přemáhá láska) \s

\rr

teď s novou vírou obcházím svět
má hlava zšedla pod tíhou let
každého zdravím větou všech vět
omnia vincit amor



\song{dva roky prázdnin}{karel černoch}

/a vzhůru na palubu, /D dálky /a volají
/a vítr už příhodný /G vane /a nám
/a tajemné příběhy /D nás teď /a čekají
/a tvým domovem bude /G oce/a án

\R  /F v lanoví plachty /C vítr nadouvá
    /F žene loď v širou /C dál
    /d kolébá boky /a plachetnice
    /d jak by si s ní jenom /E hrál
    /F posádku ani /C škuner neleká
    /F bouře ni ura/C gán
    /d přítomnost země /a oznámí nám
    /d příletem /E kormo/a rán **

náš ostrov vzdálený, z vln se vynoří
z příboje snů našich, pustý kraj
zátoku písčitou úsvit odhalí
háj palem, útesy bílých skal

\R  příď krájí vlny i tvůj čas
    srdce tvé tluče rázně
    nástrahy moře, nebezpečí
    s přáteli zvládneš vždy snáz
    v přátelství najdeš pevnou hráz
    zbaví tě smutku, bázně
    zítra až naše cesta skončí
    staneš se jedním z nás **



\song{grónská písnička}{jaromír nohavica}

/D daleko /e na severu /A je grónská /D zem
žije tam /e eskymačka s /A eskymá/D kem
\[ /D my bychom /e umrzli jim /G není zi/D ma
snídají /e nanuky /A a eskym/D a \]\S

mají se bezvadně vyspí se moc
půl roku trvá tam polární noc
\[ na jaře vzbudí se a vyběhnou ven
půl roku trvá tam polární den \]\s

když sněhu napadne nad kotníky
hrávají s medvědy na četníky
\[ medvědi těžko jsou k poražení
neboť medvědy ve sněhu vidět není \]\s

pokaždé ve středu přesně ve dvě
zaklepe na íglů hlavní medvěd
\[ dobrý den mohu dál na vteřinu
já nesu vám trochu ryb na svačinu \]\s

v kotlíku bublá čaj, kamna hřejí
psi venku hlídají před zloději
\[ smíchem se otřásá celé íglů
medvěd jim předvádí spoustu fíglů \]\s

tak žijou vesele na severu
srandu si dělají z teploměrů
\[ my bychom umrzli jim není zima
neboť jsou doma a mezi svýma \]



\song{hlídač krav}{jaromír nohavica}

/D když jsem byl malý říkali mi naši
/ dobře se uč a jez chytrou kaši
/G až jednou vyrosteš /A budeš doktorem /D práv \S

takový doktor sedí pěkně v suchu
bere velký peníze a škrábe se v uchu
já jim ale na to řek chci být hlídačem krav \s

já chci mít čapku s bambulí nahoře
jíst kaštany mýt se v lavoře
od rána po celý den
zpívat si jen
zpívat si pam pam padam pam ... \s

k vánocům mi kupovali hromady knih
co jsem ale vědět chtěl to nevyčet jsem z nich
nikde jsem se nedozvěděl jak se hlídají krávy \s

ptal jsem se starších a ptal jsem se všech
každý na mě hleděl jako na pytel blech
každý se mě opatrně tázal na moje zdraví \s

já chci mít čapku s bambulí nahoře
jíst kaštany mýt se v lavoře
od rána po celý den
zpívat si jen
zpívat si pam pam padam pam ... \s

dnes už jsem starší a vím co vím
mnohé věci nemůžu a mnohé smím
a když je mi velmi smutno lehnu do mokré trávy \s

s nohama křížem a rukama za hlavou
koukám nahoru na oblohu modravou
kde se mezi mraky honí moje strakaté krávy \s

já chci mít čapku s bambulí nahoře
jíst kaštany mýt se v lavoře
od rána po celý den
zpívat si jen
zpívat si pam pam padam pam...



\song{kluziště}{karel plíhal}

/C strejček /C\^H kovář /C\^A chytil /C\^G kleště, /F\^{maj7} uštíp' z /C noční /{F\^{maj7} G} oblohy
/C jednu /C\^H malou /C\^A kapku /C\^G deště /F\^{maj7} a ta mu /C spadla /{F\^{maj7} G} pod nohy
/C nejdřív /C\^H ale /C\^A chytil /C\^G slinu, /F\^{maj7} tak šáh' /C kamsi /{F\^{maj7} G} pro pivo
/C pak při/C\^H táhl /C\^A kova/C\^G dlinu /{F\^{maj7} C} a obrovský /{F\^{maj7} G} kladivo

\R  zatím /C tři bílé /C\^H vrány /C\^A pěkně za se/C\^G bou
    kolem /F\^{maj7} jdou, někam /C jdou, do ry/D7 tmu se kýva/G jí
    tyhle /C tři bílé /C\^H vrány /C\^A pěkně za se/C\^G bou
    kolem /F\^{maj7} jdou, někam /C jdou, nedo/F\^{maj7} jdou, nedo/C jdou **

vydal z hrdla mocný pokřik ztichlým letním večerem
pak tu kapku všude rozstřík' jedním mocným úderem
celej svět byl náhle v kapce a vysoko nad námi
na obrovské mucholapce visí nebe s hvězdami

\r zatím tři bílé vrány...

zpod víček mi vytrysk' pramen na zmačkané polštáře
kdosi mě vzal kolem ramen a políbil na tváře
kdesi v dálce rozmazaně strejda kovář odchází
do kalhot si čistí dlaně umazané od sazí

\r zatím...



\song{lachtani}{jaromír nohavica}

\R  \[ /C lach lach /F jé /C jé, /a lach lach /G jé /C jé \] **

/C jedna lachtaní /F rodi/C na
/a rozhodla se, že si vyjde /G do ki/C na
jeli vlakem, metrem, lodí a pak /F tramva/C jí
a teď /a u kina vesmír /G lachta/C jí
/G lachtaní úspory /C dali dohromady
/G koupili si lístky /C do první /G řady
/C táta lachtan řekl:``nebudem t/F řít bí/C du''
a /a pro každého koupil pytlík /G araši/C dů, ó \S

\r  lach lach...

na jižním pólu je nehezky
a tak lachtani si vyjeli na grotesky
těšili se, jak bude veselo
když zazněl gong a v sále se setmělo
co to ale vidí jejich lachtaní zraky:
sníh a mráz a sněhové mraky
pro veliký úspěch změna programu
dnes dáváme film ze života lachtanů, ó

\rr

táta lachtan vyskočil ze sedadla
nevídaná zlost ho popadla:
``proto jsem se netrmácel přes celý svět
abych tady v kině mrznul jako turecký med
tady zima, doma zima, všude jen chlad
kde má chudák lachtan relaxovat?''
nedivte se té lachtaní rodině
že pak rozšlapala arašidy po kině, jé

\rr

/C tahle lachtaní /F rodi/C na
/a od té doby nechodí už /G do ki/C na, jé



\song{mlýny}{spirituál kvintet}

\R /G slyším mlýnský kámen, jak se otá/G7 čí
   /C slyším mlýnský kámen, jak se otá/G čí
   já /G slyším mlýnský kámen, /H7 jak se otá/e čí
   /C ot/D áč/G í, otá/D čí, otá/C čí **

ty mlýny /G melou celou /C noc, melou /G celý den
melou /C7 bez výhod a melou /G stejně všem
melou doleva, /C melou /G doprava
melou /A pravdu i lež, když zrovna /D vyhrává
melou /G otrok/C áře, melou /G otroky
melou /C7 na minuty, na hodiny, /G na roky
melou /H7 pomalu a jistě, ale /e melou /C včas
já už /G slyším /D7 jejich /G hlas \S

\r slyším...

ó, já chtěl bych aspoň na chvíli být mlynářem
pane já bych mlel, až by se chvěla zem
to mi věřte, uměl bych dobře mlít
já bych věděl, komu ubrat, komu přitlačit
ty mlýny čekají někde nad námi
až zdola zazní naše volání
až zazní jeden lidský hlas
no tak už melte, je čas! \s

\r slyším...



\song{není nutno}{jaroslav uhlíř, zdeněk svěrák}

\R  /D není nutno, není nutno, aby bylo přímo vese/e lo
    /A7 hlavně nesmí býti smutno, natož aby se breče/{D A7} lo \s
    chceš-li trap se, že ti v kapse zlaté mince nechřestí
    nemít žádné kamarády, tomu já říkám neštěstí **

nemít /h prachy --- /D nevadí
nemít /h srdce --- /D vadí
zažít /h krachy --- /D nevadí
zažít /h nudu --- j/G ó, to /A vadí \S

\r není nutno...



\song{zahrada ticha}{jakub smolík}

/D je tam brána zdobená
cestu oteví/e rá
zahradu zele/C nou
/G všechno připomí/D ná \s

jako dým závojů
mlhou upředených
vstupuješ do ticha
cestou vyvolených \s

je to březový háj
je to borový les
je to anglický park
je to hluboký vřes \s

je to samota dnů
kdy jsi pomalu zrál
v zahradě zelený
kde sis za dětství hrál \s

kolik chceš, tolik máš
očí otevřených
tam venku za branou
leží studený sníh \s

z počátku uslyšíš
vítr a ptačí hlas
v zahradě zelený
přejdou do ticha zas \s

světlo připomíná
rána slunečných dnů
v zahradě zelený
v zahradě beze snů \s

uprostřed závratí
sluncem prosvícených
vstupuješ do ticha
cestou vyvolených



\song{the sound of silence}{simon and garfunkel}

/h hello, darkness, my old /A friend
i've come to talk with you /h again
because a vision softl/G y cree/D ping
left its seeds while i w/G as slee/D ping
and the /G vision that was planted in my /D brain
still re/h mains, /D within the /A sounds of /h silence \S

in restless dreams i walked alone
narrow streets of cobblestone
'neath the halo of a street lamp
i turned my collar to the cold and damp
when my eyes were stabbed by the flash of a neon light
that split the night and touched the sound of silence \s

and in the naked light i saw
ten thousand people maybe more
people talking without speaking
people hearing without listening
people writing songs that voices never shared
and no one dare disturb the sound of silence \s

``fools,'' said i, ``you do not know
silence like a cancer grows.''
``hear my words that i might teach you
take my arms that i might reach you.''
but my words like silent rain-drops fell
and echoed in the wells of silence \s

and the people bowed and prayed
to the neon god they made
and the sign flashed out its warning
in the words that it was forming
and the sign said: ``the words of the prophets are written on \rest{the subway walls}
and tenement halls
and whispered in the sounds of silence.''



\song{batalion}{}

\R  /e víno /G máš a /D marky/e tánku
    /G dlouhá noc /D se /e pro/h hý/e ří
   /e víno /G máš a /D chvilku /e spánku
    /G díky, dí/D ky /e ver/h bí/e ři **

/e dříve než se rozední, kapitán k /G osedlání /D rozkaz /e dá/h vá
/e ostruhami do slabin /{G D} koně /e po/h há/e ní
/e tam na straně polední čekají /G ženy zlaťá/D ky a /e slá/h va
/e do výstřelů karabin /G zvon /D už /e vy/h zvá/e ní \S

\R víno na kuráž a pomilovat markytánku
   zítra do burgund batalion zamíří
   víno na kuráž a k ránu dvě hodiny spánku
   díky, díky vám královští verbíři **

rozprášen je batalion, poslední vojáci se k zemi hroutí
na polštáři z kopretin budou věčně spát
neplač sladká marion, verbíři nové chlapce přivedou ti
za královský hermelín padne každý rád \s

\r víno na kuráž...

\r víno máš...



\konec{obsah}
\bye
